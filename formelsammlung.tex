\documentclass[a4paper, 11pt]{article}
\usepackage[a4paper, left=2cm, right=2cm, top=2cm]{geometry}
\usepackage[utf8]{inputenc}					% Zeichenkodierung UTF-8 falls Probleme wegen utf8 auftreten, utf8 durch utf8x ersetzen
\usepackage[ngerman]{babel}					% Deutsche Sprache und Silbentrennung
\usepackage{xcolor}
\usepackage{amsmath}						% erlaubt mathematische Formeln
\usepackage{amssymb}						% Verschiedene Symbole
\usepackage{graphicx}						% Zum Bilder einfügen benötigt
\usepackage{hyperref}						% Sprunglinks für Überschriften, Fußnoten und Weblinks
\usepackage{siunitx}                        % Zur Darstellung von SI Einheiten
\sisetup{
  locale = DE ,
  per-mode = symbol
}

\usepackage{parskip}                        % No Space after newline

\newcommand\mainformular[1]{\fbox{#1}}% Standard Formel
\definecolor{lightblue}{rgb}{204,230,255}

\begin{document}

\section{Mechanik}

\subsection{Grundlagen Mechanik}

\subsubsection{Gewichtskraft}
\begin{minipage}{0.45\textwidth}

\mainformular{$F_G = m \cdot g$} \\
$m = \cfrac{F_G}{g}$ \\
$g = \cfrac{F_G}{m}$

\end{minipage}
\begin{minipage}{0.45\textwidth}

\begin{tabular}{llll}
$m$ & Masse & $\si{\kilo\gram}$ & \\
$g$ & Fallbeschleunigung & $\si{\metre\per\square\second}$ & $9,81 \si{\metre\per\square\second}$ \\
$F_G$ & Gewichtskraft & $N$ & $\si{\kilogram\metre\per\square\second}$ \\
\end{tabular}

\end{minipage}

\subsubsection{Kräfte}
\begin{minipage}{0.45\textwidth}

\mainformular{$\overrightarrow{F}_{res} = \overrightarrow{F}_1 + \overrightarrow{F}_2 $} \\

\end{minipage}
\begin{minipage}{0.45\textwidth}

\begin{tabular}{llll}
$F_1$ & Einzelkraft 1 & $\si{\newton}$ & $\si{\kilo\gram}$ \\
$g$ & Fallbeschleunigung & $\si{\metre\per\square\second}$ & $9,81 \si{\metre\per\square\second}$ \\
$F_G$ & Gewichtskraft & $N$ & $\si{\kilogram\metre\per\square\second}$ \\
\end{tabular}

\end{minipage}

\subsubsection{Gewichtskraft}
\begin{minipage}{0.45\textwidth}

\mainformular{$F_G = m \cdot g$} \\
$m = \cfrac{F_G}{g}$ \\
$g = \cfrac{F_G}{m}$

\end{minipage}
\begin{minipage}{0.45\textwidth}

\begin{tabular}{llll}
$m$ & Masse & $\si{\kilo\gram}$ & \\
$g$ & Fallbeschleunigung & $\si{\metre\per\square\second}$ & $9,81 \si{\metre\per\square\second}$ \\
$F_G$ & Gewichtskraft & $N$ & $\si{\kilogram\metre\per\square\second}$ \\
\end{tabular}

\end{minipage}

\subsubsection{Gewichtskraft}
\begin{minipage}{0.45\textwidth}

\mainformular{$F_G = m \cdot g$} \\
$m = \cfrac{F_G}{g}$ \\
$g = \cfrac{F_G}{m}$

\end{minipage}
\begin{minipage}{0.45\textwidth}

\begin{tabular}{llll}
$m$ & Masse & $\si{\kilo\gram}$ & \\
$g$ & Fallbeschleunigung & $\si{\metre\per\square\second}$ & $9,81 \si{\metre\per\square\second}$ \\
$F_G$ & Gewichtskraft & $N$ & $\si{\kilogram\metre\per\square\second}$ \\
\end{tabular}

\end{minipage}

\subsubsection{Gewichtskraft}
\begin{minipage}{0.45\textwidth}

\mainformular{$F_G = m \cdot g$} \\
$m = \cfrac{F_G}{g}$ \\
$g = \cfrac{F_G}{m}$

\end{minipage}
\begin{minipage}{0.45\textwidth}

\begin{tabular}{llll}
$m$ & Masse & $\si{\kilo\gram}$ & \\
$g$ & Fallbeschleunigung & $\si{\metre\per\square\second}$ & $9,81 \si{\metre\per\square\second}$ \\
$F_G$ & Gewichtskraft & $N$ & $\si{\kilogram\metre\per\square\second}$ \\
\end{tabular}

\end{minipage}

\section{Wärmelehre}

\subsection{Körper und Temperaturänderung}
\subsubsection{Längenausdehnung bei Temperaturänderung}
\begin{minipage}{0.45\textwidth}
\mainformular{$\Delta l = \alpha \cdot l_0 \cdot \Delta t$}  \\
$\alpha = \cfrac{\Delta l}{l_0 \cdot \Delta t}$
\end{minipage}
\begin{minipage}{0.45\textwidth}

\begin{tabular}{llll}
$l$ & Länge & $\si{\meter} $ & $ $ \\
$\alpha$ & Längenausdehnungskoeffizient &  & \\
$t$ & Zeit & $\si{\second}$ & \\
\end{tabular}

\end{minipage}

\textbf{Resultierende Gesamtlänge} bei Längenausdehnung, proportional: \\
$l = l_{0} \cdot \left( 1 + \alpha \cdot \Delta t \right) $

\subsubsection{Dichte in Abhängigkeit der Temperatur}
\begin{minipage}{0.45\textwidth}
\mainformular{$\rho = \cfrac{\rho_0}{1 + \gamma \cdot \Delta t}$ }  \\

\end{minipage}
\begin{minipage}{0.45\textwidth}

\begin{tabular}{llll}
$\rho$ & Druck & $\si{\pascal} $ & $\si{\newton\per\square\metre}$ \\
$\gamma $ & Volumenausdehnungskoeffizient &  & $= 3 \alpha$ \\
$t$ & Zeit & $\si{\second}$ & \\
\end{tabular}

\end{minipage}

\subsubsection{Gay-Lussac}
 
\textbf{bei konstantem Druck:} \\

\begin{minipage}{0.45\textwidth}
\mainformular{$\cfrac{V_1}{V_2} = \cfrac{T_1}{T_2}$ }  \\

\end{minipage}
\begin{minipage}{0.45\textwidth}

\begin{tabular}{llll}
$V$ & Volumen & $\si{\cubic\meter}$ & \\
$T$ & Temperatur & $\si{\kelvin}$  &  \\
\end{tabular}

\end{minipage} 

\textbf{bei konstantem Volumen:}\\

\begin{minipage}{0.45\textwidth}
\mainformular{$\cfrac{\rho_1}{\rho_2} = \cfrac{T_1}{T_2}$ }  \\

\end{minipage}
\begin{minipage}{0.45\textwidth}

\begin{tabular}{llll}
$\rho$ & Druck & $\si{\pascal} $ & $\si{\newton\per\square\metre}$ \\
$T$ & Temperatur & $\si{\kelvin}$  &  \\
\end{tabular}

\end{minipage}

\subsection{Ideale Gase}

\subsubsection{Ideale Gasgleichung}
\begin{minipage}{0.45\textwidth}

\mainformular{$\rho \cdot V = m \cdot R \cdot T$} \\
\end{minipage}
\begin{minipage}{0.45\textwidth}

\begin{tabular}{llll}
$\rho$ & Druck & $\si{\pascal}$ & \\
$V$ & Volumen & $\si{\cubic\meter}$ & \\
$m$ & Masse & $\si{\kilogram}$ &  \\
$R$ & Spezifische Gaskonstante & $\si{\joule\per\kilogram\per\kelvin} $ & $ $ \\
$T$ & Temperatur & $\si{\kelvin}$ & \\
\end{tabular}

\end{minipage} 

\textbf{Universelle Gaskonstante}\\
\begin{minipage}{0.45\textwidth}
\mainformular{$R_{m} = R \cdot M$} 

\end{minipage}
\begin{minipage}{0.45\textwidth}

\begin{tabular}{llll}
$R$ & Spezifische Gaskonstante & $\si{\joule\per\kilogram\per\kelvin} $ & $ $ \\
$R_{m}$ & Universelle Gaskonstante & $\si{\joule\per\mol\per\kelvin} $ & $8,3144598 \si{\joule\per\mol\per\kelvin}$ \\
$M$ & Molare Masse & $\si{\gram\mol}$ & \\
\end{tabular}

\end{minipage}


\subsubsection{Isotherme Zustastandsänderung}
\begin{minipage}{0.45\textwidth}

\mainformular{$W = -m \cdot R \cdot T \cdot \ln{\cfrac{V_2}{V_1}}$}

$W = -m \cdot R \cdot T \cdot \ln{\cfrac{V_2}{V_1}}$ \\
$g = \cfrac{F_G}{m}$

\end{minipage}
\begin{minipage}{0.45\textwidth}

\begin{tabular}{llll}
$W$ & Arbeit & $\si{\joule}$ & \\
$m$ & Masse & $\si{\kilogram}$ &  \\
$R$ & Spezifische Gaskonstante & $\si{\joule\per\kilogram\per\kelvin} $ & $ $ \\
$T$ & Temperatur & $\si{} $ & \\
\end{tabular}

\end{minipage}

\subsection{1. Hauptsatz}
\subsubsection{Wärmelehre}
\begin{minipage}{0.45\textwidth}

\mainformular{$\Delta U + \Delta E_{pot} + \Delta E_{kin} =  \Delta Q + \Delta W$} \\
$m = \cfrac{F_G}{g}$ \\
$g = \cfrac{F_G}{m}$

\end{minipage}
\begin{minipage}{0.45\textwidth}
\noindent\fcolorbox{blue}{lightblue}{%
\begin{tabular}{llll}
$U$ & Innere Energie & $\si{\kilo\gram}$ & \\
$E_{pot}$ & Potentielle Energie & $\si{\metre\per\square\second}$ & $9,81 \si{\metre\per\square\second}$ \\
$E_{kin}$ & Kinetische Energie & $N$ & $\si{\kilogram\metre\per\square\second}$ \\
$Q$ & Wärmeenergie/Wärmeleistung & $\si{\joule}$ & \\
\end{tabular}
}%
\end{minipage}


\end{document}
