\documentclass[a4paper, 11pt]{article}
\usepackage[a4paper, left=2cm, right=2cm, top=2cm]{geometry}
\usepackage[utf8]{inputenc}					% Zeichenkodierung UTF-8 falls Probleme wegen utf8 auftreten, utf8 durch utf8x ersetzen
\usepackage[ngerman]{babel}					% Deutsche Sprache und Silbentrennung
\usepackage{xcolor}
\usepackage{amsmath}						% erlaubt mathematische Formeln
\usepackage{amssymb}						% Verschiedene Symbole
\usepackage{graphicx}						% Zum Bilder einfügen benötigt
\usepackage{hyperref}						% Sprunglinks für Überschriften, Fußnoten und Weblinks
\usepackage{siunitx}                        % Zur Darstellung von SI Einheiten
\sisetup{
  locale = DE ,
  per-mode = symbol
}
\usepackage{multicol}                       % Mehrspaltig

\usepackage{parskip}                        % No Space after newline

\newcommand\mainformular[1]{\fbox{#1}}      % Standard Formel
\newcommand\legende[1]{
    \colorbox{lightblue}{%
    \begin{tabular}{llll}
     #1
    \end{tabular}
    }
}                                           % Farbig hintelegte Legende
\definecolor{lightblue}{RGB}{204, 230, 255}

\begin{document}

\section{Mechanik}

\subsection{Grundlagen Mechanik}
\subsubsection{Dichte} 
\begin{minipage}{0.45\textwidth} 
$ \rho  = \frac{m}{V} $\\ 
$ m = \rho \cdot V $\\ 
$ V = \frac{m}{\rho } $\\ 
\end{minipage} 
\begin{minipage}{0.45\textwidth} 
 
\end{minipage} 
\subsubsection{Kräfte} 
\begin{minipage}{0.45\textwidth} 
\end{minipage} 
\begin{minipage}{0.45\textwidth} 
 
\end{minipage} 
\subsubsection{Auftrieb in Flüssigkeiten} 
\begin{minipage}{0.45\textwidth} 
$ F_{A}  =  \rho \cdot g\cdot V $\\ 
$ \rho  = \frac{F_{A} }{g\cdot V} $\\ 
$ V = \frac{F_{A} }{g \rho } $\\ 
\end{minipage} 
\begin{minipage}{0.45\textwidth} 
 
\end{minipage} 
\subsubsection{Druck} 
\begin{minipage}{0.45\textwidth} 
$ p = \frac{F}{A} $\\ 
$ F = p\cdot A $\\ 
$ A = \frac{F}{p} $\\ 
\end{minipage} 
\begin{minipage}{0.45\textwidth} 
 
\end{minipage} 
\subsubsection{Gewichtskraft} 
\begin{minipage}{0.45\textwidth} 
$ F_{G}  = m\cdot g $\\ 
$ m = \frac{F_{G} }{g} $\\ 
$ g = \frac{F_{G} }{m} $\\ 
\end{minipage} 
\begin{minipage}{0.45\textwidth} 
 
\end{minipage} 
\subsubsection{Hebelgesetz} 
\begin{minipage}{0.45\textwidth} 
$ F_{1}  = \frac{F_{2} \cdot l_{2} }{ l_{1} } $\\ 
$ l_{1}  = \frac{F_{2} \cdot l_{2} }{ F_{1} } $\\ 
\end{minipage} 
\begin{minipage}{0.45\textwidth} 
 
\end{minipage} 
\subsubsection{Schweredruck} 
\begin{minipage}{0.45\textwidth} 
$ p =  \rho \cdot g\cdot h $\\ 
$ \rho  = \frac{p}{g\cdot h} $\\ 
$ h = \frac{p}{g \rho } $\\ 
\end{minipage} 
\begin{minipage}{0.45\textwidth} 
 
\end{minipage} 
\subsubsection{Drehmoment} 
\begin{minipage}{0.45\textwidth} 
$ M = F\cdot l $\\ 
$ F = \frac{M}{l} $\\ 
$ l = \frac{M}{F} $\\ 
\end{minipage} 
\begin{minipage}{0.45\textwidth} 
 
\end{minipage} 
\subsubsection{Hookesches Gesetz} 
\begin{minipage}{0.45\textwidth} 
$ F = D\cdot s $\\ 
$ D = \frac{F}{s} $\\ 
$ s = \frac{F}{D} $\\ 
\end{minipage} 
\begin{minipage}{0.45\textwidth} 
 
\end{minipage} 
\subsubsection{Schiefe Ebene} 
\begin{minipage}{0.45\textwidth} 
$ F_{H}  = \frac{F_{G} \cdot h}{ l} $\\ 
$ F_{G}  = \frac{F_{H} \cdot l}{ h} $\\ 
$ h = \frac{F_{H} \cdot l}{ F_{G} } $\\ 
$ l = \frac{F_{G} \cdot h}{ F_{H} } $\\ 
$ F_{N}  = \frac{F_{G} \cdot b}{ l} $\\ 
$ F_{G}  = \frac{F_{N} \cdot l}{ b} $\\ 
$ b = \frac{F_{N} \cdot l}{ F_{G} } $\\ 
$ l = \frac{F_{G} \cdot b}{ F_{N} } $\\ 
\end{minipage} 
\begin{minipage}{0.45\textwidth} 
 
\end{minipage} 
\subsubsection{Reibung} 
\begin{minipage}{0.45\textwidth} 
$ F_{R}  = \mu \cdot F_{N} $\\ 
$ F_{N}  = \frac{F_{R} }{\mu } $\\ 
$ \mu  = \frac{F_{R} }{F_{N} } $\\ 
\end{minipage} 
\begin{minipage}{0.45\textwidth} 
 
\end{minipage} 
\subsubsection{Wichte} 
\begin{minipage}{0.45\textwidth} 
$ \gamma  = \frac{F_{G} }{V} $\\ 
$ F_{G}  = V\cdot \gamma $\\ 
$ V = \frac{F_{G} }{\gamma } $\\ 
\end{minipage} 
\begin{minipage}{0.45\textwidth} 
 
\end{minipage} 


\subsection{Dynamik}
\subsubsection{Mechanische Leistung} 
\begin{minipage}{0.45\textwidth} 
\mainformular{$ P = \frac{W}{t} $} 
\end{minipage} 
\begin{minipage}{0.45\textwidth} 
 
\legende{}\end{minipage} 
 
$ P = \frac{W}{t} $ - $ W = P\cdot t $ - $ t = \frac{W}{P} $ - \\ 
 
\subsubsection{Wirkungsgrad} 
\begin{minipage}{0.45\textwidth} 
\mainformular{$ \eta  = \frac{P_{2} }{P_{1} } $} 
\end{minipage} 
\begin{minipage}{0.45\textwidth} 
 
\legende{}\end{minipage} 
 
$ \eta  = \frac{P_{2} }{P_{1} } $ - $ P_{1}  = \frac{p_{2} }{\eta } $ - $ P_{2}  = \eta \cdot P_{1} $ - \\ 
 
\subsubsection{Beschleunigungsarbeit - kinetische Energie} 
\begin{minipage}{0.45\textwidth} 
\mainformular{$ W = \frac{1}{2}\cdot m\cdot v^{2} $} 
\end{minipage} 
\begin{minipage}{0.45\textwidth} 
 
\legende{}\end{minipage} 
 
$ W = \frac{1}{2}\cdot m\cdot v^{2} $ - $ m = \frac{2\cdot W}{ v^{2} } $ - $ v = \sqrt{\frac{2\cdot W}{ m}} $ - \\ 
 
\subsubsection{Spannarbeit-Spannenergie} 
\begin{minipage}{0.45\textwidth} 
\mainformular{$ W =\frac{1}{2}\cdot D\cdot s^{2} $} 
\end{minipage} 
\begin{minipage}{0.45\textwidth} 
 
\legende{}\end{minipage} 
 
$ W =\frac{1}{2}\cdot D\cdot s^{2} $ - $ s = \sqrt{\frac{2\cdot W}{ D}} $ - $ D =\frac{2\cdot W}{s^{2} } $ - \\ 
 
\subsubsection{Hubarbeit - Potentielle Energie} 
\begin{minipage}{0.45\textwidth} 
\mainformular{$ W = F_{G} \cdot h $} 
\end{minipage} 
\begin{minipage}{0.45\textwidth} 
 
\legende{}\end{minipage} 
 
$ W = F_{G} \cdot h $ - $ F_{G}  = \frac{W}{h} $ - $ h = \frac{W}{F_{G} } $ - \\ 
 
\subsubsection{Mechanische Arbeit} 
\begin{minipage}{0.45\textwidth} 
\mainformular{$ W = F\cdot s $} 
\end{minipage} 
\begin{minipage}{0.45\textwidth} 
 
\legende{}\end{minipage} 
 
$ W = F\cdot s $ - $ F = \frac{W}{s} $ - $ s = \frac{W}{F} $ - \\ 
 
\subsubsection{Impuls} 
\begin{minipage}{0.45\textwidth} 
\mainformular{$ p = m\cdot v $} 
\end{minipage} 
\begin{minipage}{0.45\textwidth} 
 
\legende{}\end{minipage} 
 
$ p = m\cdot v $ - $ m = \frac{p}{v} $ - $ v = \frac{p}{m} $ - \\ 
 
\subsubsection{Gravitationsgesetz} 
\begin{minipage}{0.45\textwidth} 
\mainformular{$ F = G \cdot  \frac{m_{1} \cdot m_{2} }{  r^{2} } $} 
\end{minipage} 
\begin{minipage}{0.45\textwidth} 
 
\legende{}\end{minipage} 
 
$ F = G \cdot  \frac{m_{1} \cdot m_{2} }{  r^{2} } $ - $ r = \sqrt{\frac{G\cdot m_{1} \cdot m_{2} }{   F}} $ - $ m_{1}  =  \frac{F\cdot r^{2} }{G\cdot m_{2} } $ - $ m_{2}  =  \frac{F\cdot r^{2} }{G\cdot m_{1} } $ - \\ 
 
\subsubsection{Zentralkraft} 
\begin{minipage}{0.45\textwidth} 
\mainformular{$ F_{z}  = m\cdot \omega ^{2} \cdot r $} 
\end{minipage} 
\begin{minipage}{0.45\textwidth} 
 
\legende{}\end{minipage} 
 
$ F_{z}  = m\cdot \omega ^{2} \cdot r $ - $ m = \frac{ F_{z} }{\omega ^{2} \cdot r} $ - $ \omega  = \sqrt{\frac{ F_{z} }{m\cdot r}} $ - $ r = \frac{ F_{z} }{m\cdot \omega ^{2} } $ - \\ 
 
\subsubsection{Schiefe Ebene} 
\begin{minipage}{0.45\textwidth} 
\mainformular{$ F_{H}  = F_{G} \cdot sin \alpha $} 
\end{minipage} 
\begin{minipage}{0.45\textwidth} 
 
\legende{}\end{minipage} 
 
$ F_{H}  = F_{G} \cdot sin \alpha $ - $ F_{G}  = \frac{ F_{H} }{sin \alpha } $ - $ sin \alpha  = \frac{F_{H} }{F_{G} } $ - $ F_{N}  = F_{G} \cdot cos \alpha $ - $ F_{G}  = \frac{ F_{N} }{cos \alpha } $ - $ cos \alpha  = \frac{F_{N} }{F_{G} } $ - \\ 
 
\subsubsection{Kraft} 
\begin{minipage}{0.45\textwidth} 
\mainformular{$ F = m\cdot a $} 
\end{minipage} 
\begin{minipage}{0.45\textwidth} 
 
\legende{}\end{minipage} 
 
$ F = m\cdot a $ - $ m = \frac{F}{a} $ - $ a = \frac{F}{m} $ - \\ 
 


\subsection{Kinematik}
\subsubsection{Zentralbeschleunigung} 
\begin{minipage}{0.45\textwidth} 
\mainformular{ $ a_{z}  = \omega ^{2} \cdot r $} 
\end{minipage} 
\begin{minipage}{0.45\textwidth} 
 
\legende{ 
$a_{z} $ &  & $ $ & $ $ \\ 
$\omega $ &  & $ $ & $ $ \\ 
$r $ &  & $ $ & $ $ \\ 
} 
\end{minipage} 
$ a_{z}  = \omega ^{2} \cdot r $ \textcolor{lightgray}{\textbf{---}} 
$ \omega  = \sqrt{\frac{a_{z} }{r}} $ \textcolor{lightgray}{\textbf{---}} 
$ r = \frac{a_{z} }{\omega } $ \textcolor{lightgray}{\textbf{---}} 

\subsubsection{Bahngeschwindigkeit} 
\begin{minipage}{0.45\textwidth} 
\mainformular{ $ v = \omega \cdot r $} 
\end{minipage} 
\begin{minipage}{0.45\textwidth} 
 
\legende{ 
$v $ & Geschwindigkeit & $\si{\metre\per\second} $ & $ $ \\ 
$\omega $ &  & $ $ & $ $ \\ 
$r $ &  & $ $ & $ $ \\ 
} 
\end{minipage} 
$ v = \omega \cdot r $ \textcolor{lightgray}{\textbf{---}} 
$ \omega  = \frac{v}{r} $ \textcolor{lightgray}{\textbf{---}} 
$ r = \frac{v}{\omega } $ \textcolor{lightgray}{\textbf{---}} 

\subsubsection{Winkelgeschwindigkeit} 
\begin{minipage}{0.45\textwidth} 
\mainformular{ $ \omega  = 2\cdot \pi \cdot f $} 
\end{minipage} 
\begin{minipage}{0.45\textwidth} 
 
\legende{ 
$\omega $ &  & $ $ & $ $ \\ 
$\pi $ &  & $ $ & $ $ \\ 
$f $ & Frequenz & $\si{\hertz} $ & $ \frac{1}{\si{\second}}$ \\ 
} 
\end{minipage} 
$ \omega  = 2\cdot \pi \cdot f $ \textcolor{lightgray}{\textbf{---}} 
$ f = \frac{\omega }{2\cdot \pi } $ \textcolor{lightgray}{\textbf{---}} 
$ \omega  = \frac{2\cdot \pi }{ T} $ \textcolor{lightgray}{\textbf{---}} 
$ T = \frac{2\cdot \pi }{ \omega } $ \textcolor{lightgray}{\textbf{---}} 

\subsubsection{Frequenz-Periodendauer} 
\begin{minipage}{0.45\textwidth} 
\mainformular{ $ f = \frac{1}{T} $} 
\end{minipage} 
\begin{minipage}{0.45\textwidth} 
 
\legende{ 
$f $ & Frequenz & $\si{\hertz} $ & $ \frac{1}{\si{\second}}$ \\ 
$T $ & absolute Temperatur & $\si{\kelvin} $ & $ 273,15 \si{\kelvin} = 0 \si{\degree}$ \\ 
} 
\end{minipage} 
$ f = \frac{1}{T} $ \textcolor{lightgray}{\textbf{---}} 
$ T = \frac{1}{f} $ \textcolor{lightgray}{\textbf{---}} 
$ f = \frac{n}{t} $ \textcolor{lightgray}{\textbf{---}} 
$ t = \frac{n}{f} $ \textcolor{lightgray}{\textbf{---}} 
$ n = f\cdot t $ \textcolor{lightgray}{\textbf{---}} 

\subsubsection{Schiefer Wurf} 
\begin{minipage}{0.45\textwidth} 
\mainformular{ $ x_{w}  = \frac{v_{0} ^{2} \cdot sin(2\cdot \alpha )}{       g} $} 
\end{minipage} 
\begin{minipage}{0.45\textwidth} 
 
\legende{ 
$x_{w} $ &  & $ $ & $ $ \\ 
$v_{0} $ & Anfangsgeschwindigkeit & $\si{\metre\per\second} $ & $ $ \\ 
$sin(2 $ &  & $ $ & $ $ \\ 
$\alpha $ & Winkel & $\si{\deg} $ & $ $ \\ 
$g $ & Fallbeschleunigung & $\si{\metre\per\square\second} $ & $ 9,8126 \si{\metre\per\square\second}$ \\ 
} 
\end{minipage} 
$ x_{w}  = \frac{v_{0} ^{2} \cdot sin(2\cdot \alpha )}{       g} $ \textcolor{lightgray}{\textbf{---}} 
$ t =\frac{v_{0} \cdot sin \alpha }{  g} $ \textcolor{lightgray}{\textbf{---}} 
$ v_{y}  =  v\cdot sin\alpha - g\cdot t $ \textcolor{lightgray}{\textbf{---}} 
$ v= \frac{ v_{y} +g\cdot t}{ sin\alpha } $ \textcolor{lightgray}{\textbf{---}} 
$ v_{x}  = v\cdot  cos\alpha $ \textcolor{lightgray}{\textbf{---}} 
$ v= \frac{ v_{x} }{ cos\alpha } $ \textcolor{lightgray}{\textbf{---}} 
$ v= \sqrt{ v_{x} ^{2} + v_{y} ^{2} } $ \textcolor{lightgray}{\textbf{---}} 
$ v_{x} = \sqrt{ v^{2}  - v_{y} ^{2} } $ \textcolor{lightgray}{\textbf{---}} 
$ v_{y} = \sqrt{ v^{2}  - v_{x} ^{2} } $ \textcolor{lightgray}{\textbf{---}} 
$ v_{y} = tan \alpha \cdot  v_{x} $ \textcolor{lightgray}{\textbf{---}} 
$ tan \alpha = \frac{v_{y} }{v_{x} } $ \textcolor{lightgray}{\textbf{---}} 
$ v_{x} = \frac{v_{y} }{tan \alpha } $ \textcolor{lightgray}{\textbf{---}} 
$ y = x\cdot tan \alpha  - \frac{   g\cdot x^{2} }{2\cdot v^{2} _{0} \cdot cos ^{2}\alpha } $ \textcolor{lightgray}{\textbf{---}} 
$ t =\frac{2\cdot v_{0} \cdot sin \alpha }{ g} $ \textcolor{lightgray}{\textbf{---}} 

\subsubsection{Waagrechter Wurf} 
\begin{minipage}{0.45\textwidth} 
\mainformular{ $ h = \frac{1}{2}\cdot g\cdot t^{2} $} 
\end{minipage} 
\begin{minipage}{0.45\textwidth} 
 
\legende{ 
$h $ & Höhe & $\si{\metre} $ & $ $ \\ 
$g $ & Fallbeschleunigung & $\si{\metre\per\square\second} $ & $ 9,8126 \si{\metre\per\square\second}$ \\ 
$t $ & Zeit & $\si{\second} $ & $ $ \\ 
} 
\end{minipage} 
$ h = \frac{1}{2}\cdot g\cdot t^{2} $ \textcolor{lightgray}{\textbf{---}} 
$ g = \frac{2\cdot h}{t^{2} } $ \textcolor{lightgray}{\textbf{---}} 
$ t = \sqrt{\frac{2\cdot h}{g}} $ \textcolor{lightgray}{\textbf{---}} 
$ s = v\cdot t $ \textcolor{lightgray}{\textbf{---}} 
$ v = \frac{s}{t} $ \textcolor{lightgray}{\textbf{---}} 

\subsubsection{Senkrechter Wurf nach oben} 
\begin{minipage}{0.45\textwidth} 
\mainformular{ $ h = h_{0}  + v_{0} \cdot t - \frac{1}{2}\cdot g\cdot t^{2} $} 
\end{minipage} 
\begin{minipage}{0.45\textwidth} 
 
\legende{ 
$h $ & Höhe & $\si{\metre} $ & $ $ \\ 
$h_{0} $ & Abwurfhöhe & $\si{\metre} $ & $ $ \\ 
$v_{0} $ & Anfangsgeschwindigkeit & $\si{\metre\per\second} $ & $ $ \\ 
$t $ & Zeit & $\si{\second} $ & $ $ \\ 
$g $ & Fallbeschleunigung & $\si{\metre\per\square\second} $ & $ 9,8126 \si{\metre\per\square\second}$ \\ 
$t $ & Zeit & $\si{\second} $ & $ $ \\ 
} 
\end{minipage} 
$ h = h_{0}  + v_{0} \cdot t - \frac{1}{2}\cdot g\cdot t^{2} $ \textcolor{lightgray}{\textbf{---}} 
$ g = - \frac{2\cdot (h - h_{0}  - v_{0} \cdot t)}{     t^{2} } $ \textcolor{lightgray}{\textbf{---}} 
$ t = \frac{-v_{0}  \pm \sqrt{v_{0} ^{2} +4\cdot 0,5\cdot g\cdot (h_{0}  -h)}}{      -g} $ \textcolor{lightgray}{\textbf{---}} 
$ h_{0}  = h - v_{0} \cdot t + \frac{1}{2}\cdot g\cdot t^{2} $ \textcolor{lightgray}{\textbf{---}} 
$ v = v_{0}  - g\cdot t $ \textcolor{lightgray}{\textbf{---}} 
$ v_{0}  = v + g\cdot t $ \textcolor{lightgray}{\textbf{---}} 
$ t = \frac{v_{0} -v}{  g} $ \textcolor{lightgray}{\textbf{---}} 
$ g = \frac{v_{0}  - v}{  t} $ \textcolor{lightgray}{\textbf{---}} 

\subsubsection{Freier Fall} 
\begin{minipage}{0.45\textwidth} 
\mainformular{ $ h = \frac{1}{2}\cdot g\cdot t^{2} $} 
\end{minipage} 
\begin{minipage}{0.45\textwidth} 
 
\legende{ 
$h $ & Höhe & $\si{\metre} $ & $ $ \\ 
$g $ & Fallbeschleunigung & $\si{\metre\per\square\second} $ & $ 9,8126 \si{\metre\per\square\second}$ \\ 
$t $ & Zeit & $\si{\second} $ & $ $ \\ 
} 
\end{minipage} 
$ h = \frac{1}{2}\cdot g\cdot t^{2} $ \textcolor{lightgray}{\textbf{---}} 
$ g = \frac{2\cdot h}{ t^{2} } $ \textcolor{lightgray}{\textbf{---}} 
$ t = \sqrt{\frac{2\cdot h}{g}} $ \textcolor{lightgray}{\textbf{---}} 
$ v = \sqrt{2\cdot h\cdot g} $ \textcolor{lightgray}{\textbf{---}} 
$ h = \frac{ v^{2} }{2\cdot g} $ \textcolor{lightgray}{\textbf{---}} 

\subsubsection{Durchschnittsbeschleunigung} 
\begin{minipage}{0.45\textwidth} 
\mainformular{ $ a = \frac{v_{1}  - v_{2} }{t_{1}  - t_{2} } $} 
\end{minipage} 
\begin{minipage}{0.45\textwidth} 
 
\legende{ 
$a $ & Beschleunigung & $\si{\metre\per\square\second} $ & $ $ \\ 
$v_{1} $ & Geschwindigkeit & $\si{\metre\per\second} $ & $ $ \\ 
$v_{2} $ & Geschwindigkeit & $\si{\metre\per\second} $ & $ $ \\ 
$t_{1} $ & aufeinanderfolgende Zeitpunkte & $\si{\second} $ & $ $ \\ 
$t_{2} $ & aufeinanderfolgende Zeitpunkte & $\si{\second} $ & $ $ \\ 
} 
\end{minipage} 
$ a = \frac{v_{1}  - v_{2} }{t_{1}  - t_{2} } $ \textcolor{lightgray}{\textbf{---}} 

\subsubsection{Durchschnittsgeschwindigkeit} 
\begin{minipage}{0.45\textwidth} 
\mainformular{ $ v = \frac{x_{1}  - x_{2} }{t_{1}  - t_{2} } $} 
\end{minipage} 
\begin{minipage}{0.45\textwidth} 
 
\legende{ 
$v $ & Geschwindigkeit & $\si{\metre\per\second} $ & $ $ \\ 
$x_{1} $ & zurückgelegter Weg & $\si{\metre} $ & $ $ \\ 
$x_{2} $ & zurückgelegter Weg & $\si{\metre} $ & $ $ \\ 
$t_{1} $ & aufeinanderfolgende Zeitpunkte & $\si{\second} $ & $ $ \\ 
$t_{2} $ & aufeinanderfolgende Zeitpunkte & $\si{\second} $ & $ $ \\ 
} 
\end{minipage} 
$ v = \frac{x_{1}  - x_{2} }{t_{1}  - t_{2} } $ \textcolor{lightgray}{\textbf{---}} 

\subsubsection{Beschleunigte Bewegung mit Anfangsgeschwindigkeit} 
\begin{minipage}{0.45\textwidth} 
\mainformular{ $ v = v_{0}  + a\cdot t $} 
\end{minipage} 
\begin{minipage}{0.45\textwidth} 
 
\legende{ 
$v $ & Geschwindigkeit & $\si{\metre\per\second} $ & $ $ \\ 
$v_{0} $ & Anfangsgeschwindigkeit & $\si{\metre\per\second} $ & $ $ \\ 
$a $ & Beschleunigung & $\si{\metre\per\square\second} $ & $ $ \\ 
$t $ & Zeit & $\si{\second} $ & $ $ \\ 
} 
\end{minipage} 
$ v = v_{0}  + a\cdot t $ \textcolor{lightgray}{\textbf{---}} 
$ v_{0}  = v - a\cdot t $ \textcolor{lightgray}{\textbf{---}} 
$ t = \frac{v - v_{0} }{a} $ \textcolor{lightgray}{\textbf{---}} 
$ a = \frac{v - v_{0} }{  t} $ \textcolor{lightgray}{\textbf{---}} 
$ s = s_{0}  + v_{0} \cdot t + \frac{1}{2}\cdot a\cdot t^{2} $ \textcolor{lightgray}{\textbf{---}} 
$ a = \frac{2\cdot (s - s_{0}  - v_{0} \cdot t)}{     t^{2} } $ \textcolor{lightgray}{\textbf{---}} 
$ t = \frac{-v_{0}  \pm \sqrt{v_{0} ^{2} -4\cdot 0,5\cdot a\cdot (s_{0}  -s)}}{       a} $ \textcolor{lightgray}{\textbf{---}} 
$ s_{0}  = s - v_{0} \cdot t - \frac{1}{2}\cdot a\cdot t^{2} $ \textcolor{lightgray}{\textbf{---}} 
$ v_{0}  =\frac{s-s_{0} -0,5\cdot a\cdot t^{2} }{    t} $ \textcolor{lightgray}{\textbf{---}} 
$ v  =\sqrt{2\cdot a \cdot s+ v_{0}^2} $ \textcolor{lightgray}{\textbf{---}} 
$ v_{0}  =\sqrt{v^2-2\cdot a \cdot s} $ \textcolor{lightgray}{\textbf{---}} 

\subsubsection{Beschleunigte Bewegung} 
\begin{minipage}{0.45\textwidth} 
\mainformular{ $ v = a\cdot t $} 
\end{minipage} 
\begin{minipage}{0.45\textwidth} 
 
\legende{ 
$v $ & Geschwindigkeit & $\si{\metre\per\second} $ & $ $ \\ 
$a $ & Beschleunigung & $\si{\metre\per\square\second} $ & $ $ \\ 
$t $ & Zeit & $\si{\second} $ & $ $ \\ 
} 
\end{minipage} 
$ v = a\cdot t $ \textcolor{lightgray}{\textbf{---}} 
$ a = \frac{v}{t} $ \textcolor{lightgray}{\textbf{---}} 
$ t = \frac{v}{a} $ \textcolor{lightgray}{\textbf{---}} 
$ s = \frac{1}{2}\cdot a\cdot t^{2} $ \textcolor{lightgray}{\textbf{---}} 
$ a = \frac{2\cdot s}{t^{2} } $ \textcolor{lightgray}{\textbf{---}} 
$ t = \sqrt{\frac{2\cdot s}{a}} $ \textcolor{lightgray}{\textbf{---}} 

\subsubsection{Geradlinige Bewegung v=konst.} 
\begin{minipage}{0.45\textwidth} 
\mainformular{ $ s = v\cdot t $} 
\end{minipage} 
\begin{minipage}{0.45\textwidth} 
 
\legende{ 
$s $ & Weg,Auslenkung & $\si{\metre} $ & $ $ \\ 
$v $ & Geschwindigkeit & $\si{\metre\per\second} $ & $ $ \\ 
$t $ & Zeit & $\si{\second} $ & $ $ \\ 
} 
\end{minipage} 
$ s = v\cdot t $ \textcolor{lightgray}{\textbf{---}} 
$ v = \frac{s}{t} $ \textcolor{lightgray}{\textbf{---}} 
$ t = \frac{s}{v} $ \textcolor{lightgray}{\textbf{---}} 



\section{Elektrotechnik}

\subsection{Allgemeine Elektrotechnik}
\subsubsection{Spannungsteiler} 
\begin{minipage}{0.45\textwidth} 
\mainformular{ $ U_{1}  = U_{g} \cdot \frac{ R_{1} }{R_{1} +R_{2} } $} 
\end{minipage} 
\begin{minipage}{0.45\textwidth} 
 
\legende{ 
$U_{1} $ &  & $ $ & $ $ \\ 
$U_{g} $ &  & $ $ & $ $ \\ 
$R_{1} $ &  & $ $ & $ $ \\ 
$R_{1} $ &  & $ $ & $ $ \\ 
$R_{2} $ &  & $ $ & $ $ \\ 
} 
\end{minipage} 
$ U_{1}  = U_{g} \cdot \frac{ R_{1} }{R_{1} +R_{2} } $ \textcolor{lightgray}{\textbf{---}} 



\subsection{Elektrischer Schwingkreis}
\subsubsection{Eigenkreisfrequenz} 
\begin{minipage}{0.45\textwidth} 
$ \omega  = \frac{ 1}{\sqrt{L\cdot C}} $\\ 
$ L= \frac{ 1}{\omega ^{2} \cdot C} $\\ 
$ C = \frac{ 1}{\omega ^{2} \cdot L} $\\ 
\end{minipage} 
\begin{minipage}{0.45\textwidth} 
 
\end{minipage} 
\subsubsection{Eigenfrequenz (Ungedämpfte elektrische Schwingung)} 
\begin{minipage}{0.45\textwidth} 
$ f = \frac{ 1}{2\cdot \pi \cdot \sqrt{L\cdot C}} $\\ 
$ L = \frac{ 1}{(2\cdot \pi \cdot f)^{2} \cdot C} $\\ 
$ C = \frac{ 1}{(2\cdot \pi \cdot f)^{2} \cdot L} $\\ 
\end{minipage} 
\begin{minipage}{0.45\textwidth} 
 
\end{minipage} 


\subsection{Elektrisches Feld}
\subsubsection{Elektrische Energie des Kondensators} 
\begin{minipage}{0.45\textwidth} 
$ W =\frac{1}{2}\cdot C\cdot U^{2} $\\ 
$ U = \sqrt{\frac{2\cdot W}{ C}} $\\ 
$ C = \frac{2\cdot W}{ U^{2} } $\\ 
\end{minipage} 
\begin{minipage}{0.45\textwidth} 
 
\end{minipage} 
\subsubsection{Parallelschaltung von Kondensatoren} 
\begin{minipage}{0.45\textwidth} 
$ C_{g}  = C_{1}  + C_{2} $\\ 
$ C_{1}  = C_{g}  - C_{2} $\\ 
$ C_{2}  = C_{g}  - C_{1} $\\ 
$ Q_{g}  = Q_{1}  + Q_{2} $\\ 
$ Q_{1}  = Q_{g}  - Q_{2} $\\ 
$ Q_{2}  = Q_{g}  - Q_{1} $\\ 
\end{minipage} 
\begin{minipage}{0.45\textwidth} 
 
\end{minipage} 
\subsubsection{Reihenschaltung von Kondensatoren} 
\begin{minipage}{0.45\textwidth} 
$ C_{g}  = \frac{C_{1} \cdot C_{2} }{C_{1} +C_{2} } $\\ 
$ C_{1}  = \frac{C_{2} \cdot C_{g} }{C_{2} -C_{g} } $\\ 
$ C_{2}  = \frac{C_{1} \cdot C_{g} }{C_{1} -C_{g} } $\\ 
$ U_{g}  = U_{1}  + U_{2} $\\ 
$ U_{1}  = U_{g}  - U_{2} $\\ 
$ U_{2}  = U_{g}  - U_{1} $\\ 
\end{minipage} 
\begin{minipage}{0.45\textwidth} 
 
\end{minipage} 
\subsubsection{Kapazität eines Kondensators} 
\begin{minipage}{0.45\textwidth} 
$ C = \frac{Q}{U} $\\ 
$ Q = C\cdot U $\\ 
$ U = \frac{Q}{C} $\\ 
$ C = \epsilon _{0} \cdot \epsilon _{r} \cdot \frac{A}{d} $\\ 
$ A = \frac{C\cdot d}{\epsilon _{0} \epsilon _{r} } $\\ 
$ d = \epsilon _{0} \cdot \epsilon _{r} \cdot \frac{A}{C} $\\ 
\end{minipage} 
\begin{minipage}{0.45\textwidth} 
 
\end{minipage} 
\subsubsection{Gesetz von Coulomb} 
\begin{minipage}{0.45\textwidth} 
$ F = \frac{ 1}{4\pi \epsilon _{0} } \cdot  \frac{Q_{1} \cdot Q_{2} }{  r^{2} } $\\ 
$ r = \sqrt{\frac{  1}{4\pi \epsilon _{0} } \cdot  \frac{Q_{1} \cdot Q_{2} }{  F}} $\\ 
$ Q_{1}  = 4\pi \epsilon _{0}  \cdot  \frac{F\cdot r^{2} }{ Q_{2} } $\\ 
\end{minipage} 
\begin{minipage}{0.45\textwidth} 
 
\end{minipage} 
\subsubsection{Elektrische Feldstärke} 
\begin{minipage}{0.45\textwidth} 
$ E = \frac{F}{Q} $\\ 
$ F = E\cdot Q $\\ 
$ Q = \frac{F}{E} $\\ 
$ E = \frac{U}{d} $\\ 
$ U = E\cdot d $\\ 
$ d = \frac{U}{E} $\\ 
\end{minipage} 
\begin{minipage}{0.45\textwidth} 
 
\end{minipage} 


\subsection{Elektrizitätslehre}
\subsubsection{Elektrische Arbeit} 
\begin{minipage}{0.45\textwidth} 
\mainformular{$ W = U\cdot I\cdot t $} 
\end{minipage} 
\begin{minipage}{0.45\textwidth} 
 
\legende{}\end{minipage} 
 
$ W = U\cdot I\cdot t $ - $ U = \frac{W}{I\cdot t} $ - $ I = \frac{W}{U\cdot t} $ - $ t = \frac{ P}{U\cdot I} $ - \\ 
 
\subsubsection{Elektrische Leistung} 
\begin{minipage}{0.45\textwidth} 
\mainformular{$ P = U\cdot I $} 
\end{minipage} 
\begin{minipage}{0.45\textwidth} 
 
\legende{}\end{minipage} 
 
$ P = U\cdot I $ - $ U = \frac{P}{I} $ - $ I = \frac{P}{U} $ - \\ 
 
\subsubsection{Spezifischer Leitwert} 
\begin{minipage}{0.45\textwidth} 
\mainformular{$ R = \frac{ l}{\kappa \cdot A} $} 
\end{minipage} 
\begin{minipage}{0.45\textwidth} 
 
\legende{}\end{minipage} 
 
$ R = \frac{ l}{\kappa \cdot A} $ - $ l = R\cdot \kappa \cdot A $ - $ A = \frac{l}{\kappa \cdot R} $ - $ \kappa  = \frac{ l}{R\cdot A} $ - \\ 
 
\subsubsection{Spezifischer Widerstand} 
\begin{minipage}{0.45\textwidth} 
\mainformular{$ R = \frac{\rho \cdot l}{ A} $} 
\end{minipage} 
\begin{minipage}{0.45\textwidth} 
 
\legende{}\end{minipage} 
 
$ R = \frac{\rho \cdot l}{ A} $ - $ l = \frac{R\cdot A}{ \rho } $ - $ \rho  = \frac{R\cdot A}{ l} $ - $ A = \frac{R\cdot \rho }{ A} $ - \\ 
 
\subsubsection{Widerstandsänderung - Temperatur} 
\begin{minipage}{0.45\textwidth} 
\mainformular{$ \Delta R = R\cdot \alpha \cdot \Delta T $} 
\end{minipage} 
\begin{minipage}{0.45\textwidth} 
 
\legende{}\end{minipage} 
 
$ \Delta R = R\cdot \alpha \cdot \Delta T $ - $ \Delta R = R\cdot \alpha \cdot \Delta T $ - $ \alpha  = \frac{R}{\Delta R\cdot \Delta T} $ - $ \Delta T = \frac{   R}{\Delta R\cdot \alpha \cdot \Delta T} $ - \\ 
 
\subsubsection{Parallelschaltung von Widerständen} 
\begin{minipage}{0.45\textwidth} 
\mainformular{$ R_{g}  = \frac{R_{1} \cdot R_{2} }{R_{1} +R_{2} } $} 
\end{minipage} 
\begin{minipage}{0.45\textwidth} 
 
\legende{}\end{minipage} 
 
$ R_{g}  = \frac{R_{1} \cdot R_{2} }{R_{1} +R_{2} } $ - $ R_{1}  = \frac{R_{2} \cdot R_{g} }{R_{2} -R_{g} } $ - $ R_{2}  = \frac{R_{1} \cdot R_{g} }{R_{1} -R_{g} } $ - $ I_{g}  = I_{1}  + I_{2} $ - $ I_{1}  = I_{g}  - I_{2} $ - $ I_{2}  = I_{g}  - I_{1} $ - \\ 
 
\subsubsection{Reihenschaltung von Widerständen} 
\begin{minipage}{0.45\textwidth} 
\mainformular{$ R_{g}  = R_{1}  + R_{2} $} 
\end{minipage} 
\begin{minipage}{0.45\textwidth} 
 
\legende{}\end{minipage} 
 
$ R_{g}  = R_{1}  + R_{2} $ - $ R_{1}  = R_{g}  - R_{2} $ - $ R_{2}  = R_{g}  - R_{1} $ - $ U_{g}  = U_{1}  + U_{2} $ - $ U_{1}  = U_{g}  - U_{2} $ - $ U_{2}  = U_{g}  - U_{1} $ - \\ 
 
\subsubsection{Ohmsches Gesetz} 
\begin{minipage}{0.45\textwidth} 
\mainformular{$ R = \frac{U}{I} $} 
\end{minipage} 
\begin{minipage}{0.45\textwidth} 
 
\legende{}\end{minipage} 
 
$ R = \frac{U}{I} $ - $ U = R\cdot I $ - $ I = \frac{U}{R} $ - \\ 
 
\subsubsection{Stromstärke} 
\begin{minipage}{0.45\textwidth} 
\mainformular{$ I = \frac{\Delta Q}{\Delta t} $} 
\end{minipage} 
\begin{minipage}{0.45\textwidth} 
 
\legende{}\end{minipage} 
 
$ I = \frac{\Delta Q}{\Delta t} $ - $ \Delta Q =I\cdot \Delta t $ - $ \Delta t = \frac{\Delta Q}{I} $ - \\ 
 


\subsection{Magnetisches Feld}
\subsubsection{Parallelschaltung (Induktivität)} 
\begin{minipage}{0.45\textwidth} 
\mainformular{Interaktiv} 
\end{minipage} 
\begin{minipage}{0.45\textwidth} 
 
\legende{}\end{minipage} 
 
Interaktiv - $ L_{1}  = \frac{L_{2} \cdot L_{g} }{L_{2} -L_{g} } $ - $ L_{2}  = \frac{L_{1} \cdot L_{g} }{L_{1} -L_{g} } $ - $ I_{g}  = I_{1}  + I_{2} $ - $ I_{1}  = I_{g}  - I_{2} $ - $ I_{2}  = I_{g}  - I_{1} $ - \\ 
 
\subsubsection{Reihenschaltung (Induktivität)} 
\begin{minipage}{0.45\textwidth} 
\mainformular{$ L_{g}  = L_{1}  + L_{2} $} 
\end{minipage} 
\begin{minipage}{0.45\textwidth} 
 
\legende{}\end{minipage} 
 
$ L_{g}  = L_{1}  + L_{2} $ - $ L_{1}  = L_{g}  - L_{2} $ - $ L_{2}  = L_{g}  - L_{1} $ - $ U_{g}  = U_{1}  + U_{2} $ - $ U_{1}  = U_{g}  - U_{2} $ - $ U_{2}  = U_{g}  - U_{1} $ - \\ 
 
\subsubsection{Magnetischer Fluß} 
\begin{minipage}{0.45\textwidth} 
\mainformular{$ \Phi  = B\cdot A\cdot cos(\delta ) $} 
\end{minipage} 
\begin{minipage}{0.45\textwidth} 
 
\legende{}\end{minipage} 
 
$ \Phi  = B\cdot A\cdot cos(\delta ) $ - $ A = \frac{ \Phi }{B\cdot cos(\delta )} $ - $ B = \frac{ \Phi }{A\cdot cos(\delta )} $ - $ \delta =arccos(\frac{ \Phi }{B\cdot A}) $ - \\ 
 
\subsubsection{Flußdichte - Feldstärke} 
\begin{minipage}{0.45\textwidth} 
\mainformular{$ B = \mu _{r} \cdot \mu _{0} \cdot H $} 
\end{minipage} 
\begin{minipage}{0.45\textwidth} 
 
\legende{}\end{minipage} 
 
$ B = \mu _{r} \cdot \mu _{0} \cdot H $ - $ H =\frac{ B}{\mu _{r} \cdot \mu _{0} } $ - $ \mu _{r} =\frac{ B}{\mu _{0} \cdot H} $ - $ \mu _{0} =\frac{ B}{\mu _{r} \cdot H} $ - \\ 
 
\subsubsection{Induktivität einer langgestreckten Spule} 
\begin{minipage}{0.45\textwidth} 
\mainformular{$ L = \mu _{0} \cdot \mu _{r} \cdot \frac{A\cdot N^{2} }{lSP} $} 
\end{minipage} 
\begin{minipage}{0.45\textwidth} 
 
\legende{}\end{minipage} 
 
$ L = \mu _{0} \cdot \mu _{r} \cdot \frac{A\cdot N^{2} }{lSP} $ - $ l_{SP} = \mu _{0} \cdot \mu _{r} \cdot \frac{A\cdot N^{2} }{L} $ - $ A = \frac{ L\cdot l}{\mu _{0} \cdot \mu _{r} \cdot N^{2} } $ - $ N = \sqrt{\frac{ L\cdot l}{\mu _{0} \cdot \mu _{r} \cdot A}} $ - \\ 
 
\subsubsection{Feldstärke einer langgestreckten Spule} 
\begin{minipage}{0.45\textwidth} 
\mainformular{$ H = \frac{I\cdot N}{ l} $} 
\end{minipage} 
\begin{minipage}{0.45\textwidth} 
 
\legende{}\end{minipage} 
 
$ H = \frac{I\cdot N}{ l} $ - $ I = \frac{H\cdot l}{ N} $ - $ N = \frac{H\cdot l}{ I} $ - $ l = \frac{I\cdot N}{ H} $ - \\ 
 
\subsubsection{Flußdichte} 
\begin{minipage}{0.45\textwidth} 
\mainformular{$ B = \frac{ F}{I\cdot l} $} 
\end{minipage} 
\begin{minipage}{0.45\textwidth} 
 
\legende{}\end{minipage} 
 
$ B = \frac{ F}{I\cdot l} $ - $ F = B\cdot I\cdot l $ - $ I = \frac{ F}{B\cdot l} $ - $ l = \frac{ F}{I\cdot B} $ - \\ 
 


\section{Optik}

\subsection{Linsen}
\subsubsection{Bildgröße - Gegenstandsgröße - Aufgabe:BildGegenUmbbb} 
\begin{minipage}{0.45\textwidth} 
$ \frac{G}{B} = \frac{g}{b} $\\ 
$ G = \frac{g\cdot B}{ b} $\\ 
$ B = \frac{G\cdot b}{ g} $\\ 
$ g = \frac{G\cdot b}{ B} $\\ 
$ b = \frac{B\cdot g}{ G} $\\ 
\end{minipage} 
\begin{minipage}{0.45\textwidth} 
 
\end{minipage} 
\subsubsection{Bildgröße - Gegenstandsgröße - Aufgabe:BildGegenUmgg} 
\begin{minipage}{0.45\textwidth} 
$ \frac{G}{B} = \frac{g}{b} $\\ 
$ G = \frac{g\cdot B}{ b} $\\ 
$ B = \frac{G\cdot b}{ g} $\\ 
$ g = \frac{G\cdot b}{ B} $\\ 
$ b = \frac{B\cdot g}{ G} $\\ 
\end{minipage} 
\begin{minipage}{0.45\textwidth} 
 
\end{minipage} 
\subsubsection{Bildgröße - Gegenstandsgröße - Aufgabe:BildGegenUmg} 
\begin{minipage}{0.45\textwidth} 
$ \frac{G}{B} = \frac{g}{b} $\\ 
$ G = \frac{g\cdot B}{ b} $\\ 
$ B = \frac{G\cdot b}{ g} $\\ 
$ g = \frac{G\cdot b}{ B} $\\ 
$ b = \frac{B\cdot g}{ G} $\\ 
\end{minipage} 
\begin{minipage}{0.45\textwidth} 
 
\end{minipage} 
\subsubsection{Bildgröße - Gegenstandsgröße} 
\begin{minipage}{0.45\textwidth} 
$ \frac{G}{B} = \frac{g}{b} $\\ 
$ G = \frac{g\cdot B}{ b} $\\ 
$ B = \frac{G\cdot b}{ g} $\\ 
$ g = \frac{G\cdot b}{ B} $\\ 
$ b = \frac{B\cdot g}{ G} $\\ 
\end{minipage} 
\begin{minipage}{0.45\textwidth} 
 
\end{minipage} 
\subsubsection{Bildgröße - Gegenstandsgröße - Aufgabe:BildGegenUmBG} 
\begin{minipage}{0.45\textwidth} 
$ \frac{G}{B} = \frac{g}{b} $\\ 
$ G = \frac{g\cdot B}{ b} $\\ 
$ B = \frac{G\cdot b}{ g} $\\ 
$ g = \frac{G\cdot b}{ B} $\\ 
$ b = \frac{B\cdot g}{ G} $\\ 
\end{minipage} 
\begin{minipage}{0.45\textwidth} 
 
\end{minipage} 
\subsubsection{Bildgröße - Gegenstandsgröße - Aufgabe:BildGegenUmggg} 
\begin{minipage}{0.45\textwidth} 
$ \frac{G}{B} = \frac{g}{b} $\\ 
$ G = \frac{g\cdot B}{ b} $\\ 
$ B = \frac{G\cdot b}{ g} $\\ 
$ g = \frac{G\cdot b}{ B} $\\ 
$ b = \frac{B\cdot g}{ G} $\\ 
\end{minipage} 
\begin{minipage}{0.45\textwidth} 
 
\end{minipage} 
\subsubsection{Brennweite} 
\begin{minipage}{0.45\textwidth} 
$ f  = \frac{g\cdot b}{g+b} $\\ 
$ b  = \frac{f\cdot g}{g-f} $\\ 
$ g  = \frac{f\cdot b}{b-f} $\\ 
\end{minipage} 
\begin{minipage}{0.45\textwidth} 
 
\end{minipage} 
\subsubsection{Brennweite - Aufgabe:BrennUmg} 
\begin{minipage}{0.45\textwidth} 
$ f  = \frac{g\cdot b}{g+b} $\\ 
$ b  = \frac{f\cdot g}{g-f} $\\ 
$ g  = \frac{f\cdot b}{b-f} $\\ 
\end{minipage} 
\begin{minipage}{0.45\textwidth} 
 
\end{minipage} 
\subsubsection{Brennweite - Aufgabe:BrennUmf} 
\begin{minipage}{0.45\textwidth} 
$ f  = \frac{g\cdot b}{g+b} $\\ 
$ b  = \frac{f\cdot g}{g-f} $\\ 
$ g  = \frac{f\cdot b}{b-f} $\\ 
\end{minipage} 
\begin{minipage}{0.45\textwidth} 
 
\end{minipage} 
\subsubsection{Brennweite - Aufgabe:BrennUmb} 
\begin{minipage}{0.45\textwidth} 
$ f  = \frac{g\cdot b}{g+b} $\\ 
$ b  = \frac{f\cdot g}{g-f} $\\ 
$ g  = \frac{f\cdot b}{b-f} $\\ 
\end{minipage} 
\begin{minipage}{0.45\textwidth} 
 
\end{minipage} 


\subsection{Reflexion und Brechung}
\subsubsection{Brechung} 
\begin{minipage}{0.45\textwidth} 
$ n = \frac{sin\alpha _{1} }{sin\alpha _{2} } $\\ 
$ sin\alpha _{1}  = n\cdot sin\alpha _{2} $\\ 
$ sin\alpha _{2}  = \frac{sin\alpha _{1} }{ n} $\\ 
\end{minipage} 
\begin{minipage}{0.45\textwidth} 
 
\end{minipage} 
\subsubsection{Reflexion} 
\begin{minipage}{0.45\textwidth} 
$ \alpha _{1}  = \alpha _{2} $\\ 
\end{minipage} 
\begin{minipage}{0.45\textwidth} 
 
\end{minipage} 


\section{Wärmelehre}

\subsection{Ausdehnung der Körper}
\subsubsection{Volumenausdehnung} 
\begin{minipage}{0.45\textwidth} 
\mainformular{$ \Delta V = V_{0} \cdot 3\cdot \alpha \cdot \Delta T $} 
\end{minipage} 
\begin{minipage}{0.45\textwidth} 
 
\legende{}\end{minipage} 
 
$ \Delta V = V_{0} \cdot 3\cdot \alpha \cdot \Delta T $ - $ V_{0}  = \frac{  \Delta V}{3\cdot \alpha \cdot \Delta T} $ - $ \alpha  = \frac{  \Delta V}{V_{0} \cdot \Delta T\cdot 3} $ - $ \Delta T = \frac{  \Delta V}{V_{0} \cdot 3\cdot \alpha } $ - \\ 
 
\subsubsection{Längenausdehnung} 
\begin{minipage}{0.45\textwidth} 
\mainformular{$ \Delta l = l_{0} \cdot \alpha \cdot \Delta T $} 
\end{minipage} 
\begin{minipage}{0.45\textwidth} 
 
\legende{}\end{minipage} 
 
$ \Delta l = l_{0} \cdot \alpha \cdot \Delta T $ - $ l_{0}  = \frac{ \Delta l}{\alpha \cdot \Delta T} $ - $ \alpha  = \frac{ \Delta l}{l_{0} \cdot \Delta T} $ - $ \Delta T = \frac{ \Delta l}{l_{0} \cdot \alpha } $ - \\ 
 
\subsubsection{Flächenausdehnung} 
\begin{minipage}{0.45\textwidth} 
\mainformular{$ \Delta A = A_{0} \cdot 2\cdot \alpha \cdot \Delta T $} 
\end{minipage} 
\begin{minipage}{0.45\textwidth} 
 
\legende{}\end{minipage} 
 
$ \Delta A = A_{0} \cdot 2\cdot \alpha \cdot \Delta T $ - $ A_{0}  = \frac{ \Delta A}{2\cdot \alpha \cdot \Delta T} $ - $ \alpha  = \frac{ \Delta A}{A_{0} \cdot \Delta T\cdot 2} $ - $ \Delta T = \frac{ \Delta A}{A_{0} \cdot 2\cdot \alpha } $ - \\ 
 


\subsection{Energie}
\subsubsection{Verdampfen und Kondensieren} 
\begin{minipage}{0.45\textwidth} 
\mainformular{ $ Q =q_{v} \cdot m $} 
\end{minipage} 
\begin{minipage}{0.45\textwidth} 
 
\legende{ 
$Q $ &  & $ $ & $ $ \\ 
$q_{v} $ &  & $ $ & $ $ \\ 
$m $ & Masse & $\si{\kilo\gram} $ & $ $ \\ 
} 
\end{minipage} 
$ Q =q_{v} \cdot m $ \textcolor{lightgray}{\textbf{---}} 
$ m = \frac{Q}{q_{v} } $ \textcolor{lightgray}{\textbf{---}} 
$ q_{v}  = \frac{Q}{m} $ \textcolor{lightgray}{\textbf{---}} 

\subsubsection{Schmelzen und Erstarren} 
\begin{minipage}{0.45\textwidth} 
\mainformular{ $ Q = q_{s} \cdot m $} 
\end{minipage} 
\begin{minipage}{0.45\textwidth} 
 
\legende{ 
$Q $ &  & $ $ & $ $ \\ 
$q_{s} $ &  & $ $ & $ $ \\ 
$m $ & Masse & $\si{\kilo\gram} $ & $ $ \\ 
} 
\end{minipage} 
$ Q = q_{s} \cdot m $ \textcolor{lightgray}{\textbf{---}} 
$ m = \frac{Q}{q_{s} } $ \textcolor{lightgray}{\textbf{---}} 
$ q_{s}  = \frac{Q}{m} $ \textcolor{lightgray}{\textbf{---}} 

\subsubsection{Verbrennungsenergie} 
\begin{minipage}{0.45\textwidth} 
\mainformular{ $ Q = H_{u} \cdot m $} 
\end{minipage} 
\begin{minipage}{0.45\textwidth} 
 
\legende{ 
$Q $ &  & $ $ & $ $ \\ 
$H_{u} $ &  & $ $ & $ $ \\ 
$m $ & Masse & $\si{\kilo\gram} $ & $ $ \\ 
} 
\end{minipage} 
$ Q = H_{u} \cdot m $ \textcolor{lightgray}{\textbf{---}} 
$ H_{u}  = \frac{Q}{m} $ \textcolor{lightgray}{\textbf{---}} 
$ m = \frac{Q}{H_{u} } $ \textcolor{lightgray}{\textbf{---}} 

\subsubsection{Wärmeenergie} 
\begin{minipage}{0.45\textwidth} 
\mainformular{ $ \Delta Q = c\cdot m\cdot \Delta T $} 
\end{minipage} 
\begin{minipage}{0.45\textwidth} 
 
\legende{ 
$\Delta Q $ & Ladungsänderung & $C $ & $ \si{\ampere\second}$ \\ 
$c $ &  & $ $ & $ $ \\ 
$m $ & Masse & $\si{\kilo\gram} $ & $ $ \\ 
$\Delta T $ &  & $ $ & $ $ \\ 
} 
\end{minipage} 
$ \Delta Q = c\cdot m\cdot \Delta T $ \textcolor{lightgray}{\textbf{---}} 
$ m = \frac{ \Delta Q}{c\cdot \Delta T} $ \textcolor{lightgray}{\textbf{---}} 
$ c = \frac{ \Delta Q}{m\cdot \Delta T} $ \textcolor{lightgray}{\textbf{---}} 
$ \Delta T = \frac{\Delta Q}{c\cdot m} $ \textcolor{lightgray}{\textbf{---}} 



\subsection{Temperatur}
\subsubsection{Temperaturdifferenz} 
\begin{minipage}{0.45\textwidth} 
\mainformular{$ \Delta T = T_{2}  - T_{1} $} 
\end{minipage} 
\begin{minipage}{0.45\textwidth} 
 
\legende{}\end{minipage} 
 
$ \Delta T = T_{2}  - T_{1} $ - $ T_{1}  = T_{2}  - \Delta T $ - $ T_{2}  = \Delta T + T_{1} $ - \\ 
 
\subsubsection{Termperatur - Umrechnungen} 
\begin{minipage}{0.45\textwidth} 
\mainformular{$ T = 273,15 + \tau $} 
\end{minipage} 
\begin{minipage}{0.45\textwidth} 
 
\legende{}\end{minipage} 
 
$ T = 273,15 + \tau $ - $ \tau  = T-273,15 $ - $ T_{F}  = \frac{9}{5}\cdot \tau  +32 $ - $ \tau  = \frac{5}{9}\cdot (T_{F}  - 32) $ - $ T_{R}  = \frac{9}{5}\cdot \tau  + 491,67 $ - $ \tau  = \frac{5}{9}\cdot (T_{R}  - 491,67) $ - \\ 
 


\subsection{Zustandsänderung der Gase}
\subsubsection{Thermische Zustandsgleichung} 
\begin{minipage}{0.45\textwidth} 
\mainformular{$ p\cdot V =\nu \cdot R_{m} \cdot T $} 
\end{minipage} 
\begin{minipage}{0.45\textwidth} 
 
\legende{}\end{minipage} 
 
$ p\cdot V =\nu \cdot R_{m} \cdot T $ - $ p =\frac{\nu \cdot R_{m} \cdot T}{  V} $ - $ V =\frac{\nu \cdot R_{m} \cdot T}{  p} $ - $ T =\frac{p\cdot V}{\nu \cdot R_{m} } $ - \\ 
 
\subsubsection{Allgemeine Gasgleichung} 
\begin{minipage}{0.45\textwidth} 
\mainformular{$ V_{1}  = \frac{V_{2} \cdot p_{} \cdot T_{1} }{  T_{2} \cdot p_{1} } $} 
\end{minipage} 
\begin{minipage}{0.45\textwidth} 
 
\legende{}\end{minipage} 
 
$ V_{1}  = \frac{V_{2} \cdot p_{} \cdot T_{1} }{  T_{2} \cdot p_{1} } $ - $ p_{1}  = \frac{V_{2} \cdot p_{2} \cdot T_{1} }{  T_{2} \cdot V_{1} } $ - $ T_{1}  = \frac{V_{1} \cdot p_{1} \cdot T_{2} }{  V_{2} \cdot p_{2} } $ - \\ 
 
\subsubsection{Ideale Gasgleichung}
\begin{minipage}{0.45\textwidth}

\mainformular{$\rho \cdot V = m \cdot R \cdot T$} \\
\end{minipage}
\begin{minipage}{0.45\textwidth}

\legende{
$\rho$ & Druck & $\si{\pascal}$ & \\
$V$ & Volumen & $\si{\cubic\meter}$ & \\
$m$ & Masse & $\si{\kilogram}$ &  \\
$R$ & Spezifische Gaskonstante & $\si{\joule\per\kilogram\per\kelvin} $ & $ $ \\
$T$ & Temperatur & $\si{\kelvin}$ & \\
}

\end{minipage} 

\textbf{Universelle Gaskonstante}\\
\begin{minipage}{0.45\textwidth}
\mainformular{$R_{m} = R \cdot M$} 

\end{minipage}
\begin{minipage}{0.45\textwidth}

\legende{
$R$ & Spez Gaskonstante & $\si{\joule\per\kilogram\per\kelvin} $ & $ $ \\
$R_{m}$ & Univ. Gaskonstante & $\si{\joule\per\mol\per\kelvin} $ & $8,3144598 \si{\joule\per\mol\per\kelvin}$ \\
$M$ & Molare Masse & $\si{\gram\mol}$ & \\
}

\end{minipage}


\subsubsection{Isotherme Zustastandsänderung}
\begin{minipage}{0.45\textwidth}

\mainformular{$W = -m \cdot R \cdot T \cdot \ln{\cfrac{V_2}{V_1}}$}

\end{minipage}
\begin{minipage}{0.45\textwidth}

\legende{
$W$ & Arbeit & $\si{\joule}$ & \\
$m$ & Masse & $\si{\kilogram}$ &  \\
$R$ & Spez. Gaskonstante & $\si{\joule\per\kilogram\per\kelvin} $ & $ $ \\
$T$ & Temperatur & $\si{} $ & \\
}

\end{minipage}
$W = -m \cdot R \cdot T \cdot \ln{\cfrac{V_2}{V_1}}$ - 
$g = \cfrac{F_G}{m}$ \\

\subsection{1. Hauptsatz}
\subsubsection{Wärmelehre}
\begin{minipage}{0.45\textwidth}

\mainformular{$\Delta U + \Delta E_{pot} + \Delta E_{kin} =  \Delta Q + \Delta W$} \\


\end{minipage}
\begin{minipage}{0.45\textwidth}

\legende{
$U$ & Innere Energie & $\si{\kilo\gram}$ & \\
$E_{pot}$ & Potentielle Energie & $\si{\metre\per\square\second}$ & $9,81 \si{\metre\per\square\second}$ \\
$E_{kin}$ & Kinetische Energie & $N$ & $\si{\kilogram\metre\per\square\second}$ \\
$Q$ & Wärmeenergie/Wärmeleistung & $\si{\joule}$ & \\
}%

\end{minipage}
$m = \cfrac{F_G}{g}$ - $g = \cfrac{F_G}{m}$ \\


\subsection{Körper und Temperaturänderung}
\subsubsection{Längenausdehnung bei Temperaturänderung}
\begin{minipage}{0.45\textwidth}
\mainformular{$\Delta l = \alpha \cdot l_0 \cdot \Delta t$}  \\
$\alpha = \cfrac{\Delta l}{l_0 \cdot \Delta t}$
\end{minipage}
\begin{minipage}{0.45\textwidth}

\legende{
$l$ & Länge & $\si{\meter} $ & $ $ \\
$\alpha$ & Längenausdehnungskoeffizient &  & \\
$t$ & Zeit & $\si{\second}$ & \\
}

\end{minipage}

\textbf{Resultierende Gesamtlänge} bei Längenausdehnung, proportional: \\
$l = l_{0} \cdot \left( 1 + \alpha \cdot \Delta t \right) $

\subsubsection{Dichte in Abhängigkeit der Temperatur}
\begin{minipage}{0.45\textwidth}
\mainformular{$\rho = \cfrac{\rho_0}{1 + \gamma \cdot \Delta t}$ }  \\

\end{minipage}
\begin{minipage}{0.45\textwidth}

\legende{
$\rho$ & Druck & $\si{\pascal} $ & $\si{\newton\per\square\metre}$ \\
$\gamma $ & Volumenausdehnungskoeffizient &  & $= 3 \alpha$ \\
$t$ & Zeit & $\si{\second}$ & \\
}

\end{minipage}

\subsubsection{Gay-Lussac}
 
\textbf{bei konstantem Druck:} \\

\begin{minipage}{0.45\textwidth}
\mainformular{$\cfrac{V_1}{V_2} = \cfrac{T_1}{T_2}$ }  \\

\end{minipage}
\begin{minipage}{0.45\textwidth}

\legende{
$V$ & Volumen & $\si{\cubic\meter}$ & \\
$T$ & Temperatur & $\si{\kelvin}$  &  \\
}

\end{minipage} 

\textbf{bei konstantem Volumen:}\\

\begin{minipage}{0.45\textwidth}
\mainformular{$\cfrac{\rho_1}{\rho_2} = \cfrac{T_1}{T_2}$ }  \\

\end{minipage}
\begin{minipage}{0.45\textwidth}

\legende{
$\rho$ & Druck & $\si{\pascal} $ & $\si{\newton\per\square\metre}$ \\
$T$ & Temperatur & $\si{\kelvin}$  &  \\
}

\end{minipage}


\end{document}
