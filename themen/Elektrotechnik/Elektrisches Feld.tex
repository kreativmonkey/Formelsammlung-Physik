//subsubsection{Elektrische Energie des Kondensators - Aufgabe:EnergieKonUmU} 
$ W =\frac{1}{2}\cdot C\cdot U^{2} $\ 
$ U = \sqrt{\frac{2\cdot W}{ C}} $\ 
$ C = \frac{2\cdot W}{ U^{2} } $\ 
//subsubsection{Elektrische Energie des Kondensators - Aufgabe:EnergieKonUmC} 
$ W =\frac{1}{2}\cdot C\cdot U^{2} $\ 
$ U = \sqrt{\frac{2\cdot W}{ C}} $\ 
$ C = \frac{2\cdot W}{ U^{2} } $\ 
//subsubsection{Parallelschaltung von Kondensatoren - Aufgabe:ParallKondenLUmQ2} 
$ C_{g}  = C_{1}  + C_{2} $\ 
$ C_{1}  = C_{g}  - C_{2} $\ 
$ C_{2}  = C_{g}  - C_{1} $\ 
$ Q_{g}  = Q_{1}  + Q_{2} $\ 
$ Q_{1}  = Q_{g}  - Q_{2} $\ 
$ Q_{2}  = Q_{g}  - Q_{1} $\ 
//subsubsection{Elektrische Energie des Kondensators} 
$ W =\frac{1}{2}\cdot C\cdot U^{2} $\ 
$ U = \sqrt{\frac{2\cdot W}{ C}} $\ 
$ C = \frac{2\cdot W}{ U^{2} } $\ 
//subsubsection{Parallelschaltung von Kondensatoren - Aufgabe:ParallKondeUmC2} 
$ C_{g}  = C_{1}  + C_{2} $\ 
$ C_{1}  = C_{g}  - C_{2} $\ 
$ C_{2}  = C_{g}  - C_{1} $\ 
$ Q_{g}  = Q_{1}  + Q_{2} $\ 
$ Q_{1}  = Q_{g}  - Q_{2} $\ 
$ Q_{2}  = Q_{g}  - Q_{1} $\ 
//subsubsection{Parallelschaltung von Kondensatoren - Aufgabe:ParallKondenLUmQg} 
$ C_{g}  = C_{1}  + C_{2} $\ 
$ C_{1}  = C_{g}  - C_{2} $\ 
$ C_{2}  = C_{g}  - C_{1} $\ 
$ Q_{g}  = Q_{1}  + Q_{2} $\ 
$ Q_{1}  = Q_{g}  - Q_{2} $\ 
$ Q_{2}  = Q_{g}  - Q_{1} $\ 
//subsubsection{Parallelschaltung von Kondensatoren} 
$ C_{g}  = C_{1}  + C_{2} $\ 
$ C_{1}  = C_{g}  - C_{2} $\ 
$ C_{2}  = C_{g}  - C_{1} $\ 
$ Q_{g}  = Q_{1}  + Q_{2} $\ 
$ Q_{1}  = Q_{g}  - Q_{2} $\ 
$ Q_{2}  = Q_{g}  - Q_{1} $\ 
//subsubsection{Parallelschaltung von Kondensatoren - Aufgabe:ParallKondeUmCg} 
$ C_{g}  = C_{1}  + C_{2} $\ 
$ C_{1}  = C_{g}  - C_{2} $\ 
$ C_{2}  = C_{g}  - C_{1} $\ 
$ Q_{g}  = Q_{1}  + Q_{2} $\ 
$ Q_{1}  = Q_{g}  - Q_{2} $\ 
$ Q_{2}  = Q_{g}  - Q_{1} $\ 
//subsubsection{Elektrische Energie des Kondensators - Aufgabe:EnergieKonUmW} 
$ W =\frac{1}{2}\cdot C\cdot U^{2} $\ 
$ U = \sqrt{\frac{2\cdot W}{ C}} $\ 
$ C = \frac{2\cdot W}{ U^{2} } $\ 
//subsubsection{Parallelschaltung von Kondensatoren - Aufgabe:ParallKondeUmC1} 
$ C_{g}  = C_{1}  + C_{2} $\ 
$ C_{1}  = C_{g}  - C_{2} $\ 
$ C_{2}  = C_{g}  - C_{1} $\ 
$ Q_{g}  = Q_{1}  + Q_{2} $\ 
$ Q_{1}  = Q_{g}  - Q_{2} $\ 
$ Q_{2}  = Q_{g}  - Q_{1} $\ 
//subsubsection{Parallelschaltung von Kondensatoren - Aufgabe:ParallKondenLUmQ1} 
$ C_{g}  = C_{1}  + C_{2} $\ 
$ C_{1}  = C_{g}  - C_{2} $\ 
$ C_{2}  = C_{g}  - C_{1} $\ 
$ Q_{g}  = Q_{1}  + Q_{2} $\ 
$ Q_{1}  = Q_{g}  - Q_{2} $\ 
$ Q_{2}  = Q_{g}  - Q_{1} $\ 
//subsubsection{Reihenschaltung von Kondensatoren} 
$ C_{g}  = \frac{C_{1} \cdot C_{2} }{C_{1} +C_{2} } $\ 
$ C_{1}  = \frac{C_{2} \cdot C_{g} }{C_{2} -C_{g} } $\ 
$ C_{2}  = \frac{C_{1} \cdot C_{g} }{C_{1} -C_{g} } $\ 
$ U_{g}  = U_{1}  + U_{2} $\ 
$ U_{1}  = U_{g}  - U_{2} $\ 
$ U_{2}  = U_{g}  - U_{1} $\ 
//subsubsection{Reihenschaltung von Kondensatoren - Aufgabe:ReihenKondeUmC1} 
$ C_{g}  = \frac{C_{1} \cdot C_{2} }{C_{1} +C_{2} } $\ 
$ C_{1}  = \frac{C_{2} \cdot C_{g} }{C_{2} -C_{g} } $\ 
$ C_{2}  = \frac{C_{1} \cdot C_{g} }{C_{1} -C_{g} } $\ 
$ U_{g}  = U_{1}  + U_{2} $\ 
$ U_{1}  = U_{g}  - U_{2} $\ 
$ U_{2}  = U_{g}  - U_{1} $\ 
//subsubsection{Reihenschaltung von Kondensatoren - Aufgabe:ReihenKondeUmc2} 
$ C_{g}  = \frac{C_{1} \cdot C_{2} }{C_{1} +C_{2} } $\ 
$ C_{1}  = \frac{C_{2} \cdot C_{g} }{C_{2} -C_{g} } $\ 
$ C_{2}  = \frac{C_{1} \cdot C_{g} }{C_{1} -C_{g} } $\ 
$ U_{g}  = U_{1}  + U_{2} $\ 
$ U_{1}  = U_{g}  - U_{2} $\ 
$ U_{2}  = U_{g}  - U_{1} $\ 
//subsubsection{Reihenschaltung von Kondensatoren - Aufgabe:ReihenKondeUmCg} 
$ C_{g}  = \frac{C_{1} \cdot C_{2} }{C_{1} +C_{2} } $\ 
$ C_{1}  = \frac{C_{2} \cdot C_{g} }{C_{2} -C_{g} } $\ 
$ C_{2}  = \frac{C_{1} \cdot C_{g} }{C_{1} -C_{g} } $\ 
$ U_{g}  = U_{1}  + U_{2} $\ 
$ U_{1}  = U_{g}  - U_{2} $\ 
$ U_{2}  = U_{g}  - U_{1} $\ 
//subsubsection{Reihenschaltung von Kondensatoren - Aufgabe:ReihenKondeUUmUg} 
$ C_{g}  = \frac{C_{1} \cdot C_{2} }{C_{1} +C_{2} } $\ 
$ C_{1}  = \frac{C_{2} \cdot C_{g} }{C_{2} -C_{g} } $\ 
$ C_{2}  = \frac{C_{1} \cdot C_{g} }{C_{1} -C_{g} } $\ 
$ U_{g}  = U_{1}  + U_{2} $\ 
$ U_{1}  = U_{g}  - U_{2} $\ 
$ U_{2}  = U_{g}  - U_{1} $\ 
//subsubsection{Kapazität eines Kondensators - Aufgabe:KapazKondAdUmA} 
$ C = \frac{Q}{U} $\ 
$ Q = C\cdot U $\ 
$ U = \frac{Q}{C} $\ 
$ C = \epsilon _{0} \cdot \epsilon _{r} \cdot \frac{A}{d} $\ 
$ A = \frac{C\cdot d}{\epsilon _{0} \epsilon _{r} } $\ 
$ d = \epsilon _{0} \cdot \epsilon _{r} \cdot \frac{A}{C} $\ 
//subsubsection{Kapazität eines Kondensators - Aufgabe:KapazKondUmU} 
$ C = \frac{Q}{U} $\ 
$ Q = C\cdot U $\ 
$ U = \frac{Q}{C} $\ 
$ C = \epsilon _{0} \cdot \epsilon _{r} \cdot \frac{A}{d} $\ 
$ A = \frac{C\cdot d}{\epsilon _{0} \epsilon _{r} } $\ 
$ d = \epsilon _{0} \cdot \epsilon _{r} \cdot \frac{A}{C} $\ 
//subsubsection{Kapazität eines Kondensators - Aufgabe:KapazKondAdUmC} 
$ C = \frac{Q}{U} $\ 
$ Q = C\cdot U $\ 
$ U = \frac{Q}{C} $\ 
$ C = \epsilon _{0} \cdot \epsilon _{r} \cdot \frac{A}{d} $\ 
$ A = \frac{C\cdot d}{\epsilon _{0} \epsilon _{r} } $\ 
$ d = \epsilon _{0} \cdot \epsilon _{r} \cdot \frac{A}{C} $\ 
//subsubsection{Reihenschaltung von Kondensatoren - Aufgabe:ReihenKondeUUmU2} 
$ C_{g}  = \frac{C_{1} \cdot C_{2} }{C_{1} +C_{2} } $\ 
$ C_{1}  = \frac{C_{2} \cdot C_{g} }{C_{2} -C_{g} } $\ 
$ C_{2}  = \frac{C_{1} \cdot C_{g} }{C_{1} -C_{g} } $\ 
$ U_{g}  = U_{1}  + U_{2} $\ 
$ U_{1}  = U_{g}  - U_{2} $\ 
$ U_{2}  = U_{g}  - U_{1} $\ 
//subsubsection{Kapazität eines Kondensators - Aufgabe:KapazKondUmQ} 
$ C = \frac{Q}{U} $\ 
$ Q = C\cdot U $\ 
$ U = \frac{Q}{C} $\ 
$ C = \epsilon _{0} \cdot \epsilon _{r} \cdot \frac{A}{d} $\ 
$ A = \frac{C\cdot d}{\epsilon _{0} \epsilon _{r} } $\ 
$ d = \epsilon _{0} \cdot \epsilon _{r} \cdot \frac{A}{C} $\ 
//subsubsection{Kapazität eines Kondensators - Aufgabe:KapazKondAdUmd} 
$ C = \frac{Q}{U} $\ 
$ Q = C\cdot U $\ 
$ U = \frac{Q}{C} $\ 
$ C = \epsilon _{0} \cdot \epsilon _{r} \cdot \frac{A}{d} $\ 
$ A = \frac{C\cdot d}{\epsilon _{0} \epsilon _{r} } $\ 
$ d = \epsilon _{0} \cdot \epsilon _{r} \cdot \frac{A}{C} $\ 
//subsubsection{Reihenschaltung von Kondensatoren - Aufgabe:ReihenKondeUUmU1} 
$ C_{g}  = \frac{C_{1} \cdot C_{2} }{C_{1} +C_{2} } $\ 
$ C_{1}  = \frac{C_{2} \cdot C_{g} }{C_{2} -C_{g} } $\ 
$ C_{2}  = \frac{C_{1} \cdot C_{g} }{C_{1} -C_{g} } $\ 
$ U_{g}  = U_{1}  + U_{2} $\ 
$ U_{1}  = U_{g}  - U_{2} $\ 
$ U_{2}  = U_{g}  - U_{1} $\ 
//subsubsection{Kapazität eines Kondensators - Aufgabe:KapazKondUmC} 
$ C = \frac{Q}{U} $\ 
$ Q = C\cdot U $\ 
$ U = \frac{Q}{C} $\ 
$ C = \epsilon _{0} \cdot \epsilon _{r} \cdot \frac{A}{d} $\ 
$ A = \frac{C\cdot d}{\epsilon _{0} \epsilon _{r} } $\ 
$ d = \epsilon _{0} \cdot \epsilon _{r} \cdot \frac{A}{C} $\ 
//subsubsection{Kapazität eines Kondensators} 
$ C = \frac{Q}{U} $\ 
$ Q = C\cdot U $\ 
$ U = \frac{Q}{C} $\ 
$ C = \epsilon _{0} \cdot \epsilon _{r} \cdot \frac{A}{d} $\ 
$ A = \frac{C\cdot d}{\epsilon _{0} \epsilon _{r} } $\ 
$ d = \epsilon _{0} \cdot \epsilon _{r} \cdot \frac{A}{C} $\ 
//subsubsection{Gesetz von Coulomb} 
$ F = \frac{ 1}{4\pi \epsilon _{0} } \cdot  \frac{Q_{1} \cdot Q_{2} }{  r^{2} } $\ 
$ r = \sqrt{\frac{  1}{4\pi \epsilon _{0} } \cdot  \frac{Q_{1} \cdot Q_{2} }{  F}} $\ 
$ Q_{1}  = 4\pi \epsilon _{0}  \cdot  \frac{F\cdot r^{2} }{ Q_{2} } $\ 
//subsubsection{Gesetz von Coulomb - Aufgabe:CoulUmF} 
$ F = \frac{ 1}{4\pi \epsilon _{0} } \cdot  \frac{Q_{1} \cdot Q_{2} }{  r^{2} } $\ 
$ r = \sqrt{\frac{  1}{4\pi \epsilon _{0} } \cdot  \frac{Q_{1} \cdot Q_{2} }{  F}} $\ 
$ Q_{1}  = 4\pi \epsilon _{0}  \cdot  \frac{F\cdot r^{2} }{ Q_{2} } $\ 
//subsubsection{Elektrische Feldstärke - Aufgabe:ElektrFeldUdUmd} 
$ E = \frac{F}{Q} $\ 
$ F = E\cdot Q $\ 
$ Q = \frac{F}{E} $\ 
$ E = \frac{U}{d} $\ 
$ U = E\cdot d $\ 
$ d = \frac{U}{E} $\ 
//subsubsection{Gesetz von Coulomb - Aufgabe:CoulUmr} 
$ F = \frac{ 1}{4\pi \epsilon _{0} } \cdot  \frac{Q_{1} \cdot Q_{2} }{  r^{2} } $\ 
$ r = \sqrt{\frac{  1}{4\pi \epsilon _{0} } \cdot  \frac{Q_{1} \cdot Q_{2} }{  F}} $\ 
$ Q_{1}  = 4\pi \epsilon _{0}  \cdot  \frac{F\cdot r^{2} }{ Q_{2} } $\ 
//subsubsection{Gesetz von Coulomb - Aufgabe:CoulUmQ1} 
$ F = \frac{ 1}{4\pi \epsilon _{0} } \cdot  \frac{Q_{1} \cdot Q_{2} }{  r^{2} } $\ 
$ r = \sqrt{\frac{  1}{4\pi \epsilon _{0} } \cdot  \frac{Q_{1} \cdot Q_{2} }{  F}} $\ 
$ Q_{1}  = 4\pi \epsilon _{0}  \cdot  \frac{F\cdot r^{2} }{ Q_{2} } $\ 
//subsubsection{Elektrische Feldstärke - Aufgabe:ElektrFeldUmE} 
$ E = \frac{F}{Q} $\ 
$ F = E\cdot Q $\ 
$ Q = \frac{F}{E} $\ 
$ E = \frac{U}{d} $\ 
$ U = E\cdot d $\ 
$ d = \frac{U}{E} $\ 
//subsubsection{Elektrische Feldstärke - Aufgabe:ElektrFeldUmQ} 
$ E = \frac{F}{Q} $\ 
$ F = E\cdot Q $\ 
$ Q = \frac{F}{E} $\ 
$ E = \frac{U}{d} $\ 
$ U = E\cdot d $\ 
$ d = \frac{U}{E} $\ 
//subsubsection{Elektrische Feldstärke} 
$ E = \frac{F}{Q} $\ 
$ F = E\cdot Q $\ 
$ Q = \frac{F}{E} $\ 
$ E = \frac{U}{d} $\ 
$ U = E\cdot d $\ 
$ d = \frac{U}{E} $\ 
//subsubsection{Elektrische Feldstärke - Aufgabe:ElektrFeldUmF} 
$ E = \frac{F}{Q} $\ 
$ F = E\cdot Q $\ 
$ Q = \frac{F}{E} $\ 
$ E = \frac{U}{d} $\ 
$ U = E\cdot d $\ 
$ d = \frac{U}{E} $\ 
//subsubsection{Elektrische Feldstärke - Aufgabe:ElektrFeldUdUmE} 
$ E = \frac{F}{Q} $\ 
$ F = E\cdot Q $\ 
$ Q = \frac{F}{E} $\ 
$ E = \frac{U}{d} $\ 
$ U = E\cdot d $\ 
$ d = \frac{U}{E} $\ 
//subsubsection{Elektrische Feldstärke - Aufgabe:ElektrFeldUdUmU} 
$ E = \frac{F}{Q} $\ 
$ F = E\cdot Q $\ 
$ Q = \frac{F}{E} $\ 
$ E = \frac{U}{d} $\ 
$ U = E\cdot d $\ 
$ d = \frac{U}{E} $\ 
