\subsubsection{Elektrische Energie des Kondensators - Aufgabe:EnergieKonUmC} 
\begin{minipage}{0.45\textwidth} 
$ W =\frac{1}{2}\cdot C\cdot U^{2} $\\ 
$ U = \sqrt{\frac{2\cdot W}{ C}} $\\ 
$ C = \frac{2\cdot W}{ U^{2} } $\\ 
\end{minipage} 
\begin{minipage}{0.45\textwidth} 
 
\end{minipage} 
\subsubsection{Elektrische Energie des Kondensators - Aufgabe:EnergieKonUmW} 
\begin{minipage}{0.45\textwidth} 
$ W =\frac{1}{2}\cdot C\cdot U^{2} $\\ 
$ U = \sqrt{\frac{2\cdot W}{ C}} $\\ 
$ C = \frac{2\cdot W}{ U^{2} } $\\ 
\end{minipage} 
\begin{minipage}{0.45\textwidth} 
 
\end{minipage} 
\subsubsection{Elektrische Energie des Kondensators} 
\begin{minipage}{0.45\textwidth} 
$ W =\frac{1}{2}\cdot C\cdot U^{2} $\\ 
$ U = \sqrt{\frac{2\cdot W}{ C}} $\\ 
$ C = \frac{2\cdot W}{ U^{2} } $\\ 
\end{minipage} 
\begin{minipage}{0.45\textwidth} 
 
\end{minipage} 
\subsubsection{Parallelschaltung von Kondensatoren - Aufgabe:ParallKondenLUmQ2} 
\begin{minipage}{0.45\textwidth} 
$ C_{g}  = C_{1}  + C_{2} $\\ 
$ C_{1}  = C_{g}  - C_{2} $\\ 
$ C_{2}  = C_{g}  - C_{1} $\\ 
$ Q_{g}  = Q_{1}  + Q_{2} $\\ 
$ Q_{1}  = Q_{g}  - Q_{2} $\\ 
$ Q_{2}  = Q_{g}  - Q_{1} $\\ 
\end{minipage} 
\begin{minipage}{0.45\textwidth} 
 
\end{minipage} 
\subsubsection{Parallelschaltung von Kondensatoren - Aufgabe:ParallKondenLUmQg} 
\begin{minipage}{0.45\textwidth} 
$ C_{g}  = C_{1}  + C_{2} $\\ 
$ C_{1}  = C_{g}  - C_{2} $\\ 
$ C_{2}  = C_{g}  - C_{1} $\\ 
$ Q_{g}  = Q_{1}  + Q_{2} $\\ 
$ Q_{1}  = Q_{g}  - Q_{2} $\\ 
$ Q_{2}  = Q_{g}  - Q_{1} $\\ 
\end{minipage} 
\begin{minipage}{0.45\textwidth} 
 
\end{minipage} 
\subsubsection{Parallelschaltung von Kondensatoren - Aufgabe:ParallKondeUmC2} 
\begin{minipage}{0.45\textwidth} 
$ C_{g}  = C_{1}  + C_{2} $\\ 
$ C_{1}  = C_{g}  - C_{2} $\\ 
$ C_{2}  = C_{g}  - C_{1} $\\ 
$ Q_{g}  = Q_{1}  + Q_{2} $\\ 
$ Q_{1}  = Q_{g}  - Q_{2} $\\ 
$ Q_{2}  = Q_{g}  - Q_{1} $\\ 
\end{minipage} 
\begin{minipage}{0.45\textwidth} 
 
\end{minipage} 
\subsubsection{Parallelschaltung von Kondensatoren - Aufgabe:ParallKondenLUmQ1} 
\begin{minipage}{0.45\textwidth} 
$ C_{g}  = C_{1}  + C_{2} $\\ 
$ C_{1}  = C_{g}  - C_{2} $\\ 
$ C_{2}  = C_{g}  - C_{1} $\\ 
$ Q_{g}  = Q_{1}  + Q_{2} $\\ 
$ Q_{1}  = Q_{g}  - Q_{2} $\\ 
$ Q_{2}  = Q_{g}  - Q_{1} $\\ 
\end{minipage} 
\begin{minipage}{0.45\textwidth} 
 
\end{minipage} 
\subsubsection{Parallelschaltung von Kondensatoren - Aufgabe:ParallKondeUmCg} 
\begin{minipage}{0.45\textwidth} 
$ C_{g}  = C_{1}  + C_{2} $\\ 
$ C_{1}  = C_{g}  - C_{2} $\\ 
$ C_{2}  = C_{g}  - C_{1} $\\ 
$ Q_{g}  = Q_{1}  + Q_{2} $\\ 
$ Q_{1}  = Q_{g}  - Q_{2} $\\ 
$ Q_{2}  = Q_{g}  - Q_{1} $\\ 
\end{minipage} 
\begin{minipage}{0.45\textwidth} 
 
\end{minipage} 
\subsubsection{Parallelschaltung von Kondensatoren - Aufgabe:ParallKondeUmC1} 
\begin{minipage}{0.45\textwidth} 
$ C_{g}  = C_{1}  + C_{2} $\\ 
$ C_{1}  = C_{g}  - C_{2} $\\ 
$ C_{2}  = C_{g}  - C_{1} $\\ 
$ Q_{g}  = Q_{1}  + Q_{2} $\\ 
$ Q_{1}  = Q_{g}  - Q_{2} $\\ 
$ Q_{2}  = Q_{g}  - Q_{1} $\\ 
\end{minipage} 
\begin{minipage}{0.45\textwidth} 
 
\end{minipage} 
\subsubsection{Parallelschaltung von Kondensatoren} 
\begin{minipage}{0.45\textwidth} 
$ C_{g}  = C_{1}  + C_{2} $\\ 
$ C_{1}  = C_{g}  - C_{2} $\\ 
$ C_{2}  = C_{g}  - C_{1} $\\ 
$ Q_{g}  = Q_{1}  + Q_{2} $\\ 
$ Q_{1}  = Q_{g}  - Q_{2} $\\ 
$ Q_{2}  = Q_{g}  - Q_{1} $\\ 
\end{minipage} 
\begin{minipage}{0.45\textwidth} 
 
\end{minipage} 
\subsubsection{Reihenschaltung von Kondensatoren - Aufgabe:ReihenKondeUUmU2} 
\begin{minipage}{0.45\textwidth} 
$ C_{g}  = \frac{C_{1} \cdot C_{2} }{C_{1} +C_{2} } $\\ 
$ C_{1}  = \frac{C_{2} \cdot C_{g} }{C_{2} -C_{g} } $\\ 
$ C_{2}  = \frac{C_{1} \cdot C_{g} }{C_{1} -C_{g} } $\\ 
$ U_{g}  = U_{1}  + U_{2} $\\ 
$ U_{1}  = U_{g}  - U_{2} $\\ 
$ U_{2}  = U_{g}  - U_{1} $\\ 
\end{minipage} 
\begin{minipage}{0.45\textwidth} 
 
\end{minipage} 
\subsubsection{Reihenschaltung von Kondensatoren - Aufgabe:ReihenKondeUUmU1} 
\begin{minipage}{0.45\textwidth} 
$ C_{g}  = \frac{C_{1} \cdot C_{2} }{C_{1} +C_{2} } $\\ 
$ C_{1}  = \frac{C_{2} \cdot C_{g} }{C_{2} -C_{g} } $\\ 
$ C_{2}  = \frac{C_{1} \cdot C_{g} }{C_{1} -C_{g} } $\\ 
$ U_{g}  = U_{1}  + U_{2} $\\ 
$ U_{1}  = U_{g}  - U_{2} $\\ 
$ U_{2}  = U_{g}  - U_{1} $\\ 
\end{minipage} 
\begin{minipage}{0.45\textwidth} 
 
\end{minipage} 
\subsubsection{Elektrische Energie des Kondensators - Aufgabe:EnergieKonUmU} 
\begin{minipage}{0.45\textwidth} 
$ W =\frac{1}{2}\cdot C\cdot U^{2} $\\ 
$ U = \sqrt{\frac{2\cdot W}{ C}} $\\ 
$ C = \frac{2\cdot W}{ U^{2} } $\\ 
\end{minipage} 
\begin{minipage}{0.45\textwidth} 
 
\end{minipage} 
\subsubsection{Reihenschaltung von Kondensatoren - Aufgabe:ReihenKondeUmc2} 
\begin{minipage}{0.45\textwidth} 
$ C_{g}  = \frac{C_{1} \cdot C_{2} }{C_{1} +C_{2} } $\\ 
$ C_{1}  = \frac{C_{2} \cdot C_{g} }{C_{2} -C_{g} } $\\ 
$ C_{2}  = \frac{C_{1} \cdot C_{g} }{C_{1} -C_{g} } $\\ 
$ U_{g}  = U_{1}  + U_{2} $\\ 
$ U_{1}  = U_{g}  - U_{2} $\\ 
$ U_{2}  = U_{g}  - U_{1} $\\ 
\end{minipage} 
\begin{minipage}{0.45\textwidth} 
 
\end{minipage} 
\subsubsection{Reihenschaltung von Kondensatoren - Aufgabe:ReihenKondeUmC1} 
\begin{minipage}{0.45\textwidth} 
$ C_{g}  = \frac{C_{1} \cdot C_{2} }{C_{1} +C_{2} } $\\ 
$ C_{1}  = \frac{C_{2} \cdot C_{g} }{C_{2} -C_{g} } $\\ 
$ C_{2}  = \frac{C_{1} \cdot C_{g} }{C_{1} -C_{g} } $\\ 
$ U_{g}  = U_{1}  + U_{2} $\\ 
$ U_{1}  = U_{g}  - U_{2} $\\ 
$ U_{2}  = U_{g}  - U_{1} $\\ 
\end{minipage} 
\begin{minipage}{0.45\textwidth} 
 
\end{minipage} 
\subsubsection{Reihenschaltung von Kondensatoren - Aufgabe:ReihenKondeUUmUg} 
\begin{minipage}{0.45\textwidth} 
$ C_{g}  = \frac{C_{1} \cdot C_{2} }{C_{1} +C_{2} } $\\ 
$ C_{1}  = \frac{C_{2} \cdot C_{g} }{C_{2} -C_{g} } $\\ 
$ C_{2}  = \frac{C_{1} \cdot C_{g} }{C_{1} -C_{g} } $\\ 
$ U_{g}  = U_{1}  + U_{2} $\\ 
$ U_{1}  = U_{g}  - U_{2} $\\ 
$ U_{2}  = U_{g}  - U_{1} $\\ 
\end{minipage} 
\begin{minipage}{0.45\textwidth} 
 
\end{minipage} 
\subsubsection{Reihenschaltung von Kondensatoren - Aufgabe:ReihenKondeUmCg} 
\begin{minipage}{0.45\textwidth} 
$ C_{g}  = \frac{C_{1} \cdot C_{2} }{C_{1} +C_{2} } $\\ 
$ C_{1}  = \frac{C_{2} \cdot C_{g} }{C_{2} -C_{g} } $\\ 
$ C_{2}  = \frac{C_{1} \cdot C_{g} }{C_{1} -C_{g} } $\\ 
$ U_{g}  = U_{1}  + U_{2} $\\ 
$ U_{1}  = U_{g}  - U_{2} $\\ 
$ U_{2}  = U_{g}  - U_{1} $\\ 
\end{minipage} 
\begin{minipage}{0.45\textwidth} 
 
\end{minipage} 
\subsubsection{Reihenschaltung von Kondensatoren} 
\begin{minipage}{0.45\textwidth} 
$ C_{g}  = \frac{C_{1} \cdot C_{2} }{C_{1} +C_{2} } $\\ 
$ C_{1}  = \frac{C_{2} \cdot C_{g} }{C_{2} -C_{g} } $\\ 
$ C_{2}  = \frac{C_{1} \cdot C_{g} }{C_{1} -C_{g} } $\\ 
$ U_{g}  = U_{1}  + U_{2} $\\ 
$ U_{1}  = U_{g}  - U_{2} $\\ 
$ U_{2}  = U_{g}  - U_{1} $\\ 
\end{minipage} 
\begin{minipage}{0.45\textwidth} 
 
\end{minipage} 
\subsubsection{Kapazität eines Kondensators - Aufgabe:KapazKondAdUmd} 
\begin{minipage}{0.45\textwidth} 
$ C = \frac{Q}{U} $\\ 
$ Q = C\cdot U $\\ 
$ U = \frac{Q}{C} $\\ 
$ C = \epsilon _{0} \cdot \epsilon _{r} \cdot \frac{A}{d} $\\ 
$ A = \frac{C\cdot d}{\epsilon _{0} \epsilon _{r} } $\\ 
$ d = \epsilon _{0} \cdot \epsilon _{r} \cdot \frac{A}{C} $\\ 
\end{minipage} 
\begin{minipage}{0.45\textwidth} 
 
\end{minipage} 
\subsubsection{Kapazität eines Kondensators - Aufgabe:KapazKondAdUmC} 
\begin{minipage}{0.45\textwidth} 
$ C = \frac{Q}{U} $\\ 
$ Q = C\cdot U $\\ 
$ U = \frac{Q}{C} $\\ 
$ C = \epsilon _{0} \cdot \epsilon _{r} \cdot \frac{A}{d} $\\ 
$ A = \frac{C\cdot d}{\epsilon _{0} \epsilon _{r} } $\\ 
$ d = \epsilon _{0} \cdot \epsilon _{r} \cdot \frac{A}{C} $\\ 
\end{minipage} 
\begin{minipage}{0.45\textwidth} 
 
\end{minipage} 
\subsubsection{Kapazität eines Kondensators - Aufgabe:KapazKondAdUmA} 
\begin{minipage}{0.45\textwidth} 
$ C = \frac{Q}{U} $\\ 
$ Q = C\cdot U $\\ 
$ U = \frac{Q}{C} $\\ 
$ C = \epsilon _{0} \cdot \epsilon _{r} \cdot \frac{A}{d} $\\ 
$ A = \frac{C\cdot d}{\epsilon _{0} \epsilon _{r} } $\\ 
$ d = \epsilon _{0} \cdot \epsilon _{r} \cdot \frac{A}{C} $\\ 
\end{minipage} 
\begin{minipage}{0.45\textwidth} 
 
\end{minipage} 
\subsubsection{Kapazität eines Kondensators - Aufgabe:KapazKondUmU} 
\begin{minipage}{0.45\textwidth} 
$ C = \frac{Q}{U} $\\ 
$ Q = C\cdot U $\\ 
$ U = \frac{Q}{C} $\\ 
$ C = \epsilon _{0} \cdot \epsilon _{r} \cdot \frac{A}{d} $\\ 
$ A = \frac{C\cdot d}{\epsilon _{0} \epsilon _{r} } $\\ 
$ d = \epsilon _{0} \cdot \epsilon _{r} \cdot \frac{A}{C} $\\ 
\end{minipage} 
\begin{minipage}{0.45\textwidth} 
 
\end{minipage} 
\subsubsection{Kapazität eines Kondensators - Aufgabe:KapazKondUmC} 
\begin{minipage}{0.45\textwidth} 
$ C = \frac{Q}{U} $\\ 
$ Q = C\cdot U $\\ 
$ U = \frac{Q}{C} $\\ 
$ C = \epsilon _{0} \cdot \epsilon _{r} \cdot \frac{A}{d} $\\ 
$ A = \frac{C\cdot d}{\epsilon _{0} \epsilon _{r} } $\\ 
$ d = \epsilon _{0} \cdot \epsilon _{r} \cdot \frac{A}{C} $\\ 
\end{minipage} 
\begin{minipage}{0.45\textwidth} 
 
\end{minipage} 
\subsubsection{Kapazität eines Kondensators} 
\begin{minipage}{0.45\textwidth} 
$ C = \frac{Q}{U} $\\ 
$ Q = C\cdot U $\\ 
$ U = \frac{Q}{C} $\\ 
$ C = \epsilon _{0} \cdot \epsilon _{r} \cdot \frac{A}{d} $\\ 
$ A = \frac{C\cdot d}{\epsilon _{0} \epsilon _{r} } $\\ 
$ d = \epsilon _{0} \cdot \epsilon _{r} \cdot \frac{A}{C} $\\ 
\end{minipage} 
\begin{minipage}{0.45\textwidth} 
 
\end{minipage} 
\subsubsection{Kapazität eines Kondensators - Aufgabe:KapazKondUmQ} 
\begin{minipage}{0.45\textwidth} 
$ C = \frac{Q}{U} $\\ 
$ Q = C\cdot U $\\ 
$ U = \frac{Q}{C} $\\ 
$ C = \epsilon _{0} \cdot \epsilon _{r} \cdot \frac{A}{d} $\\ 
$ A = \frac{C\cdot d}{\epsilon _{0} \epsilon _{r} } $\\ 
$ d = \epsilon _{0} \cdot \epsilon _{r} \cdot \frac{A}{C} $\\ 
\end{minipage} 
\begin{minipage}{0.45\textwidth} 
 
\end{minipage} 
\subsubsection{Gesetz von Coulomb - Aufgabe:CoulUmQ1} 
\begin{minipage}{0.45\textwidth} 
$ F = \frac{ 1}{4\pi \epsilon _{0} } \cdot  \frac{Q_{1} \cdot Q_{2} }{  r^{2} } $\\ 
$ r = \sqrt{\frac{  1}{4\pi \epsilon _{0} } \cdot  \frac{Q_{1} \cdot Q_{2} }{  F}} $\\ 
$ Q_{1}  = 4\pi \epsilon _{0}  \cdot  \frac{F\cdot r^{2} }{ Q_{2} } $\\ 
\end{minipage} 
\begin{minipage}{0.45\textwidth} 
 
\end{minipage} 
\subsubsection{Gesetz von Coulomb - Aufgabe:CoulUmF} 
\begin{minipage}{0.45\textwidth} 
$ F = \frac{ 1}{4\pi \epsilon _{0} } \cdot  \frac{Q_{1} \cdot Q_{2} }{  r^{2} } $\\ 
$ r = \sqrt{\frac{  1}{4\pi \epsilon _{0} } \cdot  \frac{Q_{1} \cdot Q_{2} }{  F}} $\\ 
$ Q_{1}  = 4\pi \epsilon _{0}  \cdot  \frac{F\cdot r^{2} }{ Q_{2} } $\\ 
\end{minipage} 
\begin{minipage}{0.45\textwidth} 
 
\end{minipage} 
\subsubsection{Elektrische Feldstärke - Aufgabe:ElektrFeldUdUmU} 
\begin{minipage}{0.45\textwidth} 
$ E = \frac{F}{Q} $\\ 
$ F = E\cdot Q $\\ 
$ Q = \frac{F}{E} $\\ 
$ E = \frac{U}{d} $\\ 
$ U = E\cdot d $\\ 
$ d = \frac{U}{E} $\\ 
\end{minipage} 
\begin{minipage}{0.45\textwidth} 
 
\end{minipage} 
\subsubsection{Gesetz von Coulomb - Aufgabe:CoulUmr} 
\begin{minipage}{0.45\textwidth} 
$ F = \frac{ 1}{4\pi \epsilon _{0} } \cdot  \frac{Q_{1} \cdot Q_{2} }{  r^{2} } $\\ 
$ r = \sqrt{\frac{  1}{4\pi \epsilon _{0} } \cdot  \frac{Q_{1} \cdot Q_{2} }{  F}} $\\ 
$ Q_{1}  = 4\pi \epsilon _{0}  \cdot  \frac{F\cdot r^{2} }{ Q_{2} } $\\ 
\end{minipage} 
\begin{minipage}{0.45\textwidth} 
 
\end{minipage} 
\subsubsection{Gesetz von Coulomb} 
\begin{minipage}{0.45\textwidth} 
$ F = \frac{ 1}{4\pi \epsilon _{0} } \cdot  \frac{Q_{1} \cdot Q_{2} }{  r^{2} } $\\ 
$ r = \sqrt{\frac{  1}{4\pi \epsilon _{0} } \cdot  \frac{Q_{1} \cdot Q_{2} }{  F}} $\\ 
$ Q_{1}  = 4\pi \epsilon _{0}  \cdot  \frac{F\cdot r^{2} }{ Q_{2} } $\\ 
\end{minipage} 
\begin{minipage}{0.45\textwidth} 
 
\end{minipage} 
\subsubsection{Elektrische Feldstärke - Aufgabe:ElektrFeldUdUmd} 
\begin{minipage}{0.45\textwidth} 
$ E = \frac{F}{Q} $\\ 
$ F = E\cdot Q $\\ 
$ Q = \frac{F}{E} $\\ 
$ E = \frac{U}{d} $\\ 
$ U = E\cdot d $\\ 
$ d = \frac{U}{E} $\\ 
\end{minipage} 
\begin{minipage}{0.45\textwidth} 
 
\end{minipage} 
\subsubsection{Elektrische Feldstärke - Aufgabe:ElektrFeldUdUmE} 
\begin{minipage}{0.45\textwidth} 
$ E = \frac{F}{Q} $\\ 
$ F = E\cdot Q $\\ 
$ Q = \frac{F}{E} $\\ 
$ E = \frac{U}{d} $\\ 
$ U = E\cdot d $\\ 
$ d = \frac{U}{E} $\\ 
\end{minipage} 
\begin{minipage}{0.45\textwidth} 
 
\end{minipage} 
\subsubsection{Elektrische Feldstärke - Aufgabe:ElektrFeldUmQ} 
\begin{minipage}{0.45\textwidth} 
$ E = \frac{F}{Q} $\\ 
$ F = E\cdot Q $\\ 
$ Q = \frac{F}{E} $\\ 
$ E = \frac{U}{d} $\\ 
$ U = E\cdot d $\\ 
$ d = \frac{U}{E} $\\ 
\end{minipage} 
\begin{minipage}{0.45\textwidth} 
 
\end{minipage} 
\subsubsection{Elektrische Feldstärke} 
\begin{minipage}{0.45\textwidth} 
$ E = \frac{F}{Q} $\\ 
$ F = E\cdot Q $\\ 
$ Q = \frac{F}{E} $\\ 
$ E = \frac{U}{d} $\\ 
$ U = E\cdot d $\\ 
$ d = \frac{U}{E} $\\ 
\end{minipage} 
\begin{minipage}{0.45\textwidth} 
 
\end{minipage} 
\subsubsection{Elektrische Feldstärke - Aufgabe:ElektrFeldUmE} 
\begin{minipage}{0.45\textwidth} 
$ E = \frac{F}{Q} $\\ 
$ F = E\cdot Q $\\ 
$ Q = \frac{F}{E} $\\ 
$ E = \frac{U}{d} $\\ 
$ U = E\cdot d $\\ 
$ d = \frac{U}{E} $\\ 
\end{minipage} 
\begin{minipage}{0.45\textwidth} 
 
\end{minipage} 
\subsubsection{Elektrische Feldstärke - Aufgabe:ElektrFeldUmF} 
\begin{minipage}{0.45\textwidth} 
$ E = \frac{F}{Q} $\\ 
$ F = E\cdot Q $\\ 
$ Q = \frac{F}{E} $\\ 
$ E = \frac{U}{d} $\\ 
$ U = E\cdot d $\\ 
$ d = \frac{U}{E} $\\ 
\end{minipage} 
\begin{minipage}{0.45\textwidth} 
 
\end{minipage} 
