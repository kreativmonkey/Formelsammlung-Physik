//subsubsection{Wirkleistung - Aufgabe:Wirkleis} 
$ P = U_{eff}\cdot I_{eff}\cdot cos(\phi ) $\ 
//subsubsection{Wirkleistung} 
$ P = U_{eff}\cdot I_{eff}\cdot cos(\phi ) $\ 
//subsubsection{Kapazitiver Widerstand - Aufgabe:KapaWiderUmXc} 
$ X_{C} = \frac{ 1}{\omega \cdot C} $\ 
$ C = \frac{ 1}{X_{C} \cdot \omega } $\ 
$ \omega = \frac{ 1}{X_{C} \cdot C} $\ 
//subsubsection{Kapazitiver Widerstand - Aufgabe:KapaWiderUmC} 
$ X_{C} = \frac{ 1}{\omega \cdot C} $\ 
$ C = \frac{ 1}{X_{C} \cdot \omega } $\ 
$ \omega = \frac{ 1}{X_{C} \cdot C} $\ 
//subsubsection{Kapazitiver Widerstand - Aufgabe:KapaWiderUmomega} 
$ X_{C} = \frac{ 1}{\omega \cdot C} $\ 
$ C = \frac{ 1}{X_{C} \cdot \omega } $\ 
$ \omega = \frac{ 1}{X_{C} \cdot C} $\ 
//subsubsection{Kapazitiver Widerstand} 
$ X_{C} = \frac{ 1}{\omega \cdot C} $\ 
$ C = \frac{ 1}{X_{C} \cdot \omega } $\ 
$ \omega = \frac{ 1}{X_{C} \cdot C} $\ 
//subsubsection{Induktiver Widerstand - Aufgabe:IndukWiderUmL} 
$ X_{L} =\omega \cdot L $\ 
$ L = \frac{X_{L} }{\omega } $\ 
$ \omega =\frac{X_{L} }{L} $\ 
//subsubsection{Induktiver Widerstand - Aufgabe:IndukWiderUmXL} 
$ X_{L} =\omega \cdot L $\ 
$ L = \frac{X_{L} }{\omega } $\ 
$ \omega =\frac{X_{L} }{L} $\ 
//subsubsection{Scheitel - Effektiv} 
Interaktiv\ 
$ U_{eff}  = \frac{U_{max} }{\sqrt{2}} $\ 
$ I_{max}  = \sqrt{2}\cdot I_{eff} $\ 
$ I_{eff}  = \frac{I_{max} }{\sqrt{2}} $\ 
//subsubsection{Scheitel - Effektiv  - Aufgabe:ScheitelEffektivstroUmImax} 
Interaktiv\ 
$ U_{eff}  = \frac{U_{max} }{\sqrt{2}} $\ 
$ I_{max}  = \sqrt{2}\cdot I_{eff} $\ 
$ I_{eff}  = \frac{I_{max} }{\sqrt{2}} $\ 
//subsubsection{Scheitel - Effektiv  - Aufgabe:ScheitelEffektivstroUmIeff} 
Interaktiv\ 
$ U_{eff}  = \frac{U_{max} }{\sqrt{2}} $\ 
$ I_{max}  = \sqrt{2}\cdot I_{eff} $\ 
$ I_{eff}  = \frac{I_{max} }{\sqrt{2}} $\ 
//subsubsection{Wechselspannung - Wechselstrom - Aufgabe:WechspaUmut} 
$ U_{t}  = U_{max} \cdot sin(\omega \cdot t) $\ 
$ I_{t}  = I_{max} \cdot sin(\omega \cdot t) $\ 
//subsubsection{Wechselspannung - Wechselstrom - Aufgabe:WechstromUmIt} 
$ U_{t}  = U_{max} \cdot sin(\omega \cdot t) $\ 
$ I_{t}  = I_{max} \cdot sin(\omega \cdot t) $\ 
//subsubsection{Wechselspannung - Wechselstrom} 
$ U_{t}  = U_{max} \cdot sin(\omega \cdot t) $\ 
$ I_{t}  = I_{max} \cdot sin(\omega \cdot t) $\ 
//subsubsection{Induktiver Widerstand - Aufgabe:IndukWiderUmomega} 
$ X_{L} =\omega \cdot L $\ 
$ L = \frac{X_{L} }{\omega } $\ 
$ \omega =\frac{X_{L} }{L} $\ 
//subsubsection{Scheitel - Effektiv  - Aufgabe:ScheitelEffektivspaUeff} 
Interaktiv\ 
$ U_{eff}  = \frac{U_{max} }{\sqrt{2}} $\ 
$ I_{max}  = \sqrt{2}\cdot I_{eff} $\ 
$ I_{eff}  = \frac{I_{max} }{\sqrt{2}} $\ 
//subsubsection{Induktiver Widerstand} 
$ X_{L} =\omega \cdot L $\ 
$ L = \frac{X_{L} }{\omega } $\ 
$ \omega =\frac{X_{L} }{L} $\ 
