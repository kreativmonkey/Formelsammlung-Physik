\subsubsection{Wirkleistung - Aufgabe:Wirkleis} 
\begin{minipage}{0.45\textwidth} 
$ P = U_{eff}\cdot I_{eff}\cdot cos(\phi ) $\\ 
\end{minipage} 
\begin{minipage}{0.45\textwidth} 
 
\end{minipage} 
\subsubsection{Wirkleistung} 
\begin{minipage}{0.45\textwidth} 
$ P = U_{eff}\cdot I_{eff}\cdot cos(\phi ) $\\ 
\end{minipage} 
\begin{minipage}{0.45\textwidth} 
 
\end{minipage} 
\subsubsection{Kapazitiver Widerstand - Aufgabe:KapaWiderUmomega} 
\begin{minipage}{0.45\textwidth} 
$ X_{C} = \frac{ 1}{\omega \cdot C} $\\ 
$ C = \frac{ 1}{X_{C} \cdot \omega } $\\ 
$ \omega = \frac{ 1}{X_{C} \cdot C} $\\ 
\end{minipage} 
\begin{minipage}{0.45\textwidth} 
 
\end{minipage} 
\subsubsection{Kapazitiver Widerstand - Aufgabe:KapaWiderUmXc} 
\begin{minipage}{0.45\textwidth} 
$ X_{C} = \frac{ 1}{\omega \cdot C} $\\ 
$ C = \frac{ 1}{X_{C} \cdot \omega } $\\ 
$ \omega = \frac{ 1}{X_{C} \cdot C} $\\ 
\end{minipage} 
\begin{minipage}{0.45\textwidth} 
 
\end{minipage} 
\subsubsection{Kapazitiver Widerstand - Aufgabe:KapaWiderUmC} 
\begin{minipage}{0.45\textwidth} 
$ X_{C} = \frac{ 1}{\omega \cdot C} $\\ 
$ C = \frac{ 1}{X_{C} \cdot \omega } $\\ 
$ \omega = \frac{ 1}{X_{C} \cdot C} $\\ 
\end{minipage} 
\begin{minipage}{0.45\textwidth} 
 
\end{minipage} 
\subsubsection{Induktiver Widerstand - Aufgabe:IndukWiderUmomega} 
\begin{minipage}{0.45\textwidth} 
$ X_{L} =\omega \cdot L $\\ 
$ L = \frac{X_{L} }{\omega } $\\ 
$ \omega =\frac{X_{L} }{L} $\\ 
\end{minipage} 
\begin{minipage}{0.45\textwidth} 
 
\end{minipage} 
\subsubsection{Induktiver Widerstand - Aufgabe:IndukWiderUmXL} 
\begin{minipage}{0.45\textwidth} 
$ X_{L} =\omega \cdot L $\\ 
$ L = \frac{X_{L} }{\omega } $\\ 
$ \omega =\frac{X_{L} }{L} $\\ 
\end{minipage} 
\begin{minipage}{0.45\textwidth} 
 
\end{minipage} 
\subsubsection{Kapazitiver Widerstand} 
\begin{minipage}{0.45\textwidth} 
$ X_{C} = \frac{ 1}{\omega \cdot C} $\\ 
$ C = \frac{ 1}{X_{C} \cdot \omega } $\\ 
$ \omega = \frac{ 1}{X_{C} \cdot C} $\\ 
\end{minipage} 
\begin{minipage}{0.45\textwidth} 
 
\end{minipage} 
\subsubsection{Induktiver Widerstand} 
\begin{minipage}{0.45\textwidth} 
$ X_{L} =\omega \cdot L $\\ 
$ L = \frac{X_{L} }{\omega } $\\ 
$ \omega =\frac{X_{L} }{L} $\\ 
\end{minipage} 
\begin{minipage}{0.45\textwidth} 
 
\end{minipage} 
\subsubsection{Scheitel - Effektiv  - Aufgabe:ScheitelEffektivstroUmIeff} 
\begin{minipage}{0.45\textwidth} 
Interaktiv\\ 
$ U_{eff}  = \frac{U_{max} }{\sqrt{2}} $\\ 
$ I_{max}  = \sqrt{2}\cdot I_{eff} $\\ 
$ I_{eff}  = \frac{I_{max} }{\sqrt{2}} $\\ 
\end{minipage} 
\begin{minipage}{0.45\textwidth} 
 
\end{minipage} 
\subsubsection{Induktiver Widerstand - Aufgabe:IndukWiderUmL} 
\begin{minipage}{0.45\textwidth} 
$ X_{L} =\omega \cdot L $\\ 
$ L = \frac{X_{L} }{\omega } $\\ 
$ \omega =\frac{X_{L} }{L} $\\ 
\end{minipage} 
\begin{minipage}{0.45\textwidth} 
 
\end{minipage} 
\subsubsection{Wechselspannung - Wechselstrom - Aufgabe:WechspaUmut} 
\begin{minipage}{0.45\textwidth} 
$ U_{t}  = U_{max} \cdot sin(\omega \cdot t) $\\ 
$ I_{t}  = I_{max} \cdot sin(\omega \cdot t) $\\ 
\end{minipage} 
\begin{minipage}{0.45\textwidth} 
 
\end{minipage} 
\subsubsection{Wechselspannung - Wechselstrom - Aufgabe:WechstromUmIt} 
\begin{minipage}{0.45\textwidth} 
$ U_{t}  = U_{max} \cdot sin(\omega \cdot t) $\\ 
$ I_{t}  = I_{max} \cdot sin(\omega \cdot t) $\\ 
\end{minipage} 
\begin{minipage}{0.45\textwidth} 
 
\end{minipage} 
\subsubsection{Scheitel - Effektiv} 
\begin{minipage}{0.45\textwidth} 
Interaktiv\\ 
$ U_{eff}  = \frac{U_{max} }{\sqrt{2}} $\\ 
$ I_{max}  = \sqrt{2}\cdot I_{eff} $\\ 
$ I_{eff}  = \frac{I_{max} }{\sqrt{2}} $\\ 
\end{minipage} 
\begin{minipage}{0.45\textwidth} 
 
\end{minipage} 
\subsubsection{Scheitel - Effektiv  - Aufgabe:ScheitelEffektivspaUeff} 
\begin{minipage}{0.45\textwidth} 
Interaktiv\\ 
$ U_{eff}  = \frac{U_{max} }{\sqrt{2}} $\\ 
$ I_{max}  = \sqrt{2}\cdot I_{eff} $\\ 
$ I_{eff}  = \frac{I_{max} }{\sqrt{2}} $\\ 
\end{minipage} 
\begin{minipage}{0.45\textwidth} 
 
\end{minipage} 
\subsubsection{Wechselspannung - Wechselstrom} 
\begin{minipage}{0.45\textwidth} 
$ U_{t}  = U_{max} \cdot sin(\omega \cdot t) $\\ 
$ I_{t}  = I_{max} \cdot sin(\omega \cdot t) $\\ 
\end{minipage} 
\begin{minipage}{0.45\textwidth} 
 
\end{minipage} 
\subsubsection{Scheitel - Effektiv  - Aufgabe:ScheitelEffektivstroUmImax} 
\begin{minipage}{0.45\textwidth} 
Interaktiv\\ 
$ U_{eff}  = \frac{U_{max} }{\sqrt{2}} $\\ 
$ I_{max}  = \sqrt{2}\cdot I_{eff} $\\ 
$ I_{eff}  = \frac{I_{max} }{\sqrt{2}} $\\ 
\end{minipage} 
\begin{minipage}{0.45\textwidth} 
 
\end{minipage} 
