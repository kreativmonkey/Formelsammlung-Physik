\subsubsection{Reihenschaltung (Induktivität) - Aufgabe:ReihenIndukUUmU1} 
\begin{minipage}{0.45\textwidth} 
$ L_{g}  = L_{1}  + L_{2} $\\ 
$ L_{1}  = L_{g}  - L_{2} $\\ 
$ L_{2}  = L_{g}  - L_{1} $\\ 
$ U_{g}  = U_{1}  + U_{2} $\\ 
$ U_{1}  = U_{g}  - U_{2} $\\ 
$ U_{2}  = U_{g}  - U_{1} $\\ 
\end{minipage} 
\begin{minipage}{0.45\textwidth} 
 
\end{minipage} 
\subsubsection{Parallelschaltung (Induktivität) - Aufgabe:ParallIndukIUmI2} 
\begin{minipage}{0.45\textwidth} 
Interaktiv\\ 
$ L_{1}  = \frac{L_{2} \cdot L_{g} }{L_{2} -L_{g} } $\\ 
$ L_{2}  = \frac{L_{1} \cdot L_{g} }{L_{1} -L_{g} } $\\ 
$ I_{g}  = I_{1}  + I_{2} $\\ 
$ I_{1}  = I_{g}  - I_{2} $\\ 
$ I_{2}  = I_{g}  - I_{1} $\\ 
\end{minipage} 
\begin{minipage}{0.45\textwidth} 
 
\end{minipage} 
\subsubsection{Parallelschaltung (Induktivität) - Aufgabe:ParallIndukUUmL2} 
\begin{minipage}{0.45\textwidth} 
Interaktiv\\ 
$ L_{1}  = \frac{L_{2} \cdot L_{g} }{L_{2} -L_{g} } $\\ 
$ L_{2}  = \frac{L_{1} \cdot L_{g} }{L_{1} -L_{g} } $\\ 
$ I_{g}  = I_{1}  + I_{2} $\\ 
$ I_{1}  = I_{g}  - I_{2} $\\ 
$ I_{2}  = I_{g}  - I_{1} $\\ 
\end{minipage} 
\begin{minipage}{0.45\textwidth} 
 
\end{minipage} 
\subsubsection{Parallelschaltung (Induktivität) - Aufgabe:ParallIndukIUmIg} 
\begin{minipage}{0.45\textwidth} 
Interaktiv\\ 
$ L_{1}  = \frac{L_{2} \cdot L_{g} }{L_{2} -L_{g} } $\\ 
$ L_{2}  = \frac{L_{1} \cdot L_{g} }{L_{1} -L_{g} } $\\ 
$ I_{g}  = I_{1}  + I_{2} $\\ 
$ I_{1}  = I_{g}  - I_{2} $\\ 
$ I_{2}  = I_{g}  - I_{1} $\\ 
\end{minipage} 
\begin{minipage}{0.45\textwidth} 
 
\end{minipage} 
\subsubsection{Parallelschaltung (Induktivität)} 
\begin{minipage}{0.45\textwidth} 
Interaktiv\\ 
$ L_{1}  = \frac{L_{2} \cdot L_{g} }{L_{2} -L_{g} } $\\ 
$ L_{2}  = \frac{L_{1} \cdot L_{g} }{L_{1} -L_{g} } $\\ 
$ I_{g}  = I_{1}  + I_{2} $\\ 
$ I_{1}  = I_{g}  - I_{2} $\\ 
$ I_{2}  = I_{g}  - I_{1} $\\ 
\end{minipage} 
\begin{minipage}{0.45\textwidth} 
 
\end{minipage} 
\subsubsection{Reihenschaltung (Induktivität) - Aufgabe:ReihenIndukUUmU2} 
\begin{minipage}{0.45\textwidth} 
$ L_{g}  = L_{1}  + L_{2} $\\ 
$ L_{1}  = L_{g}  - L_{2} $\\ 
$ L_{2}  = L_{g}  - L_{1} $\\ 
$ U_{g}  = U_{1}  + U_{2} $\\ 
$ U_{1}  = U_{g}  - U_{2} $\\ 
$ U_{2}  = U_{g}  - U_{1} $\\ 
\end{minipage} 
\begin{minipage}{0.45\textwidth} 
 
\end{minipage} 
\subsubsection{Parallelschaltung (Induktivität) - Aufgabe:ParallIndukUUmL1} 
\begin{minipage}{0.45\textwidth} 
Interaktiv\\ 
$ L_{1}  = \frac{L_{2} \cdot L_{g} }{L_{2} -L_{g} } $\\ 
$ L_{2}  = \frac{L_{1} \cdot L_{g} }{L_{1} -L_{g} } $\\ 
$ I_{g}  = I_{1}  + I_{2} $\\ 
$ I_{1}  = I_{g}  - I_{2} $\\ 
$ I_{2}  = I_{g}  - I_{1} $\\ 
\end{minipage} 
\begin{minipage}{0.45\textwidth} 
 
\end{minipage} 
\subsubsection{Parallelschaltung (Induktivität) - Aufgabe:ParallIndukIUmI1} 
\begin{minipage}{0.45\textwidth} 
Interaktiv\\ 
$ L_{1}  = \frac{L_{2} \cdot L_{g} }{L_{2} -L_{g} } $\\ 
$ L_{2}  = \frac{L_{1} \cdot L_{g} }{L_{1} -L_{g} } $\\ 
$ I_{g}  = I_{1}  + I_{2} $\\ 
$ I_{1}  = I_{g}  - I_{2} $\\ 
$ I_{2}  = I_{g}  - I_{1} $\\ 
\end{minipage} 
\begin{minipage}{0.45\textwidth} 
 
\end{minipage} 
\subsubsection{Reihenschaltung (Induktivität) - Aufgabe:ReihenIndukUmL1} 
\begin{minipage}{0.45\textwidth} 
$ L_{g}  = L_{1}  + L_{2} $\\ 
$ L_{1}  = L_{g}  - L_{2} $\\ 
$ L_{2}  = L_{g}  - L_{1} $\\ 
$ U_{g}  = U_{1}  + U_{2} $\\ 
$ U_{1}  = U_{g}  - U_{2} $\\ 
$ U_{2}  = U_{g}  - U_{1} $\\ 
\end{minipage} 
\begin{minipage}{0.45\textwidth} 
 
\end{minipage} 
\subsubsection{Reihenschaltung (Induktivität) - Aufgabe:ReihenIndukUUmUg} 
\begin{minipage}{0.45\textwidth} 
$ L_{g}  = L_{1}  + L_{2} $\\ 
$ L_{1}  = L_{g}  - L_{2} $\\ 
$ L_{2}  = L_{g}  - L_{1} $\\ 
$ U_{g}  = U_{1}  + U_{2} $\\ 
$ U_{1}  = U_{g}  - U_{2} $\\ 
$ U_{2}  = U_{g}  - U_{1} $\\ 
\end{minipage} 
\begin{minipage}{0.45\textwidth} 
 
\end{minipage} 
\subsubsection{Reihenschaltung (Induktivität) - Aufgabe:ReihenIndukUmL2} 
\begin{minipage}{0.45\textwidth} 
$ L_{g}  = L_{1}  + L_{2} $\\ 
$ L_{1}  = L_{g}  - L_{2} $\\ 
$ L_{2}  = L_{g}  - L_{1} $\\ 
$ U_{g}  = U_{1}  + U_{2} $\\ 
$ U_{1}  = U_{g}  - U_{2} $\\ 
$ U_{2}  = U_{g}  - U_{1} $\\ 
\end{minipage} 
\begin{minipage}{0.45\textwidth} 
 
\end{minipage} 
\subsubsection{Reihenschaltung (Induktivität)} 
\begin{minipage}{0.45\textwidth} 
$ L_{g}  = L_{1}  + L_{2} $\\ 
$ L_{1}  = L_{g}  - L_{2} $\\ 
$ L_{2}  = L_{g}  - L_{1} $\\ 
$ U_{g}  = U_{1}  + U_{2} $\\ 
$ U_{1}  = U_{g}  - U_{2} $\\ 
$ U_{2}  = U_{g}  - U_{1} $\\ 
\end{minipage} 
\begin{minipage}{0.45\textwidth} 
 
\end{minipage} 
\subsubsection{Reihenschaltung (Induktivität) - Aufgabe:ReihenIndukUmLges} 
\begin{minipage}{0.45\textwidth} 
$ L_{g}  = L_{1}  + L_{2} $\\ 
$ L_{1}  = L_{g}  - L_{2} $\\ 
$ L_{2}  = L_{g}  - L_{1} $\\ 
$ U_{g}  = U_{1}  + U_{2} $\\ 
$ U_{1}  = U_{g}  - U_{2} $\\ 
$ U_{2}  = U_{g}  - U_{1} $\\ 
\end{minipage} 
\begin{minipage}{0.45\textwidth} 
 
\end{minipage} 
\subsubsection{Induktivität einer langgestreckten Spule - Aufgabe:InduktlangSpuleUmN} 
\begin{minipage}{0.45\textwidth} 
$ L = \mu _{0} \cdot \mu _{r} \cdot \frac{A\cdot N^{2} }{lSP} $\\ 
$ l_{SP} = \mu _{0} \cdot \mu _{r} \cdot \frac{A\cdot N^{2} }{L} $\\ 
$ A = \frac{ L\cdot l}{\mu _{0} \cdot \mu _{r} \cdot N^{2} } $\\ 
$ N = \sqrt{\frac{ L\cdot l}{\mu _{0} \cdot \mu _{r} \cdot A}} $\\ 
\end{minipage} 
\begin{minipage}{0.45\textwidth} 
 
\end{minipage} 
\subsubsection{Induktivität einer langgestreckten Spule - Aufgabe:InduktlangSpuleUmlsp} 
\begin{minipage}{0.45\textwidth} 
$ L = \mu _{0} \cdot \mu _{r} \cdot \frac{A\cdot N^{2} }{lSP} $\\ 
$ l_{SP} = \mu _{0} \cdot \mu _{r} \cdot \frac{A\cdot N^{2} }{L} $\\ 
$ A = \frac{ L\cdot l}{\mu _{0} \cdot \mu _{r} \cdot N^{2} } $\\ 
$ N = \sqrt{\frac{ L\cdot l}{\mu _{0} \cdot \mu _{r} \cdot A}} $\\ 
\end{minipage} 
\begin{minipage}{0.45\textwidth} 
 
\end{minipage} 
\subsubsection{Induktivität einer langgestreckten Spule - Aufgabe:InduktlangSpuleUmL} 
\begin{minipage}{0.45\textwidth} 
$ L = \mu _{0} \cdot \mu _{r} \cdot \frac{A\cdot N^{2} }{lSP} $\\ 
$ l_{SP} = \mu _{0} \cdot \mu _{r} \cdot \frac{A\cdot N^{2} }{L} $\\ 
$ A = \frac{ L\cdot l}{\mu _{0} \cdot \mu _{r} \cdot N^{2} } $\\ 
$ N = \sqrt{\frac{ L\cdot l}{\mu _{0} \cdot \mu _{r} \cdot A}} $\\ 
\end{minipage} 
\begin{minipage}{0.45\textwidth} 
 
\end{minipage} 
\subsubsection{Induktivität einer langgestreckten Spule - Aufgabe:InduktlangSpuleUmA} 
\begin{minipage}{0.45\textwidth} 
$ L = \mu _{0} \cdot \mu _{r} \cdot \frac{A\cdot N^{2} }{lSP} $\\ 
$ l_{SP} = \mu _{0} \cdot \mu _{r} \cdot \frac{A\cdot N^{2} }{L} $\\ 
$ A = \frac{ L\cdot l}{\mu _{0} \cdot \mu _{r} \cdot N^{2} } $\\ 
$ N = \sqrt{\frac{ L\cdot l}{\mu _{0} \cdot \mu _{r} \cdot A}} $\\ 
\end{minipage} 
\begin{minipage}{0.45\textwidth} 
 
\end{minipage} 
\subsubsection{Magnetischer Fluß - Aufgabe:MagFlussUmdelta} 
\begin{minipage}{0.45\textwidth} 
$ \Phi  = B\cdot A\cdot cos(\delta ) $\\ 
$ A = \frac{ \Phi }{B\cdot cos(\delta )} $\\ 
$ B = \frac{ \Phi }{A\cdot cos(\delta )} $\\ 
$ \delta =arccos(\frac{ \Phi }{B\cdot A}) $\\ 
\end{minipage} 
\begin{minipage}{0.45\textwidth} 
 
\end{minipage} 
\subsubsection{Induktivität einer langgestreckten Spule} 
\begin{minipage}{0.45\textwidth} 
$ L = \mu _{0} \cdot \mu _{r} \cdot \frac{A\cdot N^{2} }{lSP} $\\ 
$ l_{SP} = \mu _{0} \cdot \mu _{r} \cdot \frac{A\cdot N^{2} }{L} $\\ 
$ A = \frac{ L\cdot l}{\mu _{0} \cdot \mu _{r} \cdot N^{2} } $\\ 
$ N = \sqrt{\frac{ L\cdot l}{\mu _{0} \cdot \mu _{r} \cdot A}} $\\ 
\end{minipage} 
\begin{minipage}{0.45\textwidth} 
 
\end{minipage} 
\subsubsection{Magnetischer Fluß - Aufgabe:MagFlussUmB} 
\begin{minipage}{0.45\textwidth} 
$ \Phi  = B\cdot A\cdot cos(\delta ) $\\ 
$ A = \frac{ \Phi }{B\cdot cos(\delta )} $\\ 
$ B = \frac{ \Phi }{A\cdot cos(\delta )} $\\ 
$ \delta =arccos(\frac{ \Phi }{B\cdot A}) $\\ 
\end{minipage} 
\begin{minipage}{0.45\textwidth} 
 
\end{minipage} 
\subsubsection{Magnetischer Fluß - Aufgabe:MagFlussUmA} 
\begin{minipage}{0.45\textwidth} 
$ \Phi  = B\cdot A\cdot cos(\delta ) $\\ 
$ A = \frac{ \Phi }{B\cdot cos(\delta )} $\\ 
$ B = \frac{ \Phi }{A\cdot cos(\delta )} $\\ 
$ \delta =arccos(\frac{ \Phi }{B\cdot A}) $\\ 
\end{minipage} 
\begin{minipage}{0.45\textwidth} 
 
\end{minipage} 
\subsubsection{Magnetischer Fluß - Aufgabe:MagFlussUmphi} 
\begin{minipage}{0.45\textwidth} 
$ \Phi  = B\cdot A\cdot cos(\delta ) $\\ 
$ A = \frac{ \Phi }{B\cdot cos(\delta )} $\\ 
$ B = \frac{ \Phi }{A\cdot cos(\delta )} $\\ 
$ \delta =arccos(\frac{ \Phi }{B\cdot A}) $\\ 
\end{minipage} 
\begin{minipage}{0.45\textwidth} 
 
\end{minipage} 
\subsubsection{Magnetischer Fluß} 
\begin{minipage}{0.45\textwidth} 
$ \Phi  = B\cdot A\cdot cos(\delta ) $\\ 
$ A = \frac{ \Phi }{B\cdot cos(\delta )} $\\ 
$ B = \frac{ \Phi }{A\cdot cos(\delta )} $\\ 
$ \delta =arccos(\frac{ \Phi }{B\cdot A}) $\\ 
\end{minipage} 
\begin{minipage}{0.45\textwidth} 
 
\end{minipage} 
\subsubsection{Flußdichte - Feldstärke - Aufgabe:FlussFeldUmmy0} 
\begin{minipage}{0.45\textwidth} 
$ B = \mu _{r} \cdot \mu _{0} \cdot H $\\ 
$ H =\frac{ B}{\mu _{r} \cdot \mu _{0} } $\\ 
$ \mu _{r} =\frac{ B}{\mu _{0} \cdot H} $\\ 
$ \mu _{0} =\frac{ B}{\mu _{r} \cdot H} $\\ 
\end{minipage} 
\begin{minipage}{0.45\textwidth} 
 
\end{minipage} 
\subsubsection{Flußdichte - Feldstärke - Aufgabe:FlussFeldUmH} 
\begin{minipage}{0.45\textwidth} 
$ B = \mu _{r} \cdot \mu _{0} \cdot H $\\ 
$ H =\frac{ B}{\mu _{r} \cdot \mu _{0} } $\\ 
$ \mu _{r} =\frac{ B}{\mu _{0} \cdot H} $\\ 
$ \mu _{0} =\frac{ B}{\mu _{r} \cdot H} $\\ 
\end{minipage} 
\begin{minipage}{0.45\textwidth} 
 
\end{minipage} 
\subsubsection{Flußdichte - Feldstärke - Aufgabe:FlussFeldUmmyr} 
\begin{minipage}{0.45\textwidth} 
$ B = \mu _{r} \cdot \mu _{0} \cdot H $\\ 
$ H =\frac{ B}{\mu _{r} \cdot \mu _{0} } $\\ 
$ \mu _{r} =\frac{ B}{\mu _{0} \cdot H} $\\ 
$ \mu _{0} =\frac{ B}{\mu _{r} \cdot H} $\\ 
\end{minipage} 
\begin{minipage}{0.45\textwidth} 
 
\end{minipage} 
\subsubsection{Flußdichte - Feldstärke - Aufgabe:FlussFeldUmB} 
\begin{minipage}{0.45\textwidth} 
$ B = \mu _{r} \cdot \mu _{0} \cdot H $\\ 
$ H =\frac{ B}{\mu _{r} \cdot \mu _{0} } $\\ 
$ \mu _{r} =\frac{ B}{\mu _{0} \cdot H} $\\ 
$ \mu _{0} =\frac{ B}{\mu _{r} \cdot H} $\\ 
\end{minipage} 
\begin{minipage}{0.45\textwidth} 
 
\end{minipage} 
\subsubsection{Flußdichte - Feldstärke} 
\begin{minipage}{0.45\textwidth} 
$ B = \mu _{r} \cdot \mu _{0} \cdot H $\\ 
$ H =\frac{ B}{\mu _{r} \cdot \mu _{0} } $\\ 
$ \mu _{r} =\frac{ B}{\mu _{0} \cdot H} $\\ 
$ \mu _{0} =\frac{ B}{\mu _{r} \cdot H} $\\ 
\end{minipage} 
\begin{minipage}{0.45\textwidth} 
 
\end{minipage} 
\subsubsection{Feldstärke einer langgestreckten Spule - Aufgabe:FeldlangSpuleUmN} 
\begin{minipage}{0.45\textwidth} 
$ H = \frac{I\cdot N}{ l} $\\ 
$ I = \frac{H\cdot l}{ N} $\\ 
$ N = \frac{H\cdot l}{ I} $\\ 
$ l = \frac{I\cdot N}{ H} $\\ 
\end{minipage} 
\begin{minipage}{0.45\textwidth} 
 
\end{minipage} 
\subsubsection{Feldstärke einer langgestreckten Spule - Aufgabe:FeldlangSpuleUmI} 
\begin{minipage}{0.45\textwidth} 
$ H = \frac{I\cdot N}{ l} $\\ 
$ I = \frac{H\cdot l}{ N} $\\ 
$ N = \frac{H\cdot l}{ I} $\\ 
$ l = \frac{I\cdot N}{ H} $\\ 
\end{minipage} 
\begin{minipage}{0.45\textwidth} 
 
\end{minipage} 
\subsubsection{Feldstärke einer langgestreckten Spule - Aufgabe:FeldlangSpuleUmH} 
\begin{minipage}{0.45\textwidth} 
$ H = \frac{I\cdot N}{ l} $\\ 
$ I = \frac{H\cdot l}{ N} $\\ 
$ N = \frac{H\cdot l}{ I} $\\ 
$ l = \frac{I\cdot N}{ H} $\\ 
\end{minipage} 
\begin{minipage}{0.45\textwidth} 
 
\end{minipage} 
\subsubsection{Flußdichte - Aufgabe:FlussdichteUml} 
\begin{minipage}{0.45\textwidth} 
$ B = \frac{ F}{I\cdot l} $\\ 
$ F = B\cdot I\cdot l $\\ 
$ I = \frac{ F}{B\cdot l} $\\ 
$ l = \frac{ F}{I\cdot B} $\\ 
\end{minipage} 
\begin{minipage}{0.45\textwidth} 
 
\end{minipage} 
\subsubsection{Flußdichte - Aufgabe:FlussdichteUmI} 
\begin{minipage}{0.45\textwidth} 
$ B = \frac{ F}{I\cdot l} $\\ 
$ F = B\cdot I\cdot l $\\ 
$ I = \frac{ F}{B\cdot l} $\\ 
$ l = \frac{ F}{I\cdot B} $\\ 
\end{minipage} 
\begin{minipage}{0.45\textwidth} 
 
\end{minipage} 
\subsubsection{Feldstärke einer langgestreckten Spule} 
\begin{minipage}{0.45\textwidth} 
$ H = \frac{I\cdot N}{ l} $\\ 
$ I = \frac{H\cdot l}{ N} $\\ 
$ N = \frac{H\cdot l}{ I} $\\ 
$ l = \frac{I\cdot N}{ H} $\\ 
\end{minipage} 
\begin{minipage}{0.45\textwidth} 
 
\end{minipage} 
\subsubsection{Flußdichte - Aufgabe:FlussdichteUmB} 
\begin{minipage}{0.45\textwidth} 
$ B = \frac{ F}{I\cdot l} $\\ 
$ F = B\cdot I\cdot l $\\ 
$ I = \frac{ F}{B\cdot l} $\\ 
$ l = \frac{ F}{I\cdot B} $\\ 
\end{minipage} 
\begin{minipage}{0.45\textwidth} 
 
\end{minipage} 
\subsubsection{Flußdichte} 
\begin{minipage}{0.45\textwidth} 
$ B = \frac{ F}{I\cdot l} $\\ 
$ F = B\cdot I\cdot l $\\ 
$ I = \frac{ F}{B\cdot l} $\\ 
$ l = \frac{ F}{I\cdot B} $\\ 
\end{minipage} 
\begin{minipage}{0.45\textwidth} 
 
\end{minipage} 
\subsubsection{Flußdichte - Aufgabe:FlussdichteUmf} 
\begin{minipage}{0.45\textwidth} 
$ B = \frac{ F}{I\cdot l} $\\ 
$ F = B\cdot I\cdot l $\\ 
$ I = \frac{ F}{B\cdot l} $\\ 
$ l = \frac{ F}{I\cdot B} $\\ 
\end{minipage} 
\begin{minipage}{0.45\textwidth} 
 
\end{minipage} 
\subsubsection{Feldstärke einer langgestreckten Spule - Aufgabe:FeldlangSpuleUml} 
\begin{minipage}{0.45\textwidth} 
$ H = \frac{I\cdot N}{ l} $\\ 
$ I = \frac{H\cdot l}{ N} $\\ 
$ N = \frac{H\cdot l}{ I} $\\ 
$ l = \frac{I\cdot N}{ H} $\\ 
\end{minipage} 
\begin{minipage}{0.45\textwidth} 
 
\end{minipage} 
