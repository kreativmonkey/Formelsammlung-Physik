//subsubsection{Parallelschaltung (Induktivität) - Aufgabe:ParallIndukIUmI1} 
Interaktiv\ 
$ L_{1}  = \frac{L_{2} \cdot L_{g} }{L_{2} -L_{g} } $\ 
$ L_{2}  = \frac{L_{1} \cdot L_{g} }{L_{1} -L_{g} } $\ 
$ I_{g}  = I_{1}  + I_{2} $\ 
$ I_{1}  = I_{g}  - I_{2} $\ 
$ I_{2}  = I_{g}  - I_{1} $\ 
//subsubsection{Parallelschaltung (Induktivität) - Aufgabe:ParallIndukIUmI2} 
Interaktiv\ 
$ L_{1}  = \frac{L_{2} \cdot L_{g} }{L_{2} -L_{g} } $\ 
$ L_{2}  = \frac{L_{1} \cdot L_{g} }{L_{1} -L_{g} } $\ 
$ I_{g}  = I_{1}  + I_{2} $\ 
$ I_{1}  = I_{g}  - I_{2} $\ 
$ I_{2}  = I_{g}  - I_{1} $\ 
//subsubsection{Parallelschaltung (Induktivität) - Aufgabe:ParallIndukUUmL1} 
Interaktiv\ 
$ L_{1}  = \frac{L_{2} \cdot L_{g} }{L_{2} -L_{g} } $\ 
$ L_{2}  = \frac{L_{1} \cdot L_{g} }{L_{1} -L_{g} } $\ 
$ I_{g}  = I_{1}  + I_{2} $\ 
$ I_{1}  = I_{g}  - I_{2} $\ 
$ I_{2}  = I_{g}  - I_{1} $\ 
//subsubsection{Parallelschaltung (Induktivität) - Aufgabe:ParallIndukUUmL2} 
Interaktiv\ 
$ L_{1}  = \frac{L_{2} \cdot L_{g} }{L_{2} -L_{g} } $\ 
$ L_{2}  = \frac{L_{1} \cdot L_{g} }{L_{1} -L_{g} } $\ 
$ I_{g}  = I_{1}  + I_{2} $\ 
$ I_{1}  = I_{g}  - I_{2} $\ 
$ I_{2}  = I_{g}  - I_{1} $\ 
//subsubsection{Parallelschaltung (Induktivität)} 
Interaktiv\ 
$ L_{1}  = \frac{L_{2} \cdot L_{g} }{L_{2} -L_{g} } $\ 
$ L_{2}  = \frac{L_{1} \cdot L_{g} }{L_{1} -L_{g} } $\ 
$ I_{g}  = I_{1}  + I_{2} $\ 
$ I_{1}  = I_{g}  - I_{2} $\ 
$ I_{2}  = I_{g}  - I_{1} $\ 
//subsubsection{Reihenschaltung (Induktivität) - Aufgabe:ReihenIndukUUmU2} 
$ L_{g}  = L_{1}  + L_{2} $\ 
$ L_{1}  = L_{g}  - L_{2} $\ 
$ L_{2}  = L_{g}  - L_{1} $\ 
$ U_{g}  = U_{1}  + U_{2} $\ 
$ U_{1}  = U_{g}  - U_{2} $\ 
$ U_{2}  = U_{g}  - U_{1} $\ 
//subsubsection{Reihenschaltung (Induktivität) - Aufgabe:ReihenIndukUUmUg} 
$ L_{g}  = L_{1}  + L_{2} $\ 
$ L_{1}  = L_{g}  - L_{2} $\ 
$ L_{2}  = L_{g}  - L_{1} $\ 
$ U_{g}  = U_{1}  + U_{2} $\ 
$ U_{1}  = U_{g}  - U_{2} $\ 
$ U_{2}  = U_{g}  - U_{1} $\ 
//subsubsection{Reihenschaltung (Induktivität) - Aufgabe:ReihenIndukUmL2} 
$ L_{g}  = L_{1}  + L_{2} $\ 
$ L_{1}  = L_{g}  - L_{2} $\ 
$ L_{2}  = L_{g}  - L_{1} $\ 
$ U_{g}  = U_{1}  + U_{2} $\ 
$ U_{1}  = U_{g}  - U_{2} $\ 
$ U_{2}  = U_{g}  - U_{1} $\ 
//subsubsection{Reihenschaltung (Induktivität)} 
$ L_{g}  = L_{1}  + L_{2} $\ 
$ L_{1}  = L_{g}  - L_{2} $\ 
$ L_{2}  = L_{g}  - L_{1} $\ 
$ U_{g}  = U_{1}  + U_{2} $\ 
$ U_{1}  = U_{g}  - U_{2} $\ 
$ U_{2}  = U_{g}  - U_{1} $\ 
//subsubsection{Parallelschaltung (Induktivität) - Aufgabe:ParallIndukIUmIg} 
Interaktiv\ 
$ L_{1}  = \frac{L_{2} \cdot L_{g} }{L_{2} -L_{g} } $\ 
$ L_{2}  = \frac{L_{1} \cdot L_{g} }{L_{1} -L_{g} } $\ 
$ I_{g}  = I_{1}  + I_{2} $\ 
$ I_{1}  = I_{g}  - I_{2} $\ 
$ I_{2}  = I_{g}  - I_{1} $\ 
//subsubsection{Reihenschaltung (Induktivität) - Aufgabe:ReihenIndukUmLges} 
$ L_{g}  = L_{1}  + L_{2} $\ 
$ L_{1}  = L_{g}  - L_{2} $\ 
$ L_{2}  = L_{g}  - L_{1} $\ 
$ U_{g}  = U_{1}  + U_{2} $\ 
$ U_{1}  = U_{g}  - U_{2} $\ 
$ U_{2}  = U_{g}  - U_{1} $\ 
//subsubsection{Reihenschaltung (Induktivität) - Aufgabe:ReihenIndukUmL1} 
$ L_{g}  = L_{1}  + L_{2} $\ 
$ L_{1}  = L_{g}  - L_{2} $\ 
$ L_{2}  = L_{g}  - L_{1} $\ 
$ U_{g}  = U_{1}  + U_{2} $\ 
$ U_{1}  = U_{g}  - U_{2} $\ 
$ U_{2}  = U_{g}  - U_{1} $\ 
//subsubsection{Induktivität einer langgestreckten Spule - Aufgabe:InduktlangSpuleUmL} 
$ L = \mu _{0} \cdot \mu _{r} \cdot \frac{A\cdot N^{2} }{lSP} $\ 
$ l_{SP} = \mu _{0} \cdot \mu _{r} \cdot \frac{A\cdot N^{2} }{L} $\ 
$ A = \frac{ L\cdot l}{\mu _{0} \cdot \mu _{r} \cdot N^{2} } $\ 
$ N = \sqrt{\frac{ L\cdot l}{\mu _{0} \cdot \mu _{r} \cdot A}} $\ 
//subsubsection{Reihenschaltung (Induktivität) - Aufgabe:ReihenIndukUUmU1} 
$ L_{g}  = L_{1}  + L_{2} $\ 
$ L_{1}  = L_{g}  - L_{2} $\ 
$ L_{2}  = L_{g}  - L_{1} $\ 
$ U_{g}  = U_{1}  + U_{2} $\ 
$ U_{1}  = U_{g}  - U_{2} $\ 
$ U_{2}  = U_{g}  - U_{1} $\ 
//subsubsection{Induktivität einer langgestreckten Spule - Aufgabe:InduktlangSpuleUmA} 
$ L = \mu _{0} \cdot \mu _{r} \cdot \frac{A\cdot N^{2} }{lSP} $\ 
$ l_{SP} = \mu _{0} \cdot \mu _{r} \cdot \frac{A\cdot N^{2} }{L} $\ 
$ A = \frac{ L\cdot l}{\mu _{0} \cdot \mu _{r} \cdot N^{2} } $\ 
$ N = \sqrt{\frac{ L\cdot l}{\mu _{0} \cdot \mu _{r} \cdot A}} $\ 
//subsubsection{Magnetischer Fluß - Aufgabe:MagFlussUmB} 
$ \Phi  = B\cdot A\cdot cos(\delta ) $\ 
$ A = \frac{ \Phi }{B\cdot cos(\delta )} $\ 
$ B = \frac{ \Phi }{A\cdot cos(\delta )} $\ 
$ \delta =arccos(\frac{ \Phi }{B\cdot A}) $\ 
//subsubsection{Magnetischer Fluß - Aufgabe:MagFlussUmdelta} 
$ \Phi  = B\cdot A\cdot cos(\delta ) $\ 
$ A = \frac{ \Phi }{B\cdot cos(\delta )} $\ 
$ B = \frac{ \Phi }{A\cdot cos(\delta )} $\ 
$ \delta =arccos(\frac{ \Phi }{B\cdot A}) $\ 
//subsubsection{Induktivität einer langgestreckten Spule - Aufgabe:InduktlangSpuleUmlsp} 
$ L = \mu _{0} \cdot \mu _{r} \cdot \frac{A\cdot N^{2} }{lSP} $\ 
$ l_{SP} = \mu _{0} \cdot \mu _{r} \cdot \frac{A\cdot N^{2} }{L} $\ 
$ A = \frac{ L\cdot l}{\mu _{0} \cdot \mu _{r} \cdot N^{2} } $\ 
$ N = \sqrt{\frac{ L\cdot l}{\mu _{0} \cdot \mu _{r} \cdot A}} $\ 
//subsubsection{Induktivität einer langgestreckten Spule} 
$ L = \mu _{0} \cdot \mu _{r} \cdot \frac{A\cdot N^{2} }{lSP} $\ 
$ l_{SP} = \mu _{0} \cdot \mu _{r} \cdot \frac{A\cdot N^{2} }{L} $\ 
$ A = \frac{ L\cdot l}{\mu _{0} \cdot \mu _{r} \cdot N^{2} } $\ 
$ N = \sqrt{\frac{ L\cdot l}{\mu _{0} \cdot \mu _{r} \cdot A}} $\ 
//subsubsection{Magnetischer Fluß - Aufgabe:MagFlussUmA} 
$ \Phi  = B\cdot A\cdot cos(\delta ) $\ 
$ A = \frac{ \Phi }{B\cdot cos(\delta )} $\ 
$ B = \frac{ \Phi }{A\cdot cos(\delta )} $\ 
$ \delta =arccos(\frac{ \Phi }{B\cdot A}) $\ 
//subsubsection{Induktivität einer langgestreckten Spule - Aufgabe:InduktlangSpuleUmN} 
$ L = \mu _{0} \cdot \mu _{r} \cdot \frac{A\cdot N^{2} }{lSP} $\ 
$ l_{SP} = \mu _{0} \cdot \mu _{r} \cdot \frac{A\cdot N^{2} }{L} $\ 
$ A = \frac{ L\cdot l}{\mu _{0} \cdot \mu _{r} \cdot N^{2} } $\ 
$ N = \sqrt{\frac{ L\cdot l}{\mu _{0} \cdot \mu _{r} \cdot A}} $\ 
//subsubsection{Magnetischer Fluß - Aufgabe:MagFlussUmphi} 
$ \Phi  = B\cdot A\cdot cos(\delta ) $\ 
$ A = \frac{ \Phi }{B\cdot cos(\delta )} $\ 
$ B = \frac{ \Phi }{A\cdot cos(\delta )} $\ 
$ \delta =arccos(\frac{ \Phi }{B\cdot A}) $\ 
//subsubsection{Magnetischer Fluß} 
$ \Phi  = B\cdot A\cdot cos(\delta ) $\ 
$ A = \frac{ \Phi }{B\cdot cos(\delta )} $\ 
$ B = \frac{ \Phi }{A\cdot cos(\delta )} $\ 
$ \delta =arccos(\frac{ \Phi }{B\cdot A}) $\ 
//subsubsection{Flußdichte - Feldstärke - Aufgabe:FlussFeldUmmyr} 
$ B = \mu _{r} \cdot \mu _{0} \cdot H $\ 
$ H =\frac{ B}{\mu _{r} \cdot \mu _{0} } $\ 
$ \mu _{r} =\frac{ B}{\mu _{0} \cdot H} $\ 
$ \mu _{0} =\frac{ B}{\mu _{r} \cdot H} $\ 
//subsubsection{Flußdichte - Feldstärke} 
$ B = \mu _{r} \cdot \mu _{0} \cdot H $\ 
$ H =\frac{ B}{\mu _{r} \cdot \mu _{0} } $\ 
$ \mu _{r} =\frac{ B}{\mu _{0} \cdot H} $\ 
$ \mu _{0} =\frac{ B}{\mu _{r} \cdot H} $\ 
//subsubsection{Feldstärke einer langgestreckten Spule - Aufgabe:FeldlangSpuleUmI} 
$ H = \frac{I\cdot N}{ l} $\ 
$ I = \frac{H\cdot l}{ N} $\ 
$ N = \frac{H\cdot l}{ I} $\ 
$ l = \frac{I\cdot N}{ H} $\ 
//subsubsection{Feldstärke einer langgestreckten Spule - Aufgabe:FeldlangSpuleUmH} 
$ H = \frac{I\cdot N}{ l} $\ 
$ I = \frac{H\cdot l}{ N} $\ 
$ N = \frac{H\cdot l}{ I} $\ 
$ l = \frac{I\cdot N}{ H} $\ 
//subsubsection{Feldstärke einer langgestreckten Spule - Aufgabe:FeldlangSpuleUml} 
$ H = \frac{I\cdot N}{ l} $\ 
$ I = \frac{H\cdot l}{ N} $\ 
$ N = \frac{H\cdot l}{ I} $\ 
$ l = \frac{I\cdot N}{ H} $\ 
//subsubsection{Flußdichte - Feldstärke - Aufgabe:FlussFeldUmB} 
$ B = \mu _{r} \cdot \mu _{0} \cdot H $\ 
$ H =\frac{ B}{\mu _{r} \cdot \mu _{0} } $\ 
$ \mu _{r} =\frac{ B}{\mu _{0} \cdot H} $\ 
$ \mu _{0} =\frac{ B}{\mu _{r} \cdot H} $\ 
//subsubsection{Feldstärke einer langgestreckten Spule - Aufgabe:FeldlangSpuleUmN} 
$ H = \frac{I\cdot N}{ l} $\ 
$ I = \frac{H\cdot l}{ N} $\ 
$ N = \frac{H\cdot l}{ I} $\ 
$ l = \frac{I\cdot N}{ H} $\ 
//subsubsection{Flußdichte - Feldstärke - Aufgabe:FlussFeldUmmy0} 
$ B = \mu _{r} \cdot \mu _{0} \cdot H $\ 
$ H =\frac{ B}{\mu _{r} \cdot \mu _{0} } $\ 
$ \mu _{r} =\frac{ B}{\mu _{0} \cdot H} $\ 
$ \mu _{0} =\frac{ B}{\mu _{r} \cdot H} $\ 
//subsubsection{Flußdichte - Feldstärke - Aufgabe:FlussFeldUmH} 
$ B = \mu _{r} \cdot \mu _{0} \cdot H $\ 
$ H =\frac{ B}{\mu _{r} \cdot \mu _{0} } $\ 
$ \mu _{r} =\frac{ B}{\mu _{0} \cdot H} $\ 
$ \mu _{0} =\frac{ B}{\mu _{r} \cdot H} $\ 
//subsubsection{Feldstärke einer langgestreckten Spule} 
$ H = \frac{I\cdot N}{ l} $\ 
$ I = \frac{H\cdot l}{ N} $\ 
$ N = \frac{H\cdot l}{ I} $\ 
$ l = \frac{I\cdot N}{ H} $\ 
//subsubsection{Flußdichte - Aufgabe:FlussdichteUmf} 
$ B = \frac{ F}{I\cdot l} $\ 
$ F = B\cdot I\cdot l $\ 
$ I = \frac{ F}{B\cdot l} $\ 
$ l = \frac{ F}{I\cdot B} $\ 
//subsubsection{Flußdichte} 
$ B = \frac{ F}{I\cdot l} $\ 
$ F = B\cdot I\cdot l $\ 
$ I = \frac{ F}{B\cdot l} $\ 
$ l = \frac{ F}{I\cdot B} $\ 
//subsubsection{Flußdichte - Aufgabe:FlussdichteUml} 
$ B = \frac{ F}{I\cdot l} $\ 
$ F = B\cdot I\cdot l $\ 
$ I = \frac{ F}{B\cdot l} $\ 
$ l = \frac{ F}{I\cdot B} $\ 
//subsubsection{Flußdichte - Aufgabe:FlussdichteUmI} 
$ B = \frac{ F}{I\cdot l} $\ 
$ F = B\cdot I\cdot l $\ 
$ I = \frac{ F}{B\cdot l} $\ 
$ l = \frac{ F}{I\cdot B} $\ 
//subsubsection{Flußdichte - Aufgabe:FlussdichteUmB} 
$ B = \frac{ F}{I\cdot l} $\ 
$ F = B\cdot I\cdot l $\ 
$ I = \frac{ F}{B\cdot l} $\ 
$ l = \frac{ F}{I\cdot B} $\ 
