\subsubsection{Eigenkreisfrequenz - Aufgabe:EigenkreisfreUmC} 
\begin{minipage}{0.45\textwidth} 
$ \omega  = \frac{ 1}{\sqrt{L\cdot C}} $\\ 
$ L= \frac{ 1}{\omega ^{2} \cdot C} $\\ 
$ C = \frac{ 1}{\omega ^{2} \cdot L} $\\ 
\end{minipage} 
\begin{minipage}{0.45\textwidth} 
 
\end{minipage} 
\subsubsection{Eigenkreisfrequenz - Aufgabe:EigenkreisfreUmL} 
\begin{minipage}{0.45\textwidth} 
$ \omega  = \frac{ 1}{\sqrt{L\cdot C}} $\\ 
$ L= \frac{ 1}{\omega ^{2} \cdot C} $\\ 
$ C = \frac{ 1}{\omega ^{2} \cdot L} $\\ 
\end{minipage} 
\begin{minipage}{0.45\textwidth} 
 
\end{minipage} 
\subsubsection{Eigenfrequenz (Ungedämpfte elektrische Schwingung) - Aufgabe:EigfreUmL} 
\begin{minipage}{0.45\textwidth} 
$ f = \frac{ 1}{2\cdot \pi \cdot \sqrt{L\cdot C}} $\\ 
$ L = \frac{ 1}{(2\cdot \pi \cdot f)^{2} \cdot C} $\\ 
$ C = \frac{ 1}{(2\cdot \pi \cdot f)^{2} \cdot L} $\\ 
\end{minipage} 
\begin{minipage}{0.45\textwidth} 
 
\end{minipage} 
\subsubsection{Eigenkreisfrequenz - Aufgabe:EigenkreisfreUmomega} 
\begin{minipage}{0.45\textwidth} 
$ \omega  = \frac{ 1}{\sqrt{L\cdot C}} $\\ 
$ L= \frac{ 1}{\omega ^{2} \cdot C} $\\ 
$ C = \frac{ 1}{\omega ^{2} \cdot L} $\\ 
\end{minipage} 
\begin{minipage}{0.45\textwidth} 
 
\end{minipage} 
\subsubsection{Eigenfrequenz (Ungedämpfte elektrische Schwingung) - Aufgabe:EigfreUmC} 
\begin{minipage}{0.45\textwidth} 
$ f = \frac{ 1}{2\cdot \pi \cdot \sqrt{L\cdot C}} $\\ 
$ L = \frac{ 1}{(2\cdot \pi \cdot f)^{2} \cdot C} $\\ 
$ C = \frac{ 1}{(2\cdot \pi \cdot f)^{2} \cdot L} $\\ 
\end{minipage} 
\begin{minipage}{0.45\textwidth} 
 
\end{minipage} 
\subsubsection{Eigenkreisfrequenz} 
\begin{minipage}{0.45\textwidth} 
$ \omega  = \frac{ 1}{\sqrt{L\cdot C}} $\\ 
$ L= \frac{ 1}{\omega ^{2} \cdot C} $\\ 
$ C = \frac{ 1}{\omega ^{2} \cdot L} $\\ 
\end{minipage} 
\begin{minipage}{0.45\textwidth} 
 
\end{minipage} 
\subsubsection{Eigenfrequenz (Ungedämpfte elektrische Schwingung) - Aufgabe:EigfreUmf} 
\begin{minipage}{0.45\textwidth} 
$ f = \frac{ 1}{2\cdot \pi \cdot \sqrt{L\cdot C}} $\\ 
$ L = \frac{ 1}{(2\cdot \pi \cdot f)^{2} \cdot C} $\\ 
$ C = \frac{ 1}{(2\cdot \pi \cdot f)^{2} \cdot L} $\\ 
\end{minipage} 
\begin{minipage}{0.45\textwidth} 
 
\end{minipage} 
\subsubsection{Eigenfrequenz (Ungedämpfte elektrische Schwingung)} 
\begin{minipage}{0.45\textwidth} 
$ f = \frac{ 1}{2\cdot \pi \cdot \sqrt{L\cdot C}} $\\ 
$ L = \frac{ 1}{(2\cdot \pi \cdot f)^{2} \cdot C} $\\ 
$ C = \frac{ 1}{(2\cdot \pi \cdot f)^{2} \cdot L} $\\ 
\end{minipage} 
\begin{minipage}{0.45\textwidth} 
 
\end{minipage} 
