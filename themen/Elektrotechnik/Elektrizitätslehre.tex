\subsubsection{Elektrische Leistung - Aufgabe:ElektrLeisUmI} 
\begin{minipage}{0.45\textwidth} 
$ P = U\cdot I $\\ 
$ U = \frac{P}{I} $\\ 
$ I = \frac{P}{U} $\\ 
\end{minipage} 
\begin{minipage}{0.45\textwidth} 
 
\end{minipage} 
\subsubsection{Elektrische Arbeit - Aufgabe:ElektrArbUmt} 
\begin{minipage}{0.45\textwidth} 
$ W = U\cdot I\cdot t $\\ 
$ U = \frac{W}{I\cdot t} $\\ 
$ I = \frac{W}{U\cdot t} $\\ 
$ t = \frac{ P}{U\cdot I} $\\ 
\end{minipage} 
\begin{minipage}{0.45\textwidth} 
 
\end{minipage} 
\subsubsection{Elektrische Arbeit - Aufgabe:ElektrArbUmI} 
\begin{minipage}{0.45\textwidth} 
$ W = U\cdot I\cdot t $\\ 
$ U = \frac{W}{I\cdot t} $\\ 
$ I = \frac{W}{U\cdot t} $\\ 
$ t = \frac{ P}{U\cdot I} $\\ 
\end{minipage} 
\begin{minipage}{0.45\textwidth} 
 
\end{minipage} 
\subsubsection{Elektrische Arbeit - Aufgabe:ElektrArbUmW} 
\begin{minipage}{0.45\textwidth} 
$ W = U\cdot I\cdot t $\\ 
$ U = \frac{W}{I\cdot t} $\\ 
$ I = \frac{W}{U\cdot t} $\\ 
$ t = \frac{ P}{U\cdot I} $\\ 
\end{minipage} 
\begin{minipage}{0.45\textwidth} 
 
\end{minipage} 
\subsubsection{Elektrische Arbeit} 
\begin{minipage}{0.45\textwidth} 
$ W = U\cdot I\cdot t $\\ 
$ U = \frac{W}{I\cdot t} $\\ 
$ I = \frac{W}{U\cdot t} $\\ 
$ t = \frac{ P}{U\cdot I} $\\ 
\end{minipage} 
\begin{minipage}{0.45\textwidth} 
 
\end{minipage} 
\subsubsection{Elektrische Arbeit - Aufgabe:ElektrArbUmU} 
\begin{minipage}{0.45\textwidth} 
$ W = U\cdot I\cdot t $\\ 
$ U = \frac{W}{I\cdot t} $\\ 
$ I = \frac{W}{U\cdot t} $\\ 
$ t = \frac{ P}{U\cdot I} $\\ 
\end{minipage} 
\begin{minipage}{0.45\textwidth} 
 
\end{minipage} 
\subsubsection{Elektrische Leistung - Aufgabe:ElektrLeisUmU} 
\begin{minipage}{0.45\textwidth} 
$ P = U\cdot I $\\ 
$ U = \frac{P}{I} $\\ 
$ I = \frac{P}{U} $\\ 
\end{minipage} 
\begin{minipage}{0.45\textwidth} 
 
\end{minipage} 
\subsubsection{Spezifischer Leitwert - Aufgabe:SpezifLeitUmkap} 
\begin{minipage}{0.45\textwidth} 
$ R = \frac{ l}{\kappa \cdot A} $\\ 
$ l = R\cdot \kappa \cdot A $\\ 
$ A = \frac{l}{\kappa \cdot R} $\\ 
$ \kappa  = \frac{ l}{R\cdot A} $\\ 
\end{minipage} 
\begin{minipage}{0.45\textwidth} 
 
\end{minipage} 
\subsubsection{Spezifischer Leitwert - Aufgabe:SpezifLeitUmA} 
\begin{minipage}{0.45\textwidth} 
$ R = \frac{ l}{\kappa \cdot A} $\\ 
$ l = R\cdot \kappa \cdot A $\\ 
$ A = \frac{l}{\kappa \cdot R} $\\ 
$ \kappa  = \frac{ l}{R\cdot A} $\\ 
\end{minipage} 
\begin{minipage}{0.45\textwidth} 
 
\end{minipage} 
\subsubsection{Elektrische Leistung} 
\begin{minipage}{0.45\textwidth} 
$ P = U\cdot I $\\ 
$ U = \frac{P}{I} $\\ 
$ I = \frac{P}{U} $\\ 
\end{minipage} 
\begin{minipage}{0.45\textwidth} 
 
\end{minipage} 
\subsubsection{Elektrische Leistung - Aufgabe:ElektrLeisUmP} 
\begin{minipage}{0.45\textwidth} 
$ P = U\cdot I $\\ 
$ U = \frac{P}{I} $\\ 
$ I = \frac{P}{U} $\\ 
\end{minipage} 
\begin{minipage}{0.45\textwidth} 
 
\end{minipage} 
\subsubsection{Spezifischer Leitwert} 
\begin{minipage}{0.45\textwidth} 
$ R = \frac{ l}{\kappa \cdot A} $\\ 
$ l = R\cdot \kappa \cdot A $\\ 
$ A = \frac{l}{\kappa \cdot R} $\\ 
$ \kappa  = \frac{ l}{R\cdot A} $\\ 
\end{minipage} 
\begin{minipage}{0.45\textwidth} 
 
\end{minipage} 
\subsubsection{Spezifischer Widerstand - Aufgabe:SpezifUmroh} 
\begin{minipage}{0.45\textwidth} 
$ R = \frac{\rho \cdot l}{ A} $\\ 
$ l = \frac{R\cdot A}{ \rho } $\\ 
$ \rho  = \frac{R\cdot A}{ l} $\\ 
$ A = \frac{R\cdot \rho }{ A} $\\ 
\end{minipage} 
\begin{minipage}{0.45\textwidth} 
 
\end{minipage} 
\subsubsection{Spezifischer Leitwert - Aufgabe:SpezifLeitUmR} 
\begin{minipage}{0.45\textwidth} 
$ R = \frac{ l}{\kappa \cdot A} $\\ 
$ l = R\cdot \kappa \cdot A $\\ 
$ A = \frac{l}{\kappa \cdot R} $\\ 
$ \kappa  = \frac{ l}{R\cdot A} $\\ 
\end{minipage} 
\begin{minipage}{0.45\textwidth} 
 
\end{minipage} 
\subsubsection{Spezifischer Leitwert - Aufgabe:SpezifLeitUml} 
\begin{minipage}{0.45\textwidth} 
$ R = \frac{ l}{\kappa \cdot A} $\\ 
$ l = R\cdot \kappa \cdot A $\\ 
$ A = \frac{l}{\kappa \cdot R} $\\ 
$ \kappa  = \frac{ l}{R\cdot A} $\\ 
\end{minipage} 
\begin{minipage}{0.45\textwidth} 
 
\end{minipage} 
\subsubsection{Spezifischer Widerstand - Aufgabe:SpezifUmA} 
\begin{minipage}{0.45\textwidth} 
$ R = \frac{\rho \cdot l}{ A} $\\ 
$ l = \frac{R\cdot A}{ \rho } $\\ 
$ \rho  = \frac{R\cdot A}{ l} $\\ 
$ A = \frac{R\cdot \rho }{ A} $\\ 
\end{minipage} 
\begin{minipage}{0.45\textwidth} 
 
\end{minipage} 
\subsubsection{Spezifischer Widerstand} 
\begin{minipage}{0.45\textwidth} 
$ R = \frac{\rho \cdot l}{ A} $\\ 
$ l = \frac{R\cdot A}{ \rho } $\\ 
$ \rho  = \frac{R\cdot A}{ l} $\\ 
$ A = \frac{R\cdot \rho }{ A} $\\ 
\end{minipage} 
\begin{minipage}{0.45\textwidth} 
 
\end{minipage} 
\subsubsection{Widerstandsänderung - Temperatur - Aufgabe:WideraendUmdT} 
\begin{minipage}{0.45\textwidth} 
$ \Delta R = R\cdot \alpha \cdot \Delta T $\\ 
$ \Delta R = R\cdot \alpha \cdot \Delta T $\\ 
$ \alpha  = \frac{R}{\Delta R\cdot \Delta T} $\\ 
$ \Delta T = \frac{   R}{\Delta R\cdot \alpha \cdot \Delta T} $\\ 
\end{minipage} 
\begin{minipage}{0.45\textwidth} 
 
\end{minipage} 
\subsubsection{Spezifischer Widerstand - Aufgabe:SpezifUml} 
\begin{minipage}{0.45\textwidth} 
$ R = \frac{\rho \cdot l}{ A} $\\ 
$ l = \frac{R\cdot A}{ \rho } $\\ 
$ \rho  = \frac{R\cdot A}{ l} $\\ 
$ A = \frac{R\cdot \rho }{ A} $\\ 
\end{minipage} 
\begin{minipage}{0.45\textwidth} 
 
\end{minipage} 
\subsubsection{Spezifischer Widerstand - Aufgabe:SpezifUmR} 
\begin{minipage}{0.45\textwidth} 
$ R = \frac{\rho \cdot l}{ A} $\\ 
$ l = \frac{R\cdot A}{ \rho } $\\ 
$ \rho  = \frac{R\cdot A}{ l} $\\ 
$ A = \frac{R\cdot \rho }{ A} $\\ 
\end{minipage} 
\begin{minipage}{0.45\textwidth} 
 
\end{minipage} 
\subsubsection{Widerstandsänderung - Temperatur - Aufgabe:WideraendUmalpha} 
\begin{minipage}{0.45\textwidth} 
$ \Delta R = R\cdot \alpha \cdot \Delta T $\\ 
$ \Delta R = R\cdot \alpha \cdot \Delta T $\\ 
$ \alpha  = \frac{R}{\Delta R\cdot \Delta T} $\\ 
$ \Delta T = \frac{   R}{\Delta R\cdot \alpha \cdot \Delta T} $\\ 
\end{minipage} 
\begin{minipage}{0.45\textwidth} 
 
\end{minipage} 
\subsubsection{Widerstandsänderung - Temperatur - Aufgabe:WideraendUmdR} 
\begin{minipage}{0.45\textwidth} 
$ \Delta R = R\cdot \alpha \cdot \Delta T $\\ 
$ \Delta R = R\cdot \alpha \cdot \Delta T $\\ 
$ \alpha  = \frac{R}{\Delta R\cdot \Delta T} $\\ 
$ \Delta T = \frac{   R}{\Delta R\cdot \alpha \cdot \Delta T} $\\ 
\end{minipage} 
\begin{minipage}{0.45\textwidth} 
 
\end{minipage} 
\subsubsection{Parallelschaltung von Widerständen - Aufgabe:parallIUmI1} 
\begin{minipage}{0.45\textwidth} 
$ R_{g}  = \frac{R_{1} \cdot R_{2} }{R_{1} +R_{2} } $\\ 
$ R_{1}  = \frac{R_{2} \cdot R_{g} }{R_{2} -R_{g} } $\\ 
$ R_{2}  = \frac{R_{1} \cdot R_{g} }{R_{1} -R_{g} } $\\ 
$ I_{g}  = I_{1}  + I_{2} $\\ 
$ I_{1}  = I_{g}  - I_{2} $\\ 
$ I_{2}  = I_{g}  - I_{1} $\\ 
\end{minipage} 
\begin{minipage}{0.45\textwidth} 
 
\end{minipage} 
\subsubsection{Parallelschaltung von Widerständen - Aufgabe:parallIUmIG} 
\begin{minipage}{0.45\textwidth} 
$ R_{g}  = \frac{R_{1} \cdot R_{2} }{R_{1} +R_{2} } $\\ 
$ R_{1}  = \frac{R_{2} \cdot R_{g} }{R_{2} -R_{g} } $\\ 
$ R_{2}  = \frac{R_{1} \cdot R_{g} }{R_{1} -R_{g} } $\\ 
$ I_{g}  = I_{1}  + I_{2} $\\ 
$ I_{1}  = I_{g}  - I_{2} $\\ 
$ I_{2}  = I_{g}  - I_{1} $\\ 
\end{minipage} 
\begin{minipage}{0.45\textwidth} 
 
\end{minipage} 
\subsubsection{Parallelschaltung von Widerständen - Aufgabe:parallUmR2} 
\begin{minipage}{0.45\textwidth} 
$ R_{g}  = \frac{R_{1} \cdot R_{2} }{R_{1} +R_{2} } $\\ 
$ R_{1}  = \frac{R_{2} \cdot R_{g} }{R_{2} -R_{g} } $\\ 
$ R_{2}  = \frac{R_{1} \cdot R_{g} }{R_{1} -R_{g} } $\\ 
$ I_{g}  = I_{1}  + I_{2} $\\ 
$ I_{1}  = I_{g}  - I_{2} $\\ 
$ I_{2}  = I_{g}  - I_{1} $\\ 
\end{minipage} 
\begin{minipage}{0.45\textwidth} 
 
\end{minipage} 
\subsubsection{Widerstandsänderung - Temperatur} 
\begin{minipage}{0.45\textwidth} 
$ \Delta R = R\cdot \alpha \cdot \Delta T $\\ 
$ \Delta R = R\cdot \alpha \cdot \Delta T $\\ 
$ \alpha  = \frac{R}{\Delta R\cdot \Delta T} $\\ 
$ \Delta T = \frac{   R}{\Delta R\cdot \alpha \cdot \Delta T} $\\ 
\end{minipage} 
\begin{minipage}{0.45\textwidth} 
 
\end{minipage} 
\subsubsection{Parallelschaltung von Widerständen - Aufgabe:parallIUmI2} 
\begin{minipage}{0.45\textwidth} 
$ R_{g}  = \frac{R_{1} \cdot R_{2} }{R_{1} +R_{2} } $\\ 
$ R_{1}  = \frac{R_{2} \cdot R_{g} }{R_{2} -R_{g} } $\\ 
$ R_{2}  = \frac{R_{1} \cdot R_{g} }{R_{1} -R_{g} } $\\ 
$ I_{g}  = I_{1}  + I_{2} $\\ 
$ I_{1}  = I_{g}  - I_{2} $\\ 
$ I_{2}  = I_{g}  - I_{1} $\\ 
\end{minipage} 
\begin{minipage}{0.45\textwidth} 
 
\end{minipage} 
\subsubsection{Parallelschaltung von Widerständen - Aufgabe:parallUmR1} 
\begin{minipage}{0.45\textwidth} 
$ R_{g}  = \frac{R_{1} \cdot R_{2} }{R_{1} +R_{2} } $\\ 
$ R_{1}  = \frac{R_{2} \cdot R_{g} }{R_{2} -R_{g} } $\\ 
$ R_{2}  = \frac{R_{1} \cdot R_{g} }{R_{1} -R_{g} } $\\ 
$ I_{g}  = I_{1}  + I_{2} $\\ 
$ I_{1}  = I_{g}  - I_{2} $\\ 
$ I_{2}  = I_{g}  - I_{1} $\\ 
\end{minipage} 
\begin{minipage}{0.45\textwidth} 
 
\end{minipage} 
\subsubsection{Parallelschaltung von Widerständen} 
\begin{minipage}{0.45\textwidth} 
$ R_{g}  = \frac{R_{1} \cdot R_{2} }{R_{1} +R_{2} } $\\ 
$ R_{1}  = \frac{R_{2} \cdot R_{g} }{R_{2} -R_{g} } $\\ 
$ R_{2}  = \frac{R_{1} \cdot R_{g} }{R_{1} -R_{g} } $\\ 
$ I_{g}  = I_{1}  + I_{2} $\\ 
$ I_{1}  = I_{g}  - I_{2} $\\ 
$ I_{2}  = I_{g}  - I_{1} $\\ 
\end{minipage} 
\begin{minipage}{0.45\textwidth} 
 
\end{minipage} 
\subsubsection{Parallelschaltung von Widerständen - Aufgabe:parallUmRg} 
\begin{minipage}{0.45\textwidth} 
$ R_{g}  = \frac{R_{1} \cdot R_{2} }{R_{1} +R_{2} } $\\ 
$ R_{1}  = \frac{R_{2} \cdot R_{g} }{R_{2} -R_{g} } $\\ 
$ R_{2}  = \frac{R_{1} \cdot R_{g} }{R_{1} -R_{g} } $\\ 
$ I_{g}  = I_{1}  + I_{2} $\\ 
$ I_{1}  = I_{g}  - I_{2} $\\ 
$ I_{2}  = I_{g}  - I_{1} $\\ 
\end{minipage} 
\begin{minipage}{0.45\textwidth} 
 
\end{minipage} 
\subsubsection{Reihenschaltung von Widerständen - Aufgabe:ReihenUUmU2} 
\begin{minipage}{0.45\textwidth} 
$ R_{g}  = R_{1}  + R_{2} $\\ 
$ R_{1}  = R_{g}  - R_{2} $\\ 
$ R_{2}  = R_{g}  - R_{1} $\\ 
$ U_{g}  = U_{1}  + U_{2} $\\ 
$ U_{1}  = U_{g}  - U_{2} $\\ 
$ U_{2}  = U_{g}  - U_{1} $\\ 
\end{minipage} 
\begin{minipage}{0.45\textwidth} 
 
\end{minipage} 
\subsubsection{Reihenschaltung von Widerständen - Aufgabe:ReihenUUmUg} 
\begin{minipage}{0.45\textwidth} 
$ R_{g}  = R_{1}  + R_{2} $\\ 
$ R_{1}  = R_{g}  - R_{2} $\\ 
$ R_{2}  = R_{g}  - R_{1} $\\ 
$ U_{g}  = U_{1}  + U_{2} $\\ 
$ U_{1}  = U_{g}  - U_{2} $\\ 
$ U_{2}  = U_{g}  - U_{1} $\\ 
\end{minipage} 
\begin{minipage}{0.45\textwidth} 
 
\end{minipage} 
\subsubsection{Reihenschaltung von Widerständen - Aufgabe:ReihenUmR2} 
\begin{minipage}{0.45\textwidth} 
$ R_{g}  = R_{1}  + R_{2} $\\ 
$ R_{1}  = R_{g}  - R_{2} $\\ 
$ R_{2}  = R_{g}  - R_{1} $\\ 
$ U_{g}  = U_{1}  + U_{2} $\\ 
$ U_{1}  = U_{g}  - U_{2} $\\ 
$ U_{2}  = U_{g}  - U_{1} $\\ 
\end{minipage} 
\begin{minipage}{0.45\textwidth} 
 
\end{minipage} 
\subsubsection{Reihenschaltung von Widerständen - Aufgabe:ReihenUUmU1} 
\begin{minipage}{0.45\textwidth} 
$ R_{g}  = R_{1}  + R_{2} $\\ 
$ R_{1}  = R_{g}  - R_{2} $\\ 
$ R_{2}  = R_{g}  - R_{1} $\\ 
$ U_{g}  = U_{1}  + U_{2} $\\ 
$ U_{1}  = U_{g}  - U_{2} $\\ 
$ U_{2}  = U_{g}  - U_{1} $\\ 
\end{minipage} 
\begin{minipage}{0.45\textwidth} 
 
\end{minipage} 
\subsubsection{Reihenschaltung von Widerständen} 
\begin{minipage}{0.45\textwidth} 
$ R_{g}  = R_{1}  + R_{2} $\\ 
$ R_{1}  = R_{g}  - R_{2} $\\ 
$ R_{2}  = R_{g}  - R_{1} $\\ 
$ U_{g}  = U_{1}  + U_{2} $\\ 
$ U_{1}  = U_{g}  - U_{2} $\\ 
$ U_{2}  = U_{g}  - U_{1} $\\ 
\end{minipage} 
\begin{minipage}{0.45\textwidth} 
 
\end{minipage} 
\subsubsection{Reihenschaltung von Widerständen - Aufgabe:ReihenUmr1} 
\begin{minipage}{0.45\textwidth} 
$ R_{g}  = R_{1}  + R_{2} $\\ 
$ R_{1}  = R_{g}  - R_{2} $\\ 
$ R_{2}  = R_{g}  - R_{1} $\\ 
$ U_{g}  = U_{1}  + U_{2} $\\ 
$ U_{1}  = U_{g}  - U_{2} $\\ 
$ U_{2}  = U_{g}  - U_{1} $\\ 
\end{minipage} 
\begin{minipage}{0.45\textwidth} 
 
\end{minipage} 
\subsubsection{Ohmsches Gesetz - Aufgabe:OhmUmI} 
\begin{minipage}{0.45\textwidth} 
$ R = \frac{U}{I} $\\ 
$ U = R\cdot I $\\ 
$ I = \frac{U}{R} $\\ 
\end{minipage} 
\begin{minipage}{0.45\textwidth} 
 
\end{minipage} 
\subsubsection{Reihenschaltung von Widerständen - Aufgabe:ReihenUmUg} 
\begin{minipage}{0.45\textwidth} 
$ R_{g}  = R_{1}  + R_{2} $\\ 
$ R_{1}  = R_{g}  - R_{2} $\\ 
$ R_{2}  = R_{g}  - R_{1} $\\ 
$ U_{g}  = U_{1}  + U_{2} $\\ 
$ U_{1}  = U_{g}  - U_{2} $\\ 
$ U_{2}  = U_{g}  - U_{1} $\\ 
\end{minipage} 
\begin{minipage}{0.45\textwidth} 
 
\end{minipage} 
\subsubsection{Ohmsches Gesetz} 
\begin{minipage}{0.45\textwidth} 
$ R = \frac{U}{I} $\\ 
$ U = R\cdot I $\\ 
$ I = \frac{U}{R} $\\ 
\end{minipage} 
\begin{minipage}{0.45\textwidth} 
 
\end{minipage} 
\subsubsection{Ohmsches Gesetz - Aufgabe:OhmUmR} 
\begin{minipage}{0.45\textwidth} 
$ R = \frac{U}{I} $\\ 
$ U = R\cdot I $\\ 
$ I = \frac{U}{R} $\\ 
\end{minipage} 
\begin{minipage}{0.45\textwidth} 
 
\end{minipage} 
\subsubsection{Ohmsches Gesetz - Aufgabe:OhmUmU} 
\begin{minipage}{0.45\textwidth} 
$ R = \frac{U}{I} $\\ 
$ U = R\cdot I $\\ 
$ I = \frac{U}{R} $\\ 
\end{minipage} 
\begin{minipage}{0.45\textwidth} 
 
\end{minipage} 
\subsubsection{Stromstärke - Aufgabe:StromUmt} 
\begin{minipage}{0.45\textwidth} 
$ I = \frac{\Delta Q}{\Delta t} $\\ 
$ \Delta Q =I\cdot \Delta t $\\ 
$ \Delta t = \frac{\Delta Q}{I} $\\ 
\end{minipage} 
\begin{minipage}{0.45\textwidth} 
 
\end{minipage} 
\subsubsection{Stromstärke - Aufgabe:StromUmQ} 
\begin{minipage}{0.45\textwidth} 
$ I = \frac{\Delta Q}{\Delta t} $\\ 
$ \Delta Q =I\cdot \Delta t $\\ 
$ \Delta t = \frac{\Delta Q}{I} $\\ 
\end{minipage} 
\begin{minipage}{0.45\textwidth} 
 
\end{minipage} 
\subsubsection{Stromstärke - Aufgabe:StromUmI} 
\begin{minipage}{0.45\textwidth} 
$ I = \frac{\Delta Q}{\Delta t} $\\ 
$ \Delta Q =I\cdot \Delta t $\\ 
$ \Delta t = \frac{\Delta Q}{I} $\\ 
\end{minipage} 
\begin{minipage}{0.45\textwidth} 
 
\end{minipage} 
\subsubsection{Stromstärke} 
\begin{minipage}{0.45\textwidth} 
$ I = \frac{\Delta Q}{\Delta t} $\\ 
$ \Delta Q =I\cdot \Delta t $\\ 
$ \Delta t = \frac{\Delta Q}{I} $\\ 
\end{minipage} 
\begin{minipage}{0.45\textwidth} 
 
\end{minipage} 
