//subsubsection{Elektrische Arbeit - Aufgabe:ElektrArbUmW} 
$ W = U\cdot I\cdot t $\ 
$ U = \frac{W}{I\cdot t} $\ 
$ I = \frac{W}{U\cdot t} $\ 
$ t = \frac{ P}{U\cdot I} $\ 
//subsubsection{Elektrische Arbeit} 
$ W = U\cdot I\cdot t $\ 
$ U = \frac{W}{I\cdot t} $\ 
$ I = \frac{W}{U\cdot t} $\ 
$ t = \frac{ P}{U\cdot I} $\ 
//subsubsection{Elektrische Leistung - Aufgabe:ElektrLeisUmI} 
$ P = U\cdot I $\ 
$ U = \frac{P}{I} $\ 
$ I = \frac{P}{U} $\ 
//subsubsection{Elektrische Leistung - Aufgabe:ElektrLeisUmU} 
$ P = U\cdot I $\ 
$ U = \frac{P}{I} $\ 
$ I = \frac{P}{U} $\ 
//subsubsection{Elektrische Leistung} 
$ P = U\cdot I $\ 
$ U = \frac{P}{I} $\ 
$ I = \frac{P}{U} $\ 
//subsubsection{Elektrische Leistung - Aufgabe:ElektrLeisUmP} 
$ P = U\cdot I $\ 
$ U = \frac{P}{I} $\ 
$ I = \frac{P}{U} $\ 
//subsubsection{Spezifischer Leitwert - Aufgabe:SpezifLeitUmkap} 
$ R = \frac{ l}{\kappa \cdot A} $\ 
$ l = R\cdot \kappa \cdot A $\ 
$ A = \frac{l}{\kappa \cdot R} $\ 
$ \kappa  = \frac{ l}{R\cdot A} $\ 
//subsubsection{Elektrische Arbeit - Aufgabe:ElektrArbUmI} 
$ W = U\cdot I\cdot t $\ 
$ U = \frac{W}{I\cdot t} $\ 
$ I = \frac{W}{U\cdot t} $\ 
$ t = \frac{ P}{U\cdot I} $\ 
//subsubsection{Elektrische Arbeit - Aufgabe:ElektrArbUmU} 
$ W = U\cdot I\cdot t $\ 
$ U = \frac{W}{I\cdot t} $\ 
$ I = \frac{W}{U\cdot t} $\ 
$ t = \frac{ P}{U\cdot I} $\ 
//subsubsection{Spezifischer Leitwert - Aufgabe:SpezifLeitUmA} 
$ R = \frac{ l}{\kappa \cdot A} $\ 
$ l = R\cdot \kappa \cdot A $\ 
$ A = \frac{l}{\kappa \cdot R} $\ 
$ \kappa  = \frac{ l}{R\cdot A} $\ 
//subsubsection{Spezifischer Leitwert - Aufgabe:SpezifLeitUmR} 
$ R = \frac{ l}{\kappa \cdot A} $\ 
$ l = R\cdot \kappa \cdot A $\ 
$ A = \frac{l}{\kappa \cdot R} $\ 
$ \kappa  = \frac{ l}{R\cdot A} $\ 
//subsubsection{Spezifischer Widerstand - Aufgabe:SpezifUmA} 
$ R = \frac{\rho \cdot l}{ A} $\ 
$ l = \frac{R\cdot A}{ \rho } $\ 
$ \rho  = \frac{R\cdot A}{ l} $\ 
$ A = \frac{R\cdot \rho }{ A} $\ 
//subsubsection{Spezifischer Leitwert} 
$ R = \frac{ l}{\kappa \cdot A} $\ 
$ l = R\cdot \kappa \cdot A $\ 
$ A = \frac{l}{\kappa \cdot R} $\ 
$ \kappa  = \frac{ l}{R\cdot A} $\ 
//subsubsection{Spezifischer Leitwert - Aufgabe:SpezifLeitUml} 
$ R = \frac{ l}{\kappa \cdot A} $\ 
$ l = R\cdot \kappa \cdot A $\ 
$ A = \frac{l}{\kappa \cdot R} $\ 
$ \kappa  = \frac{ l}{R\cdot A} $\ 
//subsubsection{Elektrische Arbeit - Aufgabe:ElektrArbUmt} 
$ W = U\cdot I\cdot t $\ 
$ U = \frac{W}{I\cdot t} $\ 
$ I = \frac{W}{U\cdot t} $\ 
$ t = \frac{ P}{U\cdot I} $\ 
//subsubsection{Widerstandsänderung - Temperatur - Aufgabe:WideraendUmdT} 
$ \Delta R = R\cdot \alpha \cdot \Delta T $\ 
$ \Delta R = R\cdot \alpha \cdot \Delta T $\ 
$ \alpha  = \frac{R}{\Delta R\cdot \Delta T} $\ 
$ \Delta T = \frac{   R}{\Delta R\cdot \alpha \cdot \Delta T} $\ 
//subsubsection{Widerstandsänderung - Temperatur} 
$ \Delta R = R\cdot \alpha \cdot \Delta T $\ 
$ \Delta R = R\cdot \alpha \cdot \Delta T $\ 
$ \alpha  = \frac{R}{\Delta R\cdot \Delta T} $\ 
$ \Delta T = \frac{   R}{\Delta R\cdot \alpha \cdot \Delta T} $\ 
//subsubsection{Widerstandsänderung - Temperatur - Aufgabe:WideraendUmdR} 
$ \Delta R = R\cdot \alpha \cdot \Delta T $\ 
$ \Delta R = R\cdot \alpha \cdot \Delta T $\ 
$ \alpha  = \frac{R}{\Delta R\cdot \Delta T} $\ 
$ \Delta T = \frac{   R}{\Delta R\cdot \alpha \cdot \Delta T} $\ 
//subsubsection{Parallelschaltung von Widerständen - Aufgabe:parallIUmI1} 
$ R_{g}  = \frac{R_{1} \cdot R_{2} }{R_{1} +R_{2} } $\ 
$ R_{1}  = \frac{R_{2} \cdot R_{g} }{R_{2} -R_{g} } $\ 
$ R_{2}  = \frac{R_{1} \cdot R_{g} }{R_{1} -R_{g} } $\ 
$ I_{g}  = I_{1}  + I_{2} $\ 
$ I_{1}  = I_{g}  - I_{2} $\ 
$ I_{2}  = I_{g}  - I_{1} $\ 
//subsubsection{Parallelschaltung von Widerständen - Aufgabe:parallIUmI2} 
$ R_{g}  = \frac{R_{1} \cdot R_{2} }{R_{1} +R_{2} } $\ 
$ R_{1}  = \frac{R_{2} \cdot R_{g} }{R_{2} -R_{g} } $\ 
$ R_{2}  = \frac{R_{1} \cdot R_{g} }{R_{1} -R_{g} } $\ 
$ I_{g}  = I_{1}  + I_{2} $\ 
$ I_{1}  = I_{g}  - I_{2} $\ 
$ I_{2}  = I_{g}  - I_{1} $\ 
//subsubsection{Spezifischer Widerstand - Aufgabe:SpezifUmR} 
$ R = \frac{\rho \cdot l}{ A} $\ 
$ l = \frac{R\cdot A}{ \rho } $\ 
$ \rho  = \frac{R\cdot A}{ l} $\ 
$ A = \frac{R\cdot \rho }{ A} $\ 
//subsubsection{Widerstandsänderung - Temperatur - Aufgabe:WideraendUmalpha} 
$ \Delta R = R\cdot \alpha \cdot \Delta T $\ 
$ \Delta R = R\cdot \alpha \cdot \Delta T $\ 
$ \alpha  = \frac{R}{\Delta R\cdot \Delta T} $\ 
$ \Delta T = \frac{   R}{\Delta R\cdot \alpha \cdot \Delta T} $\ 
//subsubsection{Parallelschaltung von Widerständen - Aufgabe:parallIUmIG} 
$ R_{g}  = \frac{R_{1} \cdot R_{2} }{R_{1} +R_{2} } $\ 
$ R_{1}  = \frac{R_{2} \cdot R_{g} }{R_{2} -R_{g} } $\ 
$ R_{2}  = \frac{R_{1} \cdot R_{g} }{R_{1} -R_{g} } $\ 
$ I_{g}  = I_{1}  + I_{2} $\ 
$ I_{1}  = I_{g}  - I_{2} $\ 
$ I_{2}  = I_{g}  - I_{1} $\ 
//subsubsection{Spezifischer Widerstand - Aufgabe:SpezifUmroh} 
$ R = \frac{\rho \cdot l}{ A} $\ 
$ l = \frac{R\cdot A}{ \rho } $\ 
$ \rho  = \frac{R\cdot A}{ l} $\ 
$ A = \frac{R\cdot \rho }{ A} $\ 
//subsubsection{Parallelschaltung von Widerständen - Aufgabe:parallUmR2} 
$ R_{g}  = \frac{R_{1} \cdot R_{2} }{R_{1} +R_{2} } $\ 
$ R_{1}  = \frac{R_{2} \cdot R_{g} }{R_{2} -R_{g} } $\ 
$ R_{2}  = \frac{R_{1} \cdot R_{g} }{R_{1} -R_{g} } $\ 
$ I_{g}  = I_{1}  + I_{2} $\ 
$ I_{1}  = I_{g}  - I_{2} $\ 
$ I_{2}  = I_{g}  - I_{1} $\ 
//subsubsection{Parallelschaltung von Widerständen} 
$ R_{g}  = \frac{R_{1} \cdot R_{2} }{R_{1} +R_{2} } $\ 
$ R_{1}  = \frac{R_{2} \cdot R_{g} }{R_{2} -R_{g} } $\ 
$ R_{2}  = \frac{R_{1} \cdot R_{g} }{R_{1} -R_{g} } $\ 
$ I_{g}  = I_{1}  + I_{2} $\ 
$ I_{1}  = I_{g}  - I_{2} $\ 
$ I_{2}  = I_{g}  - I_{1} $\ 
//subsubsection{Parallelschaltung von Widerständen - Aufgabe:parallUmR1} 
$ R_{g}  = \frac{R_{1} \cdot R_{2} }{R_{1} +R_{2} } $\ 
$ R_{1}  = \frac{R_{2} \cdot R_{g} }{R_{2} -R_{g} } $\ 
$ R_{2}  = \frac{R_{1} \cdot R_{g} }{R_{1} -R_{g} } $\ 
$ I_{g}  = I_{1}  + I_{2} $\ 
$ I_{1}  = I_{g}  - I_{2} $\ 
$ I_{2}  = I_{g}  - I_{1} $\ 
//subsubsection{Parallelschaltung von Widerständen - Aufgabe:parallUmRg} 
$ R_{g}  = \frac{R_{1} \cdot R_{2} }{R_{1} +R_{2} } $\ 
$ R_{1}  = \frac{R_{2} \cdot R_{g} }{R_{2} -R_{g} } $\ 
$ R_{2}  = \frac{R_{1} \cdot R_{g} }{R_{1} -R_{g} } $\ 
$ I_{g}  = I_{1}  + I_{2} $\ 
$ I_{1}  = I_{g}  - I_{2} $\ 
$ I_{2}  = I_{g}  - I_{1} $\ 
//subsubsection{Reihenschaltung von Widerständen - Aufgabe:ReihenUUmUg} 
$ R_{g}  = R_{1}  + R_{2} $\ 
$ R_{1}  = R_{g}  - R_{2} $\ 
$ R_{2}  = R_{g}  - R_{1} $\ 
$ U_{g}  = U_{1}  + U_{2} $\ 
$ U_{1}  = U_{g}  - U_{2} $\ 
$ U_{2}  = U_{g}  - U_{1} $\ 
//subsubsection{Reihenschaltung von Widerständen - Aufgabe:ReihenUmr1} 
$ R_{g}  = R_{1}  + R_{2} $\ 
$ R_{1}  = R_{g}  - R_{2} $\ 
$ R_{2}  = R_{g}  - R_{1} $\ 
$ U_{g}  = U_{1}  + U_{2} $\ 
$ U_{1}  = U_{g}  - U_{2} $\ 
$ U_{2}  = U_{g}  - U_{1} $\ 
//subsubsection{Reihenschaltung von Widerständen - Aufgabe:ReihenUmUg} 
$ R_{g}  = R_{1}  + R_{2} $\ 
$ R_{1}  = R_{g}  - R_{2} $\ 
$ R_{2}  = R_{g}  - R_{1} $\ 
$ U_{g}  = U_{1}  + U_{2} $\ 
$ U_{1}  = U_{g}  - U_{2} $\ 
$ U_{2}  = U_{g}  - U_{1} $\ 
//subsubsection{Reihenschaltung von Widerständen - Aufgabe:ReihenUUmU2} 
$ R_{g}  = R_{1}  + R_{2} $\ 
$ R_{1}  = R_{g}  - R_{2} $\ 
$ R_{2}  = R_{g}  - R_{1} $\ 
$ U_{g}  = U_{1}  + U_{2} $\ 
$ U_{1}  = U_{g}  - U_{2} $\ 
$ U_{2}  = U_{g}  - U_{1} $\ 
//subsubsection{Reihenschaltung von Widerständen} 
$ R_{g}  = R_{1}  + R_{2} $\ 
$ R_{1}  = R_{g}  - R_{2} $\ 
$ R_{2}  = R_{g}  - R_{1} $\ 
$ U_{g}  = U_{1}  + U_{2} $\ 
$ U_{1}  = U_{g}  - U_{2} $\ 
$ U_{2}  = U_{g}  - U_{1} $\ 
//subsubsection{Reihenschaltung von Widerständen - Aufgabe:ReihenUUmU1} 
$ R_{g}  = R_{1}  + R_{2} $\ 
$ R_{1}  = R_{g}  - R_{2} $\ 
$ R_{2}  = R_{g}  - R_{1} $\ 
$ U_{g}  = U_{1}  + U_{2} $\ 
$ U_{1}  = U_{g}  - U_{2} $\ 
$ U_{2}  = U_{g}  - U_{1} $\ 
//subsubsection{Spezifischer Widerstand} 
$ R = \frac{\rho \cdot l}{ A} $\ 
$ l = \frac{R\cdot A}{ \rho } $\ 
$ \rho  = \frac{R\cdot A}{ l} $\ 
$ A = \frac{R\cdot \rho }{ A} $\ 
//subsubsection{Spezifischer Widerstand - Aufgabe:SpezifUml} 
$ R = \frac{\rho \cdot l}{ A} $\ 
$ l = \frac{R\cdot A}{ \rho } $\ 
$ \rho  = \frac{R\cdot A}{ l} $\ 
$ A = \frac{R\cdot \rho }{ A} $\ 
//subsubsection{Ohmsches Gesetz - Aufgabe:OhmUmU} 
$ R = \frac{U}{I} $\ 
$ U = R\cdot I $\ 
$ I = \frac{U}{R} $\ 
//subsubsection{Stromstärke - Aufgabe:StromUmI} 
$ I = \frac{\Delta Q}{\Delta t} $\ 
$ \Delta Q =I\cdot \Delta t $\ 
$ \Delta t = \frac{\Delta Q}{I} $\ 
//subsubsection{Stromstärke - Aufgabe:StromUmQ} 
$ I = \frac{\Delta Q}{\Delta t} $\ 
$ \Delta Q =I\cdot \Delta t $\ 
$ \Delta t = \frac{\Delta Q}{I} $\ 
//subsubsection{Stromstärke} 
$ I = \frac{\Delta Q}{\Delta t} $\ 
$ \Delta Q =I\cdot \Delta t $\ 
$ \Delta t = \frac{\Delta Q}{I} $\ 
//subsubsection{Stromstärke - Aufgabe:StromUmt} 
$ I = \frac{\Delta Q}{\Delta t} $\ 
$ \Delta Q =I\cdot \Delta t $\ 
$ \Delta t = \frac{\Delta Q}{I} $\ 
//subsubsection{Ohmsches Gesetz} 
$ R = \frac{U}{I} $\ 
$ U = R\cdot I $\ 
$ I = \frac{U}{R} $\ 
//subsubsection{Ohmsches Gesetz - Aufgabe:OhmUmR} 
$ R = \frac{U}{I} $\ 
$ U = R\cdot I $\ 
$ I = \frac{U}{R} $\ 
//subsubsection{Reihenschaltung von Widerständen - Aufgabe:ReihenUmR2} 
$ R_{g}  = R_{1}  + R_{2} $\ 
$ R_{1}  = R_{g}  - R_{2} $\ 
$ R_{2}  = R_{g}  - R_{1} $\ 
$ U_{g}  = U_{1}  + U_{2} $\ 
$ U_{1}  = U_{g}  - U_{2} $\ 
$ U_{2}  = U_{g}  - U_{1} $\ 
//subsubsection{Ohmsches Gesetz - Aufgabe:OhmUmI} 
$ R = \frac{U}{I} $\ 
$ U = R\cdot I $\ 
$ I = \frac{U}{R} $\ 
