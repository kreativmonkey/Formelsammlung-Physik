\subsubsection{Elektrische Arbeit} 
\begin{minipage}{0.45\textwidth} 
\mainformular{$ W = U\cdot I\cdot t $} 
\end{minipage} 
\begin{minipage}{0.45\textwidth} 
 
\legende{}\end{minipage} 
 
$ W = U\cdot I\cdot t $ - $ U = \frac{W}{I\cdot t} $ - $ I = \frac{W}{U\cdot t} $ - $ t = \frac{ P}{U\cdot I} $ - \\ 
 
\subsubsection{Elektrische Leistung} 
\begin{minipage}{0.45\textwidth} 
\mainformular{$ P = U\cdot I $} 
\end{minipage} 
\begin{minipage}{0.45\textwidth} 
 
\legende{}\end{minipage} 
 
$ P = U\cdot I $ - $ U = \frac{P}{I} $ - $ I = \frac{P}{U} $ - \\ 
 
\subsubsection{Spezifischer Leitwert} 
\begin{minipage}{0.45\textwidth} 
\mainformular{$ R = \frac{ l}{\kappa \cdot A} $} 
\end{minipage} 
\begin{minipage}{0.45\textwidth} 
 
\legende{}\end{minipage} 
 
$ R = \frac{ l}{\kappa \cdot A} $ - $ l = R\cdot \kappa \cdot A $ - $ A = \frac{l}{\kappa \cdot R} $ - $ \kappa  = \frac{ l}{R\cdot A} $ - \\ 
 
\subsubsection{Spezifischer Widerstand} 
\begin{minipage}{0.45\textwidth} 
\mainformular{$ R = \frac{\rho \cdot l}{ A} $} 
\end{minipage} 
\begin{minipage}{0.45\textwidth} 
 
\legende{}\end{minipage} 
 
$ R = \frac{\rho \cdot l}{ A} $ - $ l = \frac{R\cdot A}{ \rho } $ - $ \rho  = \frac{R\cdot A}{ l} $ - $ A = \frac{R\cdot \rho }{ A} $ - \\ 
 
\subsubsection{Widerstandsänderung - Temperatur} 
\begin{minipage}{0.45\textwidth} 
\mainformular{$ \Delta R = R\cdot \alpha \cdot \Delta T $} 
\end{minipage} 
\begin{minipage}{0.45\textwidth} 
 
\legende{}\end{minipage} 
 
$ \Delta R = R\cdot \alpha \cdot \Delta T $ - $ \Delta R = R\cdot \alpha \cdot \Delta T $ - $ \alpha  = \frac{R}{\Delta R\cdot \Delta T} $ - $ \Delta T = \frac{   R}{\Delta R\cdot \alpha \cdot \Delta T} $ - \\ 
 
\subsubsection{Parallelschaltung von Widerständen} 
\begin{minipage}{0.45\textwidth} 
\mainformular{$ R_{g}  = \frac{R_{1} \cdot R_{2} }{R_{1} +R_{2} } $} 
\end{minipage} 
\begin{minipage}{0.45\textwidth} 
 
\legende{}\end{minipage} 
 
$ R_{g}  = \frac{R_{1} \cdot R_{2} }{R_{1} +R_{2} } $ - $ R_{1}  = \frac{R_{2} \cdot R_{g} }{R_{2} -R_{g} } $ - $ R_{2}  = \frac{R_{1} \cdot R_{g} }{R_{1} -R_{g} } $ - $ I_{g}  = I_{1}  + I_{2} $ - $ I_{1}  = I_{g}  - I_{2} $ - $ I_{2}  = I_{g}  - I_{1} $ - \\ 
 
\subsubsection{Reihenschaltung von Widerständen} 
\begin{minipage}{0.45\textwidth} 
\mainformular{$ R_{g}  = R_{1}  + R_{2} $} 
\end{minipage} 
\begin{minipage}{0.45\textwidth} 
 
\legende{}\end{minipage} 
 
$ R_{g}  = R_{1}  + R_{2} $ - $ R_{1}  = R_{g}  - R_{2} $ - $ R_{2}  = R_{g}  - R_{1} $ - $ U_{g}  = U_{1}  + U_{2} $ - $ U_{1}  = U_{g}  - U_{2} $ - $ U_{2}  = U_{g}  - U_{1} $ - \\ 
 
\subsubsection{Ohmsches Gesetz} 
\begin{minipage}{0.45\textwidth} 
\mainformular{$ R = \frac{U}{I} $} 
\end{minipage} 
\begin{minipage}{0.45\textwidth} 
 
\legende{}\end{minipage} 
 
$ R = \frac{U}{I} $ - $ U = R\cdot I $ - $ I = \frac{U}{R} $ - \\ 
 
\subsubsection{Stromstärke} 
\begin{minipage}{0.45\textwidth} 
\mainformular{$ I = \frac{\Delta Q}{\Delta t} $} 
\end{minipage} 
\begin{minipage}{0.45\textwidth} 
 
\legende{}\end{minipage} 
 
$ I = \frac{\Delta Q}{\Delta t} $ - $ \Delta Q =I\cdot \Delta t $ - $ \Delta t = \frac{\Delta Q}{I} $ - \\ 
 
