\subsubsection{Elektrische Energie des Kondensators} 
\begin{minipage}{0.45\textwidth} 
\mainformular{$ W =\frac{1}{2}\cdot C\cdot U^{2} $} 
\end{minipage} 
\begin{minipage}{0.45\textwidth} 
 
\legende{}\end{minipage} 
 
$ W =\frac{1}{2}\cdot C\cdot U^{2} $ - $ U = \sqrt{\frac{2\cdot W}{ C}} $ - $ C = \frac{2\cdot W}{ U^{2} } $ - \\ 
 
\subsubsection{Parallelschaltung von Kondensatoren} 
\begin{minipage}{0.45\textwidth} 
\mainformular{$ C_{g}  = C_{1}  + C_{2} $} 
\end{minipage} 
\begin{minipage}{0.45\textwidth} 
 
\legende{}\end{minipage} 
 
$ C_{g}  = C_{1}  + C_{2} $ - $ C_{1}  = C_{g}  - C_{2} $ - $ C_{2}  = C_{g}  - C_{1} $ - $ Q_{g}  = Q_{1}  + Q_{2} $ - $ Q_{1}  = Q_{g}  - Q_{2} $ - $ Q_{2}  = Q_{g}  - Q_{1} $ - \\ 
 
\subsubsection{Reihenschaltung von Kondensatoren} 
\begin{minipage}{0.45\textwidth} 
\mainformular{$ C_{g}  = \frac{C_{1} \cdot C_{2} }{C_{1} +C_{2} } $} 
\end{minipage} 
\begin{minipage}{0.45\textwidth} 
 
\legende{}\end{minipage} 
 
$ C_{g}  = \frac{C_{1} \cdot C_{2} }{C_{1} +C_{2} } $ - $ C_{1}  = \frac{C_{2} \cdot C_{g} }{C_{2} -C_{g} } $ - $ C_{2}  = \frac{C_{1} \cdot C_{g} }{C_{1} -C_{g} } $ - $ U_{g}  = U_{1}  + U_{2} $ - $ U_{1}  = U_{g}  - U_{2} $ - $ U_{2}  = U_{g}  - U_{1} $ - \\ 
 
\subsubsection{Gesetz von Coulomb} 
\begin{minipage}{0.45\textwidth} 
\mainformular{$ F = \frac{ 1}{4\pi \epsilon _{0} } \cdot  \frac{Q_{1} \cdot Q_{2} }{  r^{2} } $} 
\end{minipage} 
\begin{minipage}{0.45\textwidth} 
 
\legende{}\end{minipage} 
 
$ F = \frac{ 1}{4\pi \epsilon _{0} } \cdot  \frac{Q_{1} \cdot Q_{2} }{  r^{2} } $ - $ r = \sqrt{\frac{  1}{4\pi \epsilon _{0} } \cdot  \frac{Q_{1} \cdot Q_{2} }{  F}} $ - $ Q_{1}  = 4\pi \epsilon _{0}  \cdot  \frac{F\cdot r^{2} }{ Q_{2} } $ - \\ 
 
\subsubsection{Elektrische Feldstärke} 
\begin{minipage}{0.45\textwidth} 
\mainformular{$ E = \frac{F}{Q} $} 
\end{minipage} 
\begin{minipage}{0.45\textwidth} 
 
\legende{}\end{minipage} 
 
$ E = \frac{F}{Q} $ - $ F = E\cdot Q $ - $ Q = \frac{F}{E} $ - $ E = \frac{U}{d} $ - $ U = E\cdot d $ - $ d = \frac{U}{E} $ - \\ 
 
\subsubsection{Kapazität eines Kondensators} 
\begin{minipage}{0.45\textwidth} 
\mainformular{$ C = \frac{Q}{U} $} 
\end{minipage} 
\begin{minipage}{0.45\textwidth} 
 
\legende{}\end{minipage} 
 
$ C = \frac{Q}{U} $ - $ Q = C\cdot U $ - $ U = \frac{Q}{C} $ - $ C = \epsilon _{0} \cdot \epsilon _{r} \cdot \frac{A}{d} $ - $ A = \frac{C\cdot d}{\epsilon _{0} \epsilon _{r} } $ - $ d = \epsilon _{0} \cdot \epsilon _{r} \cdot \frac{A}{C} $ - \\ 
 
