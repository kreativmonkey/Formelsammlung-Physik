\subsubsection{Thermische Zustandsgleichung} 
\begin{minipage}{0.45\textwidth} 
$ p\cdot V =\nu \cdot R_{m} \cdot T $\\ 
$ p =\frac{\nu \cdot R_{m} \cdot T}{  V} $\\ 
$ V =\frac{\nu \cdot R_{m} \cdot T}{  p} $\\ 
$ T =\frac{p\cdot V}{\nu \cdot R_{m} } $\\ 
\end{minipage} 
\begin{minipage}{0.45\textwidth} 
 
\end{minipage} 
\subsubsection{Allgemeine Gasgleichung} 
\begin{minipage}{0.45\textwidth} 
$ V_{1}  = \frac{V_{2} \cdot p_{} \cdot T_{1} }{  T_{2} \cdot p_{1} } $\\ 
$ p_{1}  = \frac{V_{2} \cdot p_{2} \cdot T_{1} }{  T_{2} \cdot V_{1} } $\\ 
$ T_{1}  = \frac{V_{1} \cdot p_{1} \cdot T_{2} }{  V_{2} \cdot p_{2} } $\\ 
\end{minipage} 
\begin{minipage}{0.45\textwidth} 
 
\end{minipage} 

\subsubsection{Ideale Gasgleichung}
\begin{minipage}{0.45\textwidth}

\mainformular{$\rho \cdot V = m \cdot R \cdot T$} \\
\end{minipage}
\begin{minipage}{0.45\textwidth}

\legende{
$\rho$ & Druck & $\si{\pascal}$ & \\
$V$ & Volumen & $\si{\cubic\meter}$ & \\
$m$ & Masse & $\si{\kilogram}$ &  \\
$R$ & Spezifische Gaskonstante & $\si{\joule\per\kilogram\per\kelvin} $ & $ $ \\
$T$ & Temperatur & $\si{\kelvin}$ & \\
}

\end{minipage} 

\textbf{Universelle Gaskonstante}\\
\begin{minipage}{0.45\textwidth}
\mainformular{$R_{m} = R \cdot M$} 

\end{minipage}
\begin{minipage}{0.45\textwidth}

\legende{
$R$ & Spez Gaskonstante & $\si{\joule\per\kilogram\per\kelvin} $ & $ $ \\
$R_{m}$ & Univ. Gaskonstante & $\si{\joule\per\mol\per\kelvin} $ & $8,3144598 \si{\joule\per\mol\per\kelvin}$ \\
$M$ & Molare Masse & $\si{\gram\mol}$ & \\
}

\end{minipage}


\subsubsection{Isotherme Zustastandsänderung}
\begin{minipage}{0.45\textwidth}

\mainformular{$W = -m \cdot R \cdot T \cdot \ln{\cfrac{V_2}{V_1}}$}

$W = -m \cdot R \cdot T \cdot \ln{\cfrac{V_2}{V_1}}$ \\
$g = \cfrac{F_G}{m}$

\end{minipage}
\begin{minipage}{0.45\textwidth}

\legende{
$W$ & Arbeit & $\si{\joule}$ & \\
$m$ & Masse & $\si{\kilogram}$ &  \\
$R$ & Spez. Gaskonstante & $\si{\joule\per\kilogram\per\kelvin} $ & $ $ \\
$T$ & Temperatur & $\si{} $ & \\
}

\end{minipage}

\subsection{1. Hauptsatz}
\subsubsection{Wärmelehre}
\begin{minipage}{0.45\textwidth}

\mainformular{$\Delta U + \Delta E_{pot} + \Delta E_{kin} =  \Delta Q + \Delta W$} \\
$m = \cfrac{F_G}{g}$ \\
$g = \cfrac{F_G}{m}$

\end{minipage}
\begin{minipage}{0.45\textwidth}

\legende{
$U$ & Innere Energie & $\si{\kilo\gram}$ & \\
$E_{pot}$ & Potentielle Energie & $\si{\metre\per\square\second}$ & $9,81 \si{\metre\per\square\second}$ \\
$E_{kin}$ & Kinetische Energie & $N$ & $\si{\kilogram\metre\per\square\second}$ \\
$Q$ & Wärmeenergie/Wärmeleistung & $\si{\joule}$ & \\
}%

\end{minipage}
