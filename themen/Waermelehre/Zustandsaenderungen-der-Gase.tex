\subsubsection{Thermische Zustandsgleichung} 
\begin{minipage}{0.45\textwidth} 
\mainformular{ $ p\cdot V =\nu \cdot R_{m} \cdot T $} 
\end{minipage} 
\begin{minipage}{0.45\textwidth} 
 
\legende{ 
$p $ & Druck & $\si{\pascal} $ & $ \si{\newton\per\square\metre}$ \\ 
$V $ & Volumen & $\si{\cubic\meter} $ & $ $ \\ 
$\nu $ & Perioden-Umdrehungen & $ $ & $ $ \\ 
$R_{m} $ &  & $ $ & $ $ \\ 
$T $ & absolute Temperatur & $\si{\kelvin} $ & $ 273,15 \si{\kelvin} = 0 \si{\degree}$ \\ 
} 
\end{minipage} 
$ p\cdot V =\nu \cdot R_{m} \cdot T $ \textcolor{lightgray}{\textbf{---}} 
$ p =\frac{\nu \cdot R_{m} \cdot T}{  V} $ \textcolor{lightgray}{\textbf{---}} 
$ V =\frac{\nu \cdot R_{m} \cdot T}{  p} $ \textcolor{lightgray}{\textbf{---}} 
$ T =\frac{p\cdot V}{\nu \cdot R_{m} } $ \textcolor{lightgray}{\textbf{---}} 

\subsubsection{Allgemeine Gasgleichung} 
\begin{minipage}{0.45\textwidth} 
\mainformular{ $ V_{1}  = \frac{V_{2} \cdot p_{} \cdot T_{1} }{  T_{2} \cdot p_{1} } $} 
\end{minipage} 
\begin{minipage}{0.45\textwidth} 
 
\legende{ 
$V_{1} $ &  & $ $ & $ $ \\ 
$V_{2} $ &  & $ $ & $ $ \\ 
$p_{} $ &  & $ $ & $ $ \\ 
$T_{1} $ &  & $ $ & $ $ \\ 
$T_{2} $ &  & $ $ & $ $ \\ 
$p_{1} $ &  & $ $ & $ $ \\ 
} 
\end{minipage} 
$ V_{1}  = \frac{V_{2} \cdot p_{} \cdot T_{1} }{  T_{2} \cdot p_{1} } $ \textcolor{lightgray}{\textbf{---}} 
$ p_{1}  = \frac{V_{2} \cdot p_{2} \cdot T_{1} }{  T_{2} \cdot V_{1} } $ \textcolor{lightgray}{\textbf{---}} 
$ T_{1}  = \frac{V_{1} \cdot p_{1} \cdot T_{2} }{  V_{2} \cdot p_{2} } $ \textcolor{lightgray}{\textbf{---}} 

\subsubsection{Ideale Gasgleichung} 
\begin{minipage}{0.45\textwidth} 
\mainformular{ $\rho \cdot V = m \cdot R \cdot T$} 
\end{minipage} 
\begin{minipage}{0.45\textwidth} 
 
\legende{ 
$\rho $ & Dichte & $\si{\kilo\gram\per\cubic\metre} $ & $ $ \\ 
$V $ & Volumen & $\si{\cubic\meter} $ & $ $ \\ 
$m $ & Masse & $\si{\kilo\gram} $ & $ $ \\ 
$R $ &  & $ $ & $ $ \\ 
$T $ & absolute Temperatur & $\si{\kelvin} $ & $ 273,15 \si{\kelvin} = 0 \si{\degree}$ \\ 
} 
\end{minipage} 
$\rho \cdot V = m \cdot R \cdot T$ \textcolor{lightgray}{\textbf{---}} 

\subsubsection{Universelle Gaskonstante} 
\begin{minipage}{0.45\textwidth} 
\mainformular{ $ R_{m} = R \cdot M $} 
\end{minipage} 
\begin{minipage}{0.45\textwidth} 
 
\legende{ 
$R_{m} $ &  & $ $ & $ $ \\ 
$R $ &  & $ $ & $ $ \\ 
$M $ & Drehmoment & $\si{\newton\metre} $ & $ \si{\kilo\gram\square\metre\per\square\second}$ \\ 
} 
\end{minipage} 
$ R_{m} = R \cdot M $ \textcolor{lightgray}{\textbf{---}} 

\subsubsection{Isotherme Zustandsänderung} 
\begin{minipage}{0.45\textwidth} 
\mainformular{ $ W = -m \cdot R \cdot T \cdot \ln{\cfrac{V_2}{V_1}} $} 
\end{minipage} 
\begin{minipage}{0.45\textwidth} 
 
\legende{ 
$W $ &  & $ $ & $ $ \\ 
$m $ & Masse & $\si{\kilo\gram} $ & $ $ \\ 
$R $ &  & $ $ & $ $ \\ 
$T $ & absolute Temperatur & $\si{\kelvin} $ & $ 273,15 \si{\kelvin} = 0 \si{\degree}$ \\ 
$V_2 $ &  & $ $ & $ $ \\ 
$V_1 $ &  & $ $ & $ $ \\ 
} 
\end{minipage} 
$ W = -m \cdot R \cdot T \cdot \ln{\cfrac{V_2}{V_1}} $ \textcolor{lightgray}{\textbf{---}} 

