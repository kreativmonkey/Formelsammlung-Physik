\subsubsection{Dichte} 
\begin{minipage}{0.45\textwidth} 
$ \rho  = \frac{m}{V} $\\ 
$ m = \rho \cdot V $\\ 
$ V = \frac{m}{\rho } $\\ 
\end{minipage} 
\begin{minipage}{0.45\textwidth} 
 
\end{minipage} 
\subsubsection{Kräfte} 
\begin{minipage}{0.45\textwidth} 
\end{minipage} 
\begin{minipage}{0.45\textwidth} 
 
\end{minipage} 
\subsubsection{Auftrieb in Flüssigkeiten} 
\begin{minipage}{0.45\textwidth} 
$ F_{A}  =  \rho \cdot g\cdot V $\\ 
$ \rho  = \frac{F_{A} }{g\cdot V} $\\ 
$ V = \frac{F_{A} }{g \rho } $\\ 
\end{minipage} 
\begin{minipage}{0.45\textwidth} 
 
\end{minipage} 
\subsubsection{Druck} 
\begin{minipage}{0.45\textwidth} 
$ p = \frac{F}{A} $\\ 
$ F = p\cdot A $\\ 
$ A = \frac{F}{p} $\\ 
\end{minipage} 
\begin{minipage}{0.45\textwidth} 
 
\end{minipage} 
\subsubsection{Gewichtskraft} 
\begin{minipage}{0.45\textwidth} 
$ F_{G}  = m\cdot g $\\ 
$ m = \frac{F_{G} }{g} $\\ 
$ g = \frac{F_{G} }{m} $\\ 
\end{minipage} 
\begin{minipage}{0.45\textwidth} 
 
\end{minipage} 
\subsubsection{Hebelgesetz} 
\begin{minipage}{0.45\textwidth} 
$ F_{1}  = \frac{F_{2} \cdot l_{2} }{ l_{1} } $\\ 
$ l_{1}  = \frac{F_{2} \cdot l_{2} }{ F_{1} } $\\ 
\end{minipage} 
\begin{minipage}{0.45\textwidth} 
 
\end{minipage} 
\subsubsection{Schweredruck} 
\begin{minipage}{0.45\textwidth} 
$ p =  \rho \cdot g\cdot h $\\ 
$ \rho  = \frac{p}{g\cdot h} $\\ 
$ h = \frac{p}{g \rho } $\\ 
\end{minipage} 
\begin{minipage}{0.45\textwidth} 
 
\end{minipage} 
\subsubsection{Drehmoment} 
\begin{minipage}{0.45\textwidth} 
$ M = F\cdot l $\\ 
$ F = \frac{M}{l} $\\ 
$ l = \frac{M}{F} $\\ 
\end{minipage} 
\begin{minipage}{0.45\textwidth} 
 
\end{minipage} 
\subsubsection{Hookesches Gesetz} 
\begin{minipage}{0.45\textwidth} 
$ F = D\cdot s $\\ 
$ D = \frac{F}{s} $\\ 
$ s = \frac{F}{D} $\\ 
\end{minipage} 
\begin{minipage}{0.45\textwidth} 
 
\end{minipage} 
\subsubsection{Schiefe Ebene} 
\begin{minipage}{0.45\textwidth} 
$ F_{H}  = \frac{F_{G} \cdot h}{ l} $\\ 
$ F_{G}  = \frac{F_{H} \cdot l}{ h} $\\ 
$ h = \frac{F_{H} \cdot l}{ F_{G} } $\\ 
$ l = \frac{F_{G} \cdot h}{ F_{H} } $\\ 
$ F_{N}  = \frac{F_{G} \cdot b}{ l} $\\ 
$ F_{G}  = \frac{F_{N} \cdot l}{ b} $\\ 
$ b = \frac{F_{N} \cdot l}{ F_{G} } $\\ 
$ l = \frac{F_{G} \cdot b}{ F_{N} } $\\ 
\end{minipage} 
\begin{minipage}{0.45\textwidth} 
 
\end{minipage} 
\subsubsection{Reibung} 
\begin{minipage}{0.45\textwidth} 
$ F_{R}  = \mu \cdot F_{N} $\\ 
$ F_{N}  = \frac{F_{R} }{\mu } $\\ 
$ \mu  = \frac{F_{R} }{F_{N} } $\\ 
\end{minipage} 
\begin{minipage}{0.45\textwidth} 
 
\end{minipage} 
\subsubsection{Wichte} 
\begin{minipage}{0.45\textwidth} 
$ \gamma  = \frac{F_{G} }{V} $\\ 
$ F_{G}  = V\cdot \gamma $\\ 
$ V = \frac{F_{G} }{\gamma } $\\ 
\end{minipage} 
\begin{minipage}{0.45\textwidth} 
 
\end{minipage} 
