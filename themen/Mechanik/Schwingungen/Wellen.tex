//subsubsection{Bewegungsgleichung (harmonische Schwingung) - Aufgabe:HarmSchwUmys} 
$ y = y_{s} \cdot sin(\omega \cdot t+\phi _{0} ) $\ 
$ y_{s} = \frac{    y}{sin(\omega \cdot t+\phi _{0} )} $\ 
$ t = \frac{arcsin(y/y_{s} )-\phi _{0} }{    \omega } $\ 
//subsubsection{Bewegungsgleichung (harmonische Schwingung) - Aufgabe:HarmSchwUmy} 
$ y = y_{s} \cdot sin(\omega \cdot t+\phi _{0} ) $\ 
$ y_{s} = \frac{    y}{sin(\omega \cdot t+\phi _{0} )} $\ 
$ t = \frac{arcsin(y/y_{s} )-\phi _{0} }{    \omega } $\ 
//subsubsection{Bewegungsgleichung (harmonische Schwingung)} 
$ y = y_{s} \cdot sin(\omega \cdot t+\phi _{0} ) $\ 
$ y_{s} = \frac{    y}{sin(\omega \cdot t+\phi _{0} )} $\ 
$ t = \frac{arcsin(y/y_{s} )-\phi _{0} }{    \omega } $\ 
//subsubsection{Bewegungsgleichung (harmonische Schwingung) - Aufgabe:HarmSchwUmt} 
$ y = y_{s} \cdot sin(\omega \cdot t+\phi _{0} ) $\ 
$ y_{s} = \frac{    y}{sin(\omega \cdot t+\phi _{0} )} $\ 
$ t = \frac{arcsin(y/y_{s} )-\phi _{0} }{    \omega } $\ 
//subsubsection{Periodendauer (harmonische Schwingung) - Aufgabe:PeriodeUmm} 
$ T = 2\cdot \pi \cdot \sqrt{\frac{m}{D}} $\ 
$ D= m\cdot \frac{(2\cdot \pi )^{2} }{  T^{2} } $\ 
$ m= D\cdot \frac{ T^{2} }{(2\cdot \pi )^{2} } $\ 
//subsubsection{Periodendauer (harmonische Schwingung) - Aufgabe:PeriodeUmD} 
$ T = 2\cdot \pi \cdot \sqrt{\frac{m}{D}} $\ 
$ D= m\cdot \frac{(2\cdot \pi )^{2} }{  T^{2} } $\ 
$ m= D\cdot \frac{ T^{2} }{(2\cdot \pi )^{2} } $\ 
//subsubsection{Periodendauer (harmonische Schwingung)} 
$ T = 2\cdot \pi \cdot \sqrt{\frac{m}{D}} $\ 
$ D= m\cdot \frac{(2\cdot \pi )^{2} }{  T^{2} } $\ 
$ m= D\cdot \frac{ T^{2} }{(2\cdot \pi )^{2} } $\ 
//subsubsection{Lineares Kraftgesetz - Aufgabe:LineKrafUmD} 
$ F = -D\cdot y $\ 
$ D = \frac{-F}{y} $\ 
$ y = \frac{-F}{D} $\ 
//subsubsection{Lineares Kraftgesetz - Aufgabe:LineKrafUmF} 
$ F = -D\cdot y $\ 
$ D = \frac{-F}{y} $\ 
$ y = \frac{-F}{D} $\ 
//subsubsection{Lineares Kraftgesetz - Aufgabe:LineKrafUmy} 
$ F = -D\cdot y $\ 
$ D = \frac{-F}{y} $\ 
$ y = \frac{-F}{D} $\ 
//subsubsection{Lineares Kraftgesetz} 
$ F = -D\cdot y $\ 
$ D = \frac{-F}{y} $\ 
$ y = \frac{-F}{D} $\ 
//subsubsection{Periodendauer (harmonische Schwingung) - Aufgabe:PeriodeUmT} 
$ T = 2\cdot \pi \cdot \sqrt{\frac{m}{D}} $\ 
$ D= m\cdot \frac{(2\cdot \pi )^{2} }{  T^{2} } $\ 
$ m= D\cdot \frac{ T^{2} }{(2\cdot \pi )^{2} } $\ 
